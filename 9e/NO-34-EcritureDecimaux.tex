\documentclass[a4paper,11pt]{report}
\usepackage{frcursive}
\usepackage[showexo=true,showcorr=false]{../packages/coursclasse}
%Commenter ou enlever le commentaire sur la ligne suivante pour montrer le niveau
\toggletrue{montrerNiveaux}
%permet de gérer l'espacement entre les items des env enumerate et enumitem
\usepackage{enumitem}
\setlist[enumerate]{align=left,leftmargin=1cm,itemsep=10pt,parsep=0pt,topsep=0pt,rightmargin=0.5cm}
\setlist[itemize]{align=left,labelsep=1em,leftmargin=*,itemsep=0pt,parsep=0pt,topsep=0pt,rightmargin=0cm}
%permet de gerer l'espacement entre les colonnes de multicols
\setlength\columnsep{35pt}

\settasks{after-item-skip = 0.5em}

\begin{document}
%%%%%%%%%%%%%%%%% À MODIFIER POUR CHAQUE SERIE %%%%%%%%%%%%%%%%%%%%%%%%%%%%%
\newcommand{\chapterName}{Nombres et opérations}
\newcommand{\serieName}{Lire et écrire des nombres décimaux}


%%%%%%%%%%%%%%%%%% PREMIERE PAGE NE PAS MODIFER %%%%%%%%%%%%%%%%%%%%%%%%
% le chapitre en cours, ne pas changer au cours d'une série
\chapter*{\chapterName}
\thispagestyle{empty}

%%%%% LISTE AIDE MEMOIRE %%%%%%
\begin{amL}{\serieName}{
\item Chiffres et nombres (page 11)
\item Nombres naturels (page 12)
\item Nombres décimaux (page 21)
}
\end{amL}
%%%%%%%%%%%%%%% DEBUT DE LA SERIE NE PAS MODIFIER %%%%%%%%%%%%%%%%%%%%%%%%%%%%%
\section*{\serieName}
\setcounter{page}{1}
\thispagestyle{firstPage}



%%%%%%%%%%% LES EXERCICES %%%%%%%%%%%%%%%%%%%%%%%%%%%%%%%%%%%



\begin{resolu}{Nombres ou chiffres~?}{Complète  par {\color{blue}chiffre(s)} ou {\color{blue}nombre(s)}.
\begin{tasks}[after-item-skip = 0.5em]
\task 75 et 234 sont des {nombres}.
\task 1 et 2 sont les {\color{blue}chiffres} qui permettent d'écrire, par exemple,  les {\color{blue}nombres}  12 et 21. 
\task Les {\color{blue}nombres} 567 et 765 sont composés des {\color{blue}chiffres} 5 , 6 et 7.
\task Le {\color{blue}nombre} 7 est composé d'un seul {\color{blue}chiffre}.
\task Les {\color{blue}chiffres} 2 et 8 composent le {\color{blue}nombre} 28.
\task 9 est à la fois un {\color{blue}chiffre} et un {\color{blue}nombre}.
\end{tasks}}{1}
\end{resolu}

\begin{exop}{
Complète par {\color{blue}chiffre(s)} ou {\color{blue}nombre(s)}.

\begin{tasks}[after-item-skip = 0.5em]
\task Le $\;$\hrulefill$\;$ 432 est composé des $\;$\hrulefill$\;$ 2, 3 et 4.
\task Les $\;$\hrulefill$\;$ 9 et 7 composent le $\;$\hrulefill$\;$ 97.
\task Le $\;$\hrulefill$\;$ 501 est composé des $\;$\hrulefill$\;$ 5, 0 et 1.
\task Les $\;$\hrulefill$\;$ 8 et 2 composent le $\;$\hrulefill$\;$ 82.
\task 8 peut être à la fois un $\;$\hrulefill$\;$ et un $\;$\hrulefill$\;$.
\end{tasks}
}{1}\end{exop}

\begin{exop}{
Complète les phrases suivantes.
\begin{tasks}[after-item-skip = 0.5em]
\task Le $\;$\hrulefill$\;$ 678 est composé des $\;$\hrulefill$\;$ 6, 7 et 8.
\task Les $\;$\hrulefill$\;$ 2 et 8 composent le $\;$\hrulefill$\;$ 28.
\task Le $\;$\hrulefill$\;$ 981 est composé des $\;$\hrulefill$\;$ 9, 8 et 1.
\task Le $\;$\hrulefill$\;$ 4 et 5 composent le $\;$\hrulefill$\;$ 54.
\task 7 peut être à la fois un $\;$\hrulefill$\;$ et un $\;$\hrulefill$\;$.
\end{tasks}
}{1}\end{exop}

\begin{exo}{
Aline prétend que 6 est un chiffre mais aussi un nombre. Carlos lui rétorque que c'est faux car, dit-il, les nombres commencent à partir de dix. Qui a raison~: Carlos ou Aline~? Justifie ta réponse.
}{1}\end{exo}

\begin{resolu}{Écriture de nombres en chiffres}
{Écris les nombres suivants en chiffres en t'aidant du tableau ci-dessous si nécessaire.

\begin{tasks}[after-item-skip = 0.5em]
\task dix mille dix .
\task cinq millions deux cent quarante-sept mille neuf .
\task deux cent un milliards six cent trente-cinq mille cinquante-trois.
\task neuf cent trois virgule septante-quatre.
\task zéro virgule deux mille huit.
\task deux mille vingt-deux virgule zéro vingt-huit.
\end{tasks}
\begin{center}
\begin{tabular}{|c|c|c|c|c|c|c|c|c|c|c|c||c|c|c|c|c|c|}
\hline
\multicolumn{12}{|c||}{Partie entière} & \multicolumn{6}{|c|}{Partie décimale}\\
\hline
\multicolumn{3}{|c|}{milliards} & \multicolumn{3}{|c|}{millions} & \multicolumn{3}{|c|}{milliers} &
\multicolumn{3}{|c||}{unités} &  &  &  &   &  &  \\
\cline{1-12} 
{\rotatebox{90}{centaine de ...\ }} & {\rotatebox{90}{dizaine de ...\ }} &  {\rotatebox{90}{unités de ...\ }} &
{\rotatebox{90}{centaine de ...\ }} & {\rotatebox{90}{dizaine de ...\ }} &  {\rotatebox{90}{unités de ...\ }} &
{\rotatebox{90}{centaine de ...\ }} & {\rotatebox{90}{dizaine de ...\ }} &  {\rotatebox{90}{unités de ...\ }} &
{\rotatebox{90}{centaines\ }} & {\rotatebox{90}{dizaines\ }} &  {\rotatebox{90}{unités\ }} &
{\rotatebox{90}{dixièmes\ }} & {\rotatebox{90}{centièmes\ }} &  {\rotatebox{90}{millièmes\ }} &
{\rotatebox{90}{dix millièmes\ }} & {\rotatebox{90}{cent millièmes\ }} &  {\rotatebox{90}{millionièmes\ }} \\
\hline
  &  &  &  &  &  &  & {\cursive{{\color{blue}1}}}  & {\cursive{{\color{blue}0'}}}  & {\cursive{{\color{blue}0}}}  & {\cursive{{\color{blue}1}}}   & {\cursive{{\color{blue}0,}}} &  &  &  &  &  &  \\
\hline
  &  &  &  &  & {\cursive{{\color{blue}5'}}}  & {\cursive{{\color{blue}2}}}  & {\cursive{{\color{blue}4}}}  & {\cursive{{\color{blue}7'}}}   & {\cursive{{\color{blue}0}}}   & {\cursive{{\color{blue}0}}}  &{\cursive{{\color{blue}9,}}}   &  &  &  &  &  &  \\
\hline 
 {\cursive{{\color{blue}2}}} &{\cursive{{\color{blue}0}}}  & {\cursive{{\color{blue}1'}}}  &{\cursive{{\color{blue}0}}}  &  {\cursive{{\color{blue}0}}} & {\cursive{{\color{blue}0'}}}  & {\cursive{{\color{blue}6}}}  & {\cursive{{\color{blue}3}}} &  {\cursive{{\color{blue}5'}}}& {\cursive{{\color{blue}0}}}  &{\cursive{{\color{blue}5}}}  & {\cursive{{\color{blue}3,}}}  &  &  &  &  &  &  \\
\hline  
  &  &  &  &  &  &  &  &  & {\cursive{{\color{blue}9}}}  & {\cursive{{\color{blue}0}}} & {\cursive{{\color{blue}3,}}} & {\cursive{{\color{blue}7}}} & {\cursive{{\color{blue}4}}}  &  &  &  &  \\
\hline  
  &  &  &  &  &  &  &  &  &  &  & {\cursive{{\color{blue}0,}}} & {\cursive{{\color{blue}2}}}  & {\cursive{{\color{blue}0}}}  & {\cursive{{\color{blue}0}}}  & {\cursive{{\color{blue}8}}}  &  &  \\
\hline  
  &  &  &  &  &  &  &  &  {\cursive{{\color{blue}2'}}} &  {\cursive{{\color{blue}0}}}  &  {\cursive{{\color{blue}2}}} &  {\cursive{{\color{blue}2,}}} &  {\cursive{{\color{blue}0}}}  &  {\cursive{{\color{blue}2}}} &  {\cursive{{\color{blue}8}}}  &  &  &  \\
\hline 
 % &  &  &  &  &  &  &  &  &  &  &  &  &  &  &  &  &  \\
%\hline 
\end{tabular}
\end{center}
{\bf Voici donc les réponses~:}
{\color{blue}
\begin{tasks}[after-item-skip = 0.5em](2)
\task 10'010
\task 5'247'009
\task 201'000'635'053
\task 903,74
\task 0,2008
\task 2'022,028
\end{tasks}}
}{1}
\end{resolu}

\begin{exo}{ Écris un nombre, en lettres puis en chiffres, en utilisant une seul fois chacun des mots mille, cent(s), trente et cinq.
}{1}\end{exo}
\begin{exo}{ Écris les nombres suivants en chiffres.
\begin{tasks}[after-item-skip = 0.5em](2)
\task Cinquante-trois
\task Trente-six
\task Vingt-huit
\task Septante-cinq
\task Quarante-neuf
\task Nonante et un
\task Dix-sept
\task Trois cent vingt-deux
\task Quatre-vingt deux
\task Quarante-quatre
\task Mille trente-six
\task Quatorze
\end{tasks}

}{1}\end{exo}
\begin{exo}{ Écris les nombres suivants en chiffres.
\begin{tasks}[after-item-skip = 0.5em]
\task  sept cent septante sept mille septante 
\task cent dix-neuf millions quatre-vingt trois 
\task vingt-neuf mille nonante
\task cent huitante                             
\task dix-neuf cent quatre-vingt-dix-neuf 
\task soixante-douze mille septante-sept 
\task trois millions et demi
\end{tasks}
}{1}\end{exo}

\begin{exo}{ Écris les nombres suivants en chiffres.
\begin{tasks}[after-item-skip = 0.5em]
\task mille quatorze 
\task  trois mille trente-huit
\task deux mille quatre cent vingt-huit          
\task six cent mille cent quatre-vingts 
\task treize millions cent treize mille douze    
\task sept cent deux mille trois cent un 
\task cent un milliard cent mille vingt           
\task cinq cent cinquante-cinq mille cinq cents 
\task quatre-vingt mille cent quarante            
\end{tasks}
}{1}\end{exo}


\begin{exo}{ Écris les nombres suivants en lettres.
\begin{tasks}[after-item-skip = 0.5em](4)
\task 456     \task  36    \task 244     \task 801
\task 904     \task 372   \task 1210    \task 623
\task 3450    \task 5842 \task 33601  \task 57200
\task 600234 \task 13   \task 802032 \task 16230000
\task 236     \task 1024    \task 19000005     \task 316484
\task 6021349517     \task 470   \task 6732    \task 430015
\end{tasks}

}{1}\end{exo}

\exol{NO161}{48}{1}


\begin{exo}{ Nomme la position occupée par chacun des chiffres dans les nombres ci-dessous, en t'aidant du tableau à la fin de la série, si nécessaire.
\begin{tasks}[after-item-skip = 0.5em](2)
\task 3456
\task 2598
\task 25892164
\task115698
\end{tasks}

}{1}\end{exo}
\begin{exop}{ Pour chaque nombre ci-dessous, entoure en rouge le chiffre des dizaine de milliers.
\begin{tasks}[after-item-skip = 0.5em](3)
\task 43567
\task 1234
\task 89054
\task 333333
\task 4579230
\end{tasks}

}{1}\end{exop}
\begin{exop}{ Pour chaque nombre ci-dessous, entoure en rouge le chiffre des  unités et en bleu le chiffre des centaines.
\begin{tasks}[after-item-skip = 0.5em](3)
\task 43567
\task 1234
\task 89054
\task 333333
\task 4579230
\end{tasks}

}{1}\end{exop}


\begin{exo}{ Nomme la position occupée par chacun des chiffres dans les nombres ci-dessous.
\begin{tasks}[after-item-skip = 0.5em](3)
\task 7421
\task 1982
\task 1230932
\task 12084353
\end{tasks}

}{1}\end{exo}
\begin{exop}{ Pour chaque nombre ci-dessous, entoure en rouge le chiffre des dizaine de milliers.
\begin{tasks}[after-item-skip = 0.5em](3)
\task 235632
\task 345345
\task 23435434
\task 34545654
\task 11134535
\end{tasks}

}{1}\end{exop}
\begin{exop}{ Pour chaque nombre ci-dessous, entoure en rouge le chiffre des  unités et en bleu le chiffre des centaines.
\begin{tasks}[after-item-skip = 0.5em](3)
\task 3254
\task 120983
\task 892310
\task 4444444
\task 9080701
\end{tasks}

}{1}\end{exop}
\begin{exo}{ Sur un compteur électrique, on peut lire les nombres suivants.
\begin{tasks}[after-item-skip = 0.5em](2)
\task 000147312
\task 001020080
\task 4000000000001
\task 000082302
\task 1230000000
\task 00002030
\end{tasks}

Supprime les zéros inutiles.
}{1}\end{exo}

\begin{exop}{ Sur un compteur électrique, on peut lire les nombres suivants.
\begin{tasks}[after-item-skip = 0.5em](2)
\task 000000344
\task 0010234340
\task 400000000000002
\task 000034454000
\task 3454650000
\task 0000200300
\end{tasks}

Supprime les zéros inutiles.
}{1}\end{exop}


\begin{exop}{ Le professeur a dicté le nombre . "trois cent quatre millièmes" Entoure la bonne réponse.

		\[304000 \qquad ;\qquad 0,00304 \qquad ;\qquad 0,304 \qquad ;\qquad 3004,00\]

}{1}\end{exop}
\begin{exo}{ Réponds aux questions suivantes.
\begin{tasks}[after-item-skip = 0.5em]
\task Où doit-on ajouter ou supprimer un zéro pour que le nombre 3,0005 devienne plus petit que 3,0004~?
\task Où doit-on ajouter ou supprimer un zéro pour que le nombre 12,01 devient plus grand que 12,02~?
\end{tasks}
}{1}\end{exo}




\begin{exop}{
Pour chaque nombre ci-dessous, entoure en bleu la partie entière et en rouge la partie décimale.
\begin{tasks}[after-item-skip = 0.5em](4)
\task $51,\!15$
\task $22,\!643$
\task $102,\!001$
\task $17$
\task $0,0201$
\task $10,\!40801$
\task $98,\!99$
\task $3,\!2$
\task $2,\!3$
\task $0,\!00002$
\task $120,\!$
\task $0,\!1111$
\end{tasks}
}{1}\end{exop}




\begin{exop}{ 
\begin{tasks}[after-item-skip = 0.5em]
\task Dans le nombre 15'630,9
\begin{enumerate}
\item 156 est le nombre de \hrulefill
\item 1 est le nombre de  \hrulefill
\item 1563 est le nombre de \hrulefill
\item 15 est le nombre de  \hrulefill
\end{enumerate}
\task Dans le nombre 123,4
\begin{enumerate}
\item le nombre d'unités est~: \hrulefill
\item le nombre de centaines est~: \hrulefill
\item le nombre de dizaines  est~: \hrulefill
\item le nombre de milliers est~: \hrulefill
\end{enumerate}
\end{tasks}
}{1}\end{exop}

\newpage
\begin{exop}{
\begin{tasks}[after-item-skip = 0.5em]
\task Dans le nombre 4 091,807
\begin{enumerate}[itemsep=7pt]
\item 409 est le nombre de \hrulefill
\item 4091 807 est le nombre de     \hrulefill
\item 40 est le nombre de \hrulefill
\item 40918 est le nombre de  \hrulefill
\end{enumerate}
\task Dans le nombre 738,59
\begin{enumerate}[itemsep=7pt]
\item le nombre de dixièmes est~: \hrulefill
\item le nombre de centaines est~: \hrulefill
\item le nombre de centièmes  est~: \hrulefill
\item le nombre de millièmes est~: \hrulefill
\end{enumerate}
\end{tasks}
}{1}\end{exop}

\begin{exop}{
\begin{tasks}[after-item-skip = 0.5em]
\task Dans le nombre 472 613,95
\begin{enumerate}
\item 9 est le chiffre des \hrulefill
\item 7 est le chiffre  \hrulefill
\item 5 est le chiffre \hrulefill
\item 3 est le chiffre  \hrulefill
\end{enumerate}
\task Dans le nombre 129,8
\begin{enumerate}
\item le chiffre des dixièmes est~: \hrulefill
\item le chiffre des unités est~: \hrulefill
\item le chiffre des dizaines est~: \hrulefill
\item le chiffre des centaines est~: \hrulefill
\end{enumerate}
\end{tasks}
}{1}\end{exop}

\begin{exop}{
\begin{tasks}[after-item-skip = 0.5em]
\task Dans le nombre 124 738,59
\begin{enumerate}
\item 9 est le chiffre des \hrulefill
\item 7 est le chiffre  \hrulefill
\item 5 est le chiffre \hrulefill
\item 3 est le chiffre  \hrulefill
\end{enumerate}
\task Dans le nombre 84,735
\begin{enumerate}
\item le chiffre des dixièmes est~: \hrulefill
\item le chiffre des unités est~: \hrulefill
\item le chiffre des millièmes est~: \hrulefill
\item le chiffre des centaines est~: \hrulefill
\end{enumerate}
\end{tasks}
}{1}\end{exop}

\begin{exop}{
Pour chaque nombre ci-dessous, donne le {\bf nombre de dizaines}.
\begin{tasks}[after-item-skip = 0.5em](2)
\task 4'432~: \hrulefill
\task 21'092~: \hrulefill
\task 19'237~: \hrulefill
\task 754,2~: \hrulefill
\end{tasks}
}{1}\end{exop}
\begin{exop}{
Pour chaque nombre ci-dessous, donne le {\bf nombre de centaines de milliers}.
\begin{tasks}[after-item-skip = 0.5em](2)
\task 54'987~: \hrulefill
\task 409'710~: \hrulefill
\task 4'378'129~: \hrulefill
\task 8'987'397,7~: \hrulefill
\end{tasks}
}{1}\end{exop}
\begin{exop}{
Pour chaque nombre ci-dessous, donne le {\bf nombre d'unités}.
\begin{tasks}[after-item-skip = 0.5em](2)
\task 21,5~: \hrulefill
\task 2002,002~: \hrulefill
\task 120~: \hrulefill
\task 876,3~: \hrulefill
\end{tasks}
}{1}\end{exop}




\begin{exop}{
Donne l'écriture décimale des nombres
\begin{tasks}[before-skip=-0.7em]
\task Quinze unités et trois dixièmes~: \hrulefill
\task Six cent six unités et douze centièmes~: \hrulefill
\task Neuf unités et onze millièmes~: \hrulefill
\task Trois centaines et un dixième~: \hrulefill
\task Douze dizaines et quinze millièmes~: \hrulefill
\task Quatre dizaines et onze dixipmes~: \hrulefill
\end{tasks}
}{1}\end{exop}

\exol{NO160}{48}{1}


\begin{exop}{
Pour chaque nombre ci-dessous, donne le {\bf nombre de dizaines}.
\begin{tasks}[before-skip=-0.7em](2)
\task 3'659~: \hrulefill
\task 17'543~: \hrulefill
\task 20'173~: \hrulefill
\task 852,2~: \hrulefill
\end{tasks}
}{1}\end{exop}

\begin{exop}{
Pour chaque nombre ci-dessous, donne le {\bf nombre de centaines de milliers}.
\begin{tasks}[before-skip=-0.7em](2)
\task 12'345~: \hrulefill
\task 320'476~: \hrulefill
\task 3'976'932~: \hrulefill
\task 1'243'876,3~: \hrulefill
\end{tasks}
}{1}\end{exop}
\begin{exop}{
Pour chaque nombre ci-dessous, donne le {\bf nombre d'unités}.
\begin{tasks}[before-skip=-0.7em](2)
\task 13,5~: \hrulefill
\task 1001,001~: \hrulefill
\task 23~: \hrulefill
\task 853,32~: \hrulefill
\end{tasks}
}{1}\end{exop}
\begin{exop}{
Pour chaque nombre ci-dessous, donne le {\bf nombre de dixièmes}.
\begin{tasks}[before-skip=-0.7em](2)
\task 0,123~: \hrulefill
\task 5,43~: \hrulefill
\task 15,3~: \hrulefill
\task 25~: \hrulefill
\end{tasks}
}{1}\end{exop}

\begin{exop}{
Pour chaque nombre ci-dessous, donne le {\bf nombre de centièmes}.
\begin{tasks}[before-skip=-0.7em](2)
\task 0,12~: \hrulefill
\task 5,4~: \hrulefill
\task 4,3~: \hrulefill
\task 12~: \hrulefill
\end{tasks}

}{1}\end{exop}

\begin{exop}{
Pour chaque nombre ci-dessous, donne le {\bf chiffre des centièmes}.
\begin{tasks}[before-skip=-0.7em](2)
\task 0,12~: \hrulefill
\task 5,41~: \hrulefill
\task 4,013~: \hrulefill
\task 1,02~: \hrulefill
\end{tasks}


}{1}\end{exop}

	\begin{qmoodle}{Lire un nombre décimal}{2}{
	\begin{center}	
		Q-1

\includegraphics[scale=1]{media/qr/ndlend1}

\tiny{{https://edu.ge.ch/qr/ndlend1}}
\end{center}
	\begin{center}	
		Q-2

\includegraphics[scale=1]{media/qr/ndlend2}

\tiny{{https://edu.ge.ch/qr/ndlend2}}
\end{center}
}
\end{qmoodle}

	\begin{qmoodle}{Écrire un nombre décimal}{3}{
	\begin{center}	
		Q-1

\includegraphics[scale=1]{media/qr/ndlend3}

\tiny{{https://edu.ge.ch/qr/ndlend3}}
\end{center}
	\begin{center}	
		Q-2

\includegraphics[scale=1]{media/qr/ndlend4}

\tiny{{https://edu.ge.ch/qr/ndlend4}}
\end{center}
	\begin{center}	
		Q-3

\includegraphics[scale=1]{media/qr/ndlend5}

\tiny{{https://edu.ge.ch/qr/ndlend5}}
\end{center}
}
\end{qmoodle}


%27
\begin{exo}{
Dans le nombre 123 987, place la virgule  et/ ou le(s) zéro(s) si besoin pour que
\begin{tasks}[after-skip=-0.7em,before-skip=-0.7em, after-item-skip=0.18em](2)
\task 3 soit le chiffre des unités  
\task 7 soit le chiffre des dixièmes  
\task 3 soit le chiffre des dizaines 
\task 7 soit le chiffre des centaines 
%\task 1 soit le chiffre des dizaines 
\end{tasks}

}{1}\end{exo}



\begin{exo}{
Dans le nombre 314159, place la virgule  et/ ou le(s) zéro(s) , si besoin pour que
\begin{tasks}[after-skip=-0.7em,before-skip=-0.7em, after-item-skip=0.18em](2)
\task 4 soit le chiffre des unités.  
\task 5 soit le chiffre des dixièmes. 
\task 3 soit le chiffre des dizaines. 
\task 4 soit le chiffre des millièmes. 
\task 9 soit le chiffre des dizaines.
\task 9 soit le chiffre des centièmes.
\end{tasks}

}{1}\end{exo}

\begin{exo}{
\begin{tasks}[before-skip=-0.7em, after-item-skip=0.18em]
\task Quel est le nombre dont le chiffre des dizaines et des centièmes est 8, le chiffre des centaines et des dixièmes est 5 et tous les autres chiffres sont nuls~?
\task Donne un nombre dont le nombre de dizaines est 13 et le chiffre des dixièmes est 5.
\end{tasks}
}{1}\end{exo}

\exol{NO90}{31}{2}

\newpage

\begin{center}
\begin{tabular}{|c|c|c|c|c|c|c|c|c|c|c|c||c|c|c|c|c|c|}
\hline
\multicolumn{12}{|c||}{Partie entière} & \multicolumn{6}{|c|}{Partie décimale}\\
\hline
\multicolumn{3}{|c|}{milliards} & \multicolumn{3}{|c|}{millions} & \multicolumn{3}{|c|}{milliers} &
\multicolumn{3}{|c||}{unités} &  &  &  &   &  &  \\
\cline{1-12} 
{\rotatebox{90}{centaine de ...\ }} & {\rotatebox{90}{dizaine de ...\ }} &  {\rotatebox{90}{unités de ...\ }} &
{\rotatebox{90}{centaine de ...\ }} & {\rotatebox{90}{dizaine de ...\ }} &  {\rotatebox{90}{unités de ...\ }} &
{\rotatebox{90}{centaine de ...\ }} & {\rotatebox{90}{dizaine de ...\ }} &  {\rotatebox{90}{unités de ...\ }} &
{\rotatebox{90}{centaines\ }} & {\rotatebox{90}{dizaines\ }} &  {\rotatebox{90}{unités\ }} &
{\rotatebox{90}{dixièmes\ }} & {\rotatebox{90}{centièmes\ }} &  {\rotatebox{90}{millièmes\ }} &
{\rotatebox{90}{dix millièmes\ }} & {\rotatebox{90}{cent millièmes\ }} &  {\rotatebox{90}{millionièmes\ }} \\
\hline
  &  &  &  &  &  &  &   &   &   &    &  &  &  &  &  &  &  \\
\hline
  &  &  &  &  &  &  &   &   &   &    &  &  &  &  &  &  &  \\
\hline
  &  &  &  &  &  &  &   &   &   &    &  &  &  &  &  &  &  \\
\hline
  &  &  &  &  &  &  &   &   &   &    &  &  &  &  &  &  &  \\
\hline
  &  &  &  &  &  &  &   &   &   &    &  &  &  &  &  &  &  \\
\hline
  &  &  &  &  &  &  &   &   &   &    &  &  &  &  &  &  &  \\
\hline
  &  &  &  &  &  &  &   &   &   &    &  &  &  &  &  &  &  \\
\hline
  &  &  &  &  &  &  &   &   &   &    &  &  &  &  &  &  &  \\
\hline
  &  &  &  &  &  &  &   &   &   &    &  &  &  &  &  &  &  \\
\hline
  &  &  &  &  &  &  &   &   &   &    &  &  &  &  &  &  &  \\
\hline
  &  &  &  &  &  &  &   &   &   &    &  &  &  &  &  &  &  \\
\hline
  &  &  &  &  &  &  &   &   &   &    &  &  &  &  &  &  &  \\
\hline
  &  &  &  &  &  &  &   &   &   &    &  &  &  &  &  &  &  \\
\hline
  &  &  &  &  &  &  &   &   &   &    &  &  &  &  &  &  &  \\
\hline
  &  &  &  &  &  &  &   &   &   &    &  &  &  &  &  &  &  \\
\hline
  &  &  &  &  &  &  &   &   &   &    &  &  &  &  &  &  &  \\
\hline
  &  &  &  &  &  &  &   &   &   &    &  &  &  &  &  &  &  \\
\hline
  &  &  &  &  &  &  &   &   &   &    &  &  &  &  &  &  &  \\
\hline
  &  &  &  &  &  &  &   &   &   &    &  &  &  &  &  &  &  \\
\hline
  &  &  &  &  &  &  &   &   &   &    &  &  &  &  &  &  &  \\
\hline
  &  &  &  &  &  &  &   &   &   &    &  &  &  &  &  &  &  \\
\hline
  &  &  &  &  &  &  &   &   &   &    &  &  &  &  &  &  &  \\
\hline


 % &  &  &  &  &  &  &  &  &  &  &  &  &  &  &  &  &  \\
%\hline 
\end{tabular}
\end{center}

\newpage


\begin{center}
\begin{tabular}{|c|c|c|c|c|c|c|c|c|c|c|c||c|c|c|c|c|c|}
\hline
\multicolumn{12}{|c||}{Partie entière} & \multicolumn{6}{|c|}{Partie décimale}\\
\hline
\multicolumn{3}{|c|}{milliards} & \multicolumn{3}{|c|}{millions} & \multicolumn{3}{|c|}{milliers} &
\multicolumn{3}{|c||}{unités} &  &  &  &   &  &  \\
\cline{1-12} 
{\rotatebox{90}{centaine de ...\ }} & {\rotatebox{90}{dizaine de ...\ }} &  {\rotatebox{90}{unités de ...\ }} &
{\rotatebox{90}{centaine de ...\ }} & {\rotatebox{90}{dizaine de ...\ }} &  {\rotatebox{90}{unités de ...\ }} &
{\rotatebox{90}{centaine de ...\ }} & {\rotatebox{90}{dizaine de ...\ }} &  {\rotatebox{90}{unités de ...\ }} &
{\rotatebox{90}{centaines\ }} & {\rotatebox{90}{dizaines\ }} &  {\rotatebox{90}{unités\ }} &
{\rotatebox{90}{dixièmes\ }} & {\rotatebox{90}{centièmes\ }} &  {\rotatebox{90}{millièmes\ }} &
{\rotatebox{90}{dix millièmes\ }} & {\rotatebox{90}{cent millièmes\ }} &  {\rotatebox{90}{millionièmes\ }} \\
\hline
  &  &  &  &  &  &  &   &   &   &    &  &  &  &  &  &  &  \\
\hline
  &  &  &  &  &  &  &   &   &   &    &  &  &  &  &  &  &  \\
\hline
  &  &  &  &  &  &  &   &   &   &    &  &  &  &  &  &  &  \\
\hline
  &  &  &  &  &  &  &   &   &   &    &  &  &  &  &  &  &  \\
\hline
  &  &  &  &  &  &  &   &   &   &    &  &  &  &  &  &  &  \\
\hline
  &  &  &  &  &  &  &   &   &   &    &  &  &  &  &  &  &  \\
\hline
  &  &  &  &  &  &  &   &   &   &    &  &  &  &  &  &  &  \\
\hline
  &  &  &  &  &  &  &   &   &   &    &  &  &  &  &  &  &  \\
\hline
  &  &  &  &  &  &  &   &   &   &    &  &  &  &  &  &  &  \\
\hline
  &  &  &  &  &  &  &   &   &   &    &  &  &  &  &  &  &  \\
\hline
  &  &  &  &  &  &  &   &   &   &    &  &  &  &  &  &  &  \\
\hline
  &  &  &  &  &  &  &   &   &   &    &  &  &  &  &  &  &  \\
\hline
  &  &  &  &  &  &  &   &   &   &    &  &  &  &  &  &  &  \\
\hline
  &  &  &  &  &  &  &   &   &   &    &  &  &  &  &  &  &  \\
\hline
  &  &  &  &  &  &  &   &   &   &    &  &  &  &  &  &  &  \\
\hline
  &  &  &  &  &  &  &   &   &   &    &  &  &  &  &  &  &  \\
\hline
  &  &  &  &  &  &  &   &   &   &    &  &  &  &  &  &  &  \\
\hline
  &  &  &  &  &  &  &   &   &   &    &  &  &  &  &  &  &  \\
\hline
  &  &  &  &  &  &  &   &   &   &    &  &  &  &  &  &  &  \\
\hline
  &  &  &  &  &  &  &   &   &   &    &  &  &  &  &  &  &  \\
\hline
  &  &  &  &  &  &  &   &   &   &    &  &  &  &  &  &  &  \\
\hline
  &  &  &  &  &  &  &   &   &   &    &  &  &  &  &  &  &  \\
\hline


 % &  &  &  &  &  &  &  &  &  &  &  &  &  &  &  &  &  \\
%\hline 
\end{tabular}
\end{center}



\end{document}
