\documentclass[a4paper,11pt]{report}
\usepackage[showexo=true,showcorr=false]{../packages/coursclasse}
%Commenter ou enlever le commentaire sur la ligne suivante pour montrer le niveau
\toggletrue{montrerNiveaux}
%permet de gérer l'espacement entre les items des env enumerate et enumitem
\usepackage{enumitem}
\setlist[enumerate]{align=left,leftmargin=1cm,itemsep=10pt,parsep=0pt,topsep=0pt,rightmargin=0.5cm}
\setlist[itemize]{align=left,labelsep=1em,leftmargin=*,itemsep=0pt,parsep=0pt,topsep=0pt,rightmargin=0cm}
%permet de gerer l'espacement entre les colonnes de multicols
\setlength\columnsep{35pt}


% \newcount\segmentsleft
% \tikzset{pics/.cd,
%   circle fraction/.style args={#1/#2}{code={%
% \segmentsleft=#1\relax
% \pgfmathloop
% \ifnum\segmentsleft<1\else
% \ifnum\segmentsleft<#2 \edef\n{\the\segmentsleft}\else\def\n{#2}\fi
% \begin{scope}[shift={(\pgfmathcounter*0.85,0)}]
% \foreach \i [evaluate={\a=360/#2*(\i-1)+90;}] in {1,...,\n}
%   \fill[fill=gray!70] (0,0) -- (\a:3/8) arc (\a:\a+360/#2:3/8) -- cycle;
% \draw circle [radius=3/8];
% \ifnum#2>1
%   \foreach \i [evaluate={\a=360/#2*(\i-1);}] in {1,...,#2}
%     \draw (0,0) -- (90+\a:3/8);
% \fi
% \end{scope}
% \advance\segmentsleft by-#2
% \repeatpgfmathloop
%   }}
% }



    
\begin{document}
%%%%%%%%%%%%%%%%% À MODIFIER POUR CHAQUE SERIE %%%%%%%%%%%%%%%%%%%%%%%%%%%%%
\newcommand{\chapterName}{Nombres et opérations}
\newcommand{\serieName}{Dire, écrire et représenter les fractions}


%%%%%%%%%%%%%%%%%% PREMIERE PAGE NE PAS MODIFER %%%%%%%%%%%%%%%%%%%%%%%%
% le chapitre en cours, ne pas changer au cours d'une série
\chapter*{\chapterName}
\thispagestyle{empty}

%%%%% LISTE AIDE MEMOIRE %%%%%%
\begin{amL}{\serieName}{
\item Définition d'une fraction (page 27)
\item Passage d'une écriture fractionnaire à une écriture décimale (page 28)
\item Passer d'une écriture décimale à une écriture fractionnaire (page 28)
}
\end{amL}
%%%%%%%%%%%%%%% DEBUT DE LA SERIE NE PAS MODIFIER %%%%%%%%%%%%%%%%%%%%%%%%%%%%%
\section*{\serieName}
\setcounter{page}{1}
\thispagestyle{firstPage}



%%%%%%%%%%% LES EXERCICES %%%%%%%%%%%%%%%%%%%%%%%%%%%%%%%%%%%

\begin{exop}{
	Détermine à quelle fraction correspond chacune des parties grisées.
		\begin{tasks}(3)
		\task 

			\scalebox{2.5}{
		\begin{tikzpicture}
  %\node at (-1/2,0) {$\ldots\ldots\ldots$};
  \pic  at (0, 0) {circle fraction={3/4}};
\end{tikzpicture}}

\hrulefill
		\task 

			\scalebox{2.5}{
		\begin{tikzpicture}
  %\node at (-1/2,0) {$\ldots\ldots\ldots$};
  \pic  at (0, 0) {circle fraction={9/13}};
\end{tikzpicture}}

\hrulefill
		\task 

			\scalebox{2.5}{
		\begin{tikzpicture}
  %\node at (-1/2,0) {$\ldots\ldots\ldots$};
  \pic  at (0, 0) {circle fraction={7/10}};
\end{tikzpicture}}

\hrulefill
		\task 

			\scalebox{2.5}{
				\begin{tikzpicture}
  %\node at (-1/2,0) {$\ldots\ldots\ldots$};
  \pic  at (0, 0) {circle fraction={2/3}};
\end{tikzpicture}}

\hrulefill
		\task 

			\scalebox{2.5}{
				\begin{tikzpicture}
  %\node at (-1/2,0) {$\ldots\ldots\ldots$};
  \pic  at (0, 0) {circle fraction={5/8}};
\end{tikzpicture}}

\smallskip

\hrulefill

		\task

			\scalebox{2.5}{
				\begin{tikzpicture}
  %\node at (-1/2,0) {$\ldots\ldots\ldots$};
  \pic  at (0, 0) {circle fraction={4/5}};
\end{tikzpicture}}


\vspace{2pt} 

\hrulefill
\end{tasks}
}{1}\end{exop}


\begin{exop}{
	Détermine à quelle fraction correspond chacune des parties grisées.
		\begin{tasks}(2)
		\task 

			\scalebox{2.5}{
		\begin{tikzpicture}
  %\node at (-1/2,0) {$\ldots\ldots\ldots$};
  \pic  at (0, 0) {circle fraction={9/5}};
\end{tikzpicture}}

\hrulefill

		\task 


			\scalebox{2.5}{
		\begin{tikzpicture}
  %\node at (-1/2,0) {$\ldots\ldots\ldots$};
  \pic  at (0, 0) {circle fraction={7/3}};
\end{tikzpicture}}

\hrulefill
		\task 

			\scalebox{2.5}{
		\begin{tikzpicture}
  %\node at (-1/2,0) {$\ldots\ldots\ldots$};
  \pic  at (0, 0) {circle fraction={6/2}};
\end{tikzpicture}}

\hrulefill
		\task 

			\scalebox{2.5}{
				\begin{tikzpicture}
  %\node at (-1/2,0) {$\ldots\ldots\ldots$};
  \pic  at (0, 0) {circle fraction={11/10}};
\end{tikzpicture}}

\hrulefill
		\task 

			\scalebox{2.5}{
				\begin{tikzpicture}
  %\node at (-1/2,0) {$\ldots\ldots\ldots$};
  \pic  at (0, 0) {circle fraction={3/1}};
\end{tikzpicture}}

\hrulefill
		\task 

			\scalebox{2.5}{
				\begin{tikzpicture}
  %\node at (-1/2,0) {$\ldots\ldots\ldots$};
  \pic  at (0, 0) {circle fraction={7/4}};
\end{tikzpicture}}

\hrulefill
	\end{tasks}
}{1}\end{exop}

\begin{exop}{
	Détermine la fraction représentée par les parties grisées.
	\begin{tasks}(2)
		\task 
			
			$\fracrect[xscale=4]{3}{4}$

			\hrulefill

		\task 
			
			$\fracrect[xscale=4]{5}{8}$

\hrulefill

		\task 
			
			$\fracrect[xscale=4]{6}{6}$	
			\vspace{2pt}

		\hspace{24pt}$\fracrect[xscale=4]{4}{6}$

\hrulefill

		\task 
			
			$\fracrect[xscale=4]{7}{11}$

\hrulefill

		\task 
			
			$\fracrect[xscale=4]{3}{10}$

\hrulefill

		\task 
			
			$\emptyfracrect[xscale=4]{9}$

\hrulefill

\vspace{1cm}
\phantom{test}

	\end{tasks}
}{1}\end{exop}


\begin{exol}{NO168}{50}{1}
\end{exol}
\begin{exof}{NO175}{62}{1}
\end{exof}
\begin{exof}{NO176}{63}{3}
\end{exof}
\vfill
\begin{exof}{NO177}{64}{2}
\end{exof}
\vfill
\begin{exol}{NO164}{48}{1}
\end{exol}
\vfill
\begin{exol}{NO165}{49}{1}
\end{exol}
\vfill
\begin{exol}{NO166}{49}{2}
\end{exol}

\vfill
\begin{exo}{
Représente graphiquement les fractions suivantes.
	\begin{tasks}(3)
		\task $\dfrac{3}{4}$
		\task $\dfrac{5}{6}$
		\task $\dfrac{4}{10}$
		\task $\dfrac{2}{5}$
		\task $\dfrac{7}{9}$
		\task $\dfrac{1}{3}$
	\end{tasks}
 \vspace{1pt}
}{1}\end{exo}

\vfill
\begin{exo}{
Représente graphiquement les fractions suivantes.
	\begin{tasks}(3)
		\task $\dfrac{3}{2}$
		\task $\dfrac{12}{20}$
		\task $\dfrac{3}{7}$
		\task $\dfrac{14}{15}$
		\task $\dfrac{20}{9}$
		\task $\dfrac{13}{18}$
	\end{tasks}
 \vspace{1pt}
}{1}\end{exo}

\vfill
\begin{exol}{NO172}{50}{1}
\end{exol}
\vfill
\begin{exol}{NO167}{49}{1}
\end{exol}
\vfill
\begin{exol}{NO212}{55}{1}
\end{exol}
\vfill
\begin{exol}{NO213}{56}{2}
\end{exol}

\newpage
\begin{exop}{
Écris les nombres suivants sous forme de fraction.
\begin{tasks}(2)
    \task un quart: \hrulefill
    \task trois huitièmes: \hrulefill
    \task neuf cinquièmes: \hrulefill
    \task deux tiers: \hrulefill
    \task six demis: \hrulefill
    \task quarante centièmes: \hrulefill
\end{tasks}
 \vspace{1pt}
}{1}\end{exop}


\begin{exop}{
Écris les nombres suivants sous forme de fraction.
\begin{tasks}(2)
    \task dix-neuf vingtièmes: \hrulefill
    \task deux: \hrulefill
    \task vingt quarts: \hrulefill
    \task cinq tiers: \hrulefill
    \task sept: \hrulefill
    \task huit demis: \hrulefill
\end{tasks}
 \vspace{1pt}
}{1}\end{exop}

\begin{exop}{
Écris les nombres suivants en lettres.
\begin{tasks}(2)
    \task $\dfrac{13}{7}$: \hrulefill
    \task $\dfrac{3}{2}$: \hrulefill
    \task $\dfrac{4}{5}$: \hrulefill
    \task $\dfrac{2}{4}$: \hrulefill
    \task $\dfrac{7}{3}$: \hrulefill
    \task $\dfrac{9}{10}$: \hrulefill
\end{tasks}
 \vspace{1pt}

}{1}\end{exop}

\begin{exop}{
Écris les nombres suivants en lettres.
\begin{tasks}(2)
    \task $\dfrac{9}{5}$: \hrulefill
    \task $\dfrac{10}{8}$: \hrulefill
    \task $\dfrac{1}{2}$: \hrulefill
    \task $\dfrac{3}{4}$: \hrulefill
    \task $\dfrac{6}{9}$: \hrulefill
    \task $\dfrac{5}{3}$: \hrulefill
\end{tasks}
 \vspace{1pt}

}{1}\end{exop}

\vfill

\begin{exo}{
	Détermine le nombre décimal correspondant aux fractions suivantes.
	\begin{tasks}(3)
		\task $\dfrac{3}{10}$
		\task $\dfrac{4}{100}$
		\task $\dfrac{56}{10}$
		\task $\dfrac{82}{100}$
		\task $\dfrac{7}{1}$
		\task $\dfrac{403}{100}$
	\end{tasks}
 \vspace{1pt}
}{1}\end{exo}


\vfill


\begin{exol}{NO190}{52}{2}
\end{exol}


\vfill

\begin{exo}{
	Détermine le nombre décimal correspondant aux fractions suivantes.
	\begin{tasks}(3)
		\task $\dfrac{370}{10}$
		\task $\dfrac{91}{1000}$
		\task $\dfrac{218}{10000}$
		\task $\dfrac{111}{10}$
		\task $\dfrac{2020}{10}$
		\task $\dfrac{35311}{100000}$
	\end{tasks}
 \vspace{1pt}
}{1}\end{exo}


\vfill

\begin{qmun}{De l'écriture fractionnaire à l'écriture décimale}{
		\begin{center}
\includegraphics[scale=1]{media/qr/derfq1}

\tiny{{https://edu.ge.ch/qr/derfq1}}
		\end{center}
	}
\end{qmun}


\vfill

\newpage

\begin{exo}{
	Détermine la fraction décimale correspondant aux nombres suivants.
	\begin{tasks}[after-item-skip=0.2em, after-skip=-0.5em](3)
		\task $0,13$
		\task $0,4$
		\task $0,752$
		\task $0,5353$
		\task $0,14$
		\task $0,33$
	\end{tasks}
 \vspace{1pt}
}{1}\end{exo}


\begin{exo}{
	Détermine la fraction décimale correspondant aux nombres suivants.
	\begin{tasks}[after-item-skip=0.2em](3)
		\task $7,942$
		\task $1,32$
		\task $11,7$
		\task $980,01$
		\task $97,361$
		\task $0,920101$
	\end{tasks}
 \vspace{1pt}
}{2}\end{exo}

\begin{resolu}{De l'écriture fractionnaire à décimale}
{Écris les nombres suivants sous forme décimale.

{\color{blue} Afin de déterminer l'écriture décimale d'une fraction, il faut calculer le quotient de la fraction.}

\begin{tasks}(2)
    \task $\dfrac{3}{4}={{\color{blue}3\div 4=0,75}}$
    \task $\dfrac{5}{10}={{\color{blue}5\div 10=0,5}}$
    \task $\dfrac{10}{8}={{\color{blue}10\div 8=1,25}}$
    \task $\dfrac{8}{5}={{\color{blue}8\div 5=1,6}}$
\end{tasks}
 \vspace{1pt}
}{1}
\end{resolu}

\begin{exop}{
Écris les nombres suivants sous forme décimale.
\begin{tasks}(2)
	\task $\dfrac{4}{5}=\hrulefill$\hspace{0.5cm}
    \task $\dfrac{30}{25}=\hrulefill$
    \task $\dfrac{1}{2}=\hrulefill$\hspace{0.5cm}
    \task $\dfrac{8}{10}=\hrulefill$
    \task $\dfrac{3}{15}=\hrulefill$\hspace{0.5cm}
    \task $\dfrac{8}{5}=\hrulefill$
\end{tasks}
 \vspace{1pt}

}{1}\end{exop}

\begin{exo}{
Écris les nombres suivants dans leur forme périodique.
\begin{tasks}(2)
    \task $\dfrac{2}{3}=\hrulefill$\hspace{0.5cm}
    \task $\dfrac{2}{9}=\hrulefill$
    \task $\dfrac{7}{11}=\hrulefill$\hspace{0.5cm}
    \task $\dfrac{5}{9}=\hrulefill$
\end{tasks}
 \vspace{1pt}

}{2}\end{exo}

\begin{exo}{
Écris les nombres suivants sous forme décimale ou périodique.
\begin{tasks}(2)
    \task $\dfrac{1}{3}=\hrulefill$\hspace{0.5cm}
    \task $\dfrac{5}{2}=\hrulefill$
    \task $\dfrac{6}{100}=\hrulefill$\hspace{0.5cm}
    \task $\dfrac{10}{4}=\hrulefill$
    \task $\dfrac{8}{9}=\hrulefill$\hspace{0.5cm}
    \task $\dfrac{9}{7}=\hrulefill$
\end{tasks}
 \vspace{1pt}

}{3}\end{exo}


\begin{exol}{NO192}{53}{1}
\end{exol}

\begin{exo}{
	Détermine la longueur de la période et la période des nombres périodiques suivants.
	\begin{tasks}[after-item-skip=0.2em, after-skip=-0.5em](3)
		\task $0,\overline{3}$
		\task $0,\overline{32}$
		\task $0,\overline{63717171}$
		\task $1,3\overline{4}$
		\task $2,346346346\ldots$
		\task $5,1\overline{5}$
	\end{tasks}
 \vspace{1pt}
}{1}\end{exo}

\begin{exo}{
	Détermine si les nombres suivants sont périodiques. Si c'est le cas, indique la longueur de la période et la période du nombre en question.
	\begin{tasks}[after-item-skip=0.2em, after-skip=-0.5em](3)
		\task $14,12\overline{1}$
		\task $134,875875875$
		\task $76,\overline{54}$
		\task $1,434343\ldots$
		\task $0,3333333$
		\task $1,\overline{43}$
	\end{tasks}
 \vspace{1pt}
}{2}\end{exo}



\begin{exop}{
Détermine l'écriture décimale des nombres suivants.
\begin{tasks}(2)
    \task $\dfrac{7}{3}=\hrulefill$\hspace{0.5cm}
    \task $\dfrac{8}{9}=\hrulefill$
    \task $\dfrac{4}{11}=\hrulefill$\hspace{0.5cm}
    \task $\dfrac{11}{15}=\hrulefill$
    \task $\dfrac{19}{33}=\hrulefill$\hspace{0.5cm}
    \task $\dfrac{3}{7}=\hrulefill$
\end{tasks}
 \vspace{1pt}

}{3}\end{exop}


\begin{qmun}{De l'écriture décimale à l'écriture fractionnaire}{
		\begin{center}
\includegraphics[scale=1]{media/qr/derfq2}

\tiny{{https://edu.ge.ch/qr/derfq2}}
		\end{center}
	}
\end{qmun}

%\exof{NO169}{58}{1}
%\exof{NO170}{58}{2}
%\exof{NO171}{59}{1}

\end{document}
