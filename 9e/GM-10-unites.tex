\documentclass[a4paper,11pt]{report} 
\usepackage[showexo=true,showcorr=false]{../packages/coursclasse}

\toggletrue{montrerNiveaux}

\usepackage{enumitem} 
\setlist[enumerate]{align=left,leftmargin=1cm,itemsep=10pt,parsep=0pt,topsep=0pt,rightmargin=0.5cm}
\setlist[itemize]{align=left,labelsep=1em,leftmargin=*,itemsep=0pt,parsep=0pt,topsep=0pt,rightmargin=0cm}

\setlength\columnsep{20pt} 
\begin{document} 

\newcommand{\chapterName}{Grandeurs et Mesures}
\newcommand{\serieName}{Transformation d'unités}

\chapter*{\chapterName}
\thispagestyle{empty}

\begin{amL}{\serieName}{
\item Unités de longueur (page 156)
\item Unités d'aire (page 157)
\item Unités de volume et de capacité (page 158)
\item Unités de masse (page 160)
}
\end{amL}
\section*{\serieName}
\setcounter{page}{1}
\thispagestyle{firstPage}





%Connaissance et utilisation des règles et conventions usuelles d'écriture algébrique (Niv 2s-3s)
%Détermination de la valeur numérique d'une expression littérale (, 4x +5, abc, x3 …) en substituant des nombres aux lettres (Niv 2-3)
%Élaboration d'expressions littérales à partir de figures géométriques ou d'expressions verbales (Niv 2s-3s). • d’énoncés de problèmes, de figures géométriques ou d’expressions verbales.La répartition de cette progression suit celle des expressions étudiées

\begin{QSJ}{183}{1}
\end{QSJ}

%unité de longueur
\begin{resolu}{Unité de longueur avec des entiers 1}
{À l'aide du tableau, transforme dans l'unité indiquée.

	\begin{tasks}[after-item-skip=2em](2)
   \task $\tunit{35}{\m}= \tunit{3500}{\centi  m}$
    \task $\tunit{450}{\deca m} = \tunit{4500000}{\milli m}$ 
\end{tasks}
\bgroup
\def\arraystretch{0.9}
\begin{center}
\begin{tabular}{*{7}{|x{1.5cm}}|}
 \hline
   km & hm & dam &m &dm &cm&mm\\   
   \hhline{*{7}{|=}|}
 &   & 3& 5&0 & 0& \\ 
 \hline
4 & 5  & 0& 0&0 &0 &0 \\
\hline
\end{tabular}
\end{center}
\egroup
\vspace{-0.8cm}
}
{1}
\end{resolu}


\begin{exop}
{À l'aide du tableau, transforme dans l'unité indiquée.
\begin{tasks}[after-item-skip=0.2em](2)
    \task \tunit{26}{\m}= \dotfill \hspace{0.3cm}
    \tunit{}{\centi  m} \hspace{0.3cm}
    \task \tunit{12}{\centi m} = \dotfill \hspace{0.3cm}
    \tunit{}{\milli  m} \hspace{0.3cm}
  \task \tunit{356}{\m}= \dotfill \hspace{0.3cm}
    \tunit{}{\milli  m} \hspace{0.3cm}
    \task \tunit{78}{\hecto m}= \dotfill \hspace{0.3cm}
    \tunit{}{\m} \hspace{0.3cm}
     \task \tunit{60}{\deca m}= \dotfill \hspace{0.3cm}
    \tunit{}{\deci  m} \hspace{0.3cm}
     \task \tunit{1}{\kilo m}= \dotfill \hspace{0.3cm}
    \tunit{}{\deca  m} \hspace{0.3cm}
     \task \tunit{32}{\deci m}= \dotfill \hspace{0.3cm}
    \tunit{}{\milli  m} \hspace{0.3cm}
     \task \tunit{5}{\kilo m}= \dotfill \hspace{0.3cm}
    \tunit{}{\deci  m} \hspace{0.3cm}
\end{tasks}
\begin{center}
\bgroup
\def\arraystretch{0.9}
\begin{tabular}{*{7}{|x{1.5cm}}|}
 \hline
   km & hm & dam &m &dm &cm&mm\\   
   \hhline{*{7}{|=}|}
 &   & & & & & \\ 
 \hline

 &   & & & & & \\
\hline
 &   & & & & & \\
\hline
 &   & & & & & \\
\hline
 &   & & & & & \\
\hline
 &   & & & & & \\
\hline
 &   & & & & & \\
\hline
 &   & & & & & \\
\hline

\end{tabular}
\egroup
\end{center}
}
{1}
\end{exop}
    
\begin{resolu}{Unité de longueur avec des entiers 2}
{À l'aide du tableau, transforme dans l'unité indiquée.

\begin{tasks}(2)
    \task $\tunit{35}{\deci m}= \tunit{0,35}{\deca  m}$
    \task $\tunit{450}{\deca m} = \tunit{4,5}{\kilo m}$
    
\end{tasks}
\begin{center}


\begin{tabular}{*{7}{|x{1.5cm}}|}
 \hline
   km & hm & dam &m &dm &cm&mm\\   
   \hhline{*{7}{|=}|}
 &   & 0& 3&5 & & \\ 
 \hline

4 & 5  & 0& & & & \\
\hline

\end{tabular}
\end{center}}
{1}
\end{resolu}


\begin{exop}
{À l'aide du tableau, transforme dans l'unité indiquée.
\begin{tasks}(2)
    \task \tunit{26}{\m}= \dotfill \hspace{0.3cm}
    \tunit{}{\deca  m} \hspace{0.3cm}
    \task \tunit{12}{\hecto m} = \dotfill \hspace{0.3cm}
    \tunit{}{\kilo  m} \hspace{0.3cm}
  \task \tunit{356}{\m}= \dotfill \hspace{0.3cm}
    \tunit{}{\kilo  m} \hspace{0.3cm}
    \task \tunit{78}{\deci m}= \dotfill \hspace{0.3cm}
    \tunit{}{\kilo m} \hspace{0.3cm}
     \task \tunit{60}{\deca m}= \dotfill \hspace{0.3cm}
    \tunit{}{\hecto  m} \hspace{0.3cm}
     \task \tunit{11}{\centi m}= \dotfill \hspace{0.3cm}
    \tunit{}{\deci  m} \hspace{0.3cm}
     \task \tunit{32}{\deci m}= \dotfill \hspace{0.3cm}
    \tunit{}{\deca  m} \hspace{0.3cm}
     \task \tunit{567}{\m}= \dotfill \hspace{0.3cm}
    \tunit{}{\deca  m} \hspace{0.3cm}
\end{tasks}
\begin{center}


\begin{tabular}{*{7}{|x{1.5cm}}|}
 \hline
   km & hm & dam &m &dm &cm&mm\\   
   \hhline{*{7}{|=}|}
 &   & & & & & \\ 
 \hline

 &   & & & & & \\
\hline
 &   & & & & & \\
\hline
 &   & & & & & \\
\hline
 &   & & & & & \\
\hline
 &   & & & & & \\
\hline
 &   & & & & & \\
\hline
 &   & & & & & \\
\hline

\end{tabular}
\end{center}
}
{1}
\end{exop}


\begin{exop}
{À l'aide du tableau, transforme dans l'unité indiquée.
\begin{tasks}(2)
    \task \tunit{43}{\deci m}= \dotfill \hspace{0.3cm}
    \tunit{}{\deca  m} \hspace{0.3cm}
    \task \tunit{37}{\kilo m} = \dotfill \hspace{0.3cm}
    \tunit{}{\m} \hspace{0.3cm}
  \task \tunit{127}{\m}= \dotfill \hspace{0.3cm}
    \tunit{}{\kilo  m} \hspace{0.3cm}
    \task \tunit{14}{\hecto m}= \dotfill \hspace{0.3cm}
    \tunit{}{\centi m} \hspace{0.3cm}
     \task \tunit{60}{\deci m}= \dotfill \hspace{0.3cm}
    \tunit{}{\hecto  m} \hspace{0.3cm}
     \task \tunit{11}{\centi m}= \dotfill \hspace{0.3cm}
    \tunit{}{\milli  m} \hspace{0.3cm}
     \task \tunit{32}{\centi m}= \dotfill \hspace{0.3cm}
    \tunit{}{\deca  m} \hspace{0.3cm}
     \task \tunit{567}{\deci m}= \dotfill \hspace{0.3cm}
    \tunit{}{\milli  m} \hspace{0.3cm}
\end{tasks}
\begin{center}

\begin{tabular}{*{7}{|x{1.5cm}}|}
 \hline
   km & hm & dam &m &dm &cm&mm\\   
   \hhline{*{7}{|=}|}
 &   & & & & & \\ 
 \hline

 &   & & & & & \\
\hline
 &   & & & & & \\
\hline
 &   & & & & & \\
\hline
 &   & & & & & \\
\hline
 &   & & & & & \\
\hline
 &   & & & & & \\
\hline
 &   & & & & & \\
\hline

\end{tabular} 
\end{center}
}
{1}
\end{exop}

\begin{exop}
{Transforme dans l'unité indiquée.
\begin{tasks}(2)
    \task \tunit{7}{\milli m}= \dotfill \hspace{0.3cm}
    \tunit{}{\deca  m} \hspace{0.3cm}
    \task \tunit{64}{\kilo m} = \dotfill \hspace{0.3cm}
    \tunit{}{\m} \hspace{0.3cm}
  \task \tunit{43}{\m}= \dotfill \hspace{0.3cm}
    \tunit{}{\deci  m} \hspace{0.3cm}
    \task \tunit{120}{\milli m}= \dotfill \hspace{0.3cm}
    \tunit{}{\centi m} \hspace{0.3cm}
     \task \tunit{42}{\deca m}= \dotfill \hspace{0.3cm}
    \tunit{}{\hecto  m} \hspace{0.3cm}
     \task \tunit{8}{\centi m}= \dotfill \hspace{0.3cm}
    \tunit{}{\milli  m} \hspace{0.3cm}
     \task \tunit{65}{\m}= \dotfill \hspace{0.3cm}
    \tunit{}{\deca  m} \hspace{0.3cm}
     \task \tunit{9}{\kilo m}= \dotfill \hspace{0.3cm}
    \tunit{}{\milli  m} \hspace{0.3cm}
\end{tasks}
}
{1}
\end{exop}


\begin{resolu}{Unité de longueur avec des décimaux 1}
{À l'aide du tableau, transforme dans l'unité indiquée.

\begin{tasks}(2)
    \task $\tunit{3,5}{\m}= \tunit{350}{\centi  m}$
    \task $\tunit{4,5}{\kilo m} = \tunit{4500}{\m}$
    
\end{tasks}
\begin{center}


\begin{tabular}{*{7}{|x{1.5cm}}|}
 \hline
   km & hm & dam &m &dm &cm&mm\\   
   \hhline{*{7}{|=}|}
 &   & & 3&5 & 0& \\ 
 \hline

4 & 5  & 0&0 & & & \\
\hline

\end{tabular}
\end{center}
}
{1}
\end{resolu}


\begin{exop}
{À l'aide du tableau, transforme dans l'unité indiquée.
\begin{tasks}(2)
    \task \tunit{4,3}{\deci m}= \dotfill \hspace{0.3cm}
    \tunit{}{\centi  m} \hspace{0.3cm}
    \task \tunit{3,7}{\hecto m} = \dotfill \hspace{0.3cm}
    \tunit{}{\m} \hspace{0.3cm}
  \task \tunit{12,7}{\m}= \dotfill \hspace{0.3cm}
    \tunit{}{\milli  m} \hspace{0.3cm}
    \task \tunit{1,45}{\hecto m}= \dotfill \hspace{0.3cm}
    \tunit{}{\deca m} \hspace{0.3cm}
     \task \tunit{0,6}{\deca m}= \dotfill \hspace{0.3cm}
    \tunit{}{\centi  m} \hspace{0.3cm}
     \task \tunit{1,1}{\m}= \dotfill \hspace{0.3cm}
    \tunit{}{\deci  m} \hspace{0.3cm}
     \task \tunit{3,2}{\deca m}= \dotfill \hspace{0.3cm}
    \tunit{}{\deci  m} \hspace{0.3cm}
     \task \tunit{5,67}{\kilo m}= \dotfill \hspace{0.3cm}
    \tunit{}{\milli  m} \hspace{0.3cm}
\end{tasks}
\begin{center}


\begin{tabular}{*{7}{|x{1.5cm}}|}
 \hline
   km & hm & dam &m &dm &cm&mm\\   
   \hhline{*{7}{|=}|}
 &   & & & & & \\ 
 \hline

 &   & & & & & \\
\hline
 &   & & & & & \\
\hline
 &   & & & & & \\
\hline
 &   & & & & & \\
\hline
 &   & & & & & \\
\hline
 &   & & & & & \\
\hline
 &   & & & & & \\
\hline

\end{tabular}
\end{center}
}
{1}
\end{exop}



\begin{resolu}{Unité de longueur avec des décimaux 2}
{À l'aide du tableau, transforme dans l'unité indiquée.

\begin{tasks}(2)
  \task $\tunit{3,5}{\m}= \tunit{0,35}{\deca  m}$
    \task $\tunit{4,5}{\centi m} = \tunit{0,45}{\deci m}$
    
\end{tasks}
\begin{center}


\begin{tabular}{*{7}{|x{1.5cm}}|}
 \hline
   km & hm & dam &m &dm &cm&mm\\   
   \hhline{*{7}{|=}|}
 &   &0 & 3&5 & & \\ 
 \hline

 &  & & & 0& 4&5 \\
\hline

\end{tabular}
\end{center}
}
{1}
\end{resolu}


\begin{exop}
{À l'aide du tableau, transforme dans l'unité indiquée.
\begin{tasks}(2)
    \task \tunit{14,3}{\deci m}= \dotfill \hspace{0.3cm}
    \tunit{}{\deca m} \hspace{0.3cm}
    \task \tunit{5,7}{\hecto m} = \dotfill \hspace{0.3cm}
    \tunit{}{\kilo m} \hspace{0.3cm}
  \task \tunit{12,7}{\m}= \dotfill \hspace{0.3cm}
    \tunit{}{\kilo  m} \hspace{0.3cm}
    \task \tunit{12,45}{\deci m}= \dotfill \hspace{0.3cm}
    \tunit{}{\deca m} \hspace{0.3cm}
     \task \tunit{0,12}{\deca m}= \dotfill \hspace{0.3cm}
    \tunit{}{\hecto m} \hspace{0.3cm}
     \task \tunit{0,1}{\m}= \dotfill \hspace{0.3cm}
    \tunit{}{\hecto m} \hspace{0.3cm}
     \task \tunit{9,2}{\deca m}= \dotfill \hspace{0.3cm}
    \tunit{}{\hecto m} \hspace{0.3cm}
     \task \tunit{43,67}{\ m}= \dotfill \hspace{0.3cm}
    \tunit{}{\kilo m} \hspace{0.3cm}
\end{tasks}
\begin{center}


\begin{tabular}{*{7}{|x{1.5cm}}|}
 \hline
   km & hm & dam &m &dm &cm&mm\\   
   \hhline{*{7}{|=}|}
 &   & & & & & \\ 
 \hline

 &   & & & & & \\
\hline
 &   & & & & & \\
\hline
 &   & & & & & \\
\hline
 &   & & & & & \\
\hline
 &   & & & & & \\
\hline
 &   & & & & & \\
\hline
 &   & & & & & \\
\hline

\end{tabular}
\end{center}
}
{1}
\end{exop}

\begin{exop}
{À l'aide du tableau, transforme dans l'unité indiquée.
\begin{tasks}(2)
    \task \tunit{1,3}{\centi m}= \dotfill \hspace{0.3cm}
    \tunit{}{\m} \hspace{0.3cm}
    \task \tunit{53,4}{\hecto m} = \dotfill \hspace{0.3cm}
    \tunit{}{\deci m} \hspace{0.3cm}
  \task \tunit{1,27}{\m}= \dotfill \hspace{0.3cm}
    \tunit{}{\kilo  m} \hspace{0.3cm}
    \task \tunit{124,5}{\centi m}= \dotfill \hspace{0.3cm}
    \tunit{}{\m} \hspace{0.3cm}
     \task \tunit{0,142}{\deca m}= \dotfill \hspace{0.3cm}
    \tunit{}{\m} \hspace{0.3cm}
     \task \tunit{10,1}{\m}= \dotfill \hspace{0.3cm}
    \tunit{}{\centi m} \hspace{0.3cm}
     \task \tunit{96,2}{\deca m}= \dotfill \hspace{0.3cm}
    \tunit{}{\kilo m} \hspace{0.3cm}
     \task \tunit{436,7}{\deci m}= \dotfill \hspace{0.3cm}
    \tunit{}{\milli m} \hspace{0.3cm}
\end{tasks}
\begin{center}


\begin{tabular}{*{7}{|x{1.5cm}}|}
 \hline
   km & hm & dam &m &dm &cm&mm\\   
   \hhline{*{7}{|=}|}
 &   & & & & & \\ 
 \hline

 &   & & & & & \\
\hline
 &   & & & & & \\
\hline
 &   & & & & & \\
\hline
 &   & & & & & \\
\hline
 &   & & & & & \\
\hline
 &   & & & & & \\
\hline
 &   & & & & & \\
\hline

\end{tabular}
\end{center}
}
{1}
\end{exop}

\begin{exop}
{Transforme dans l'unité indiquée.
\begin{tasks}(2)
    \task \tunit{5,7}{\milli m}= \dotfill \hspace{0.3cm}
    \tunit{}{\deca  m} \hspace{0.3cm}
    \task \tunit{61,4}{\kilo m} = \dotfill \hspace{0.3cm}
    \tunit{}{\m} \hspace{0.3cm}
  \task \tunit{0,43}{\m}= \dotfill \hspace{0.3cm}
    \tunit{}{\deci  m} \hspace{0.3cm}
    \task \tunit{12,04}{\deci m}= \dotfill \hspace{0.3cm}
    \tunit{}{\m} \hspace{0.3cm}
     \task \tunit{4,2}{\deca m}= \dotfill \hspace{0.3cm}
    \tunit{}{\hecto  m} \hspace{0.3cm}
     \task \tunit{0,8}{\centi m}= \dotfill \hspace{0.3cm}
    \tunit{}{\milli  m} \hspace{0.3cm}
     \task \tunit{0,65}{\m}= \dotfill \hspace{0.3cm}
    \tunit{}{\deca  m} \hspace{0.3cm}
     \task \tunit{9,1}{\kilo m}= \dotfill \hspace{0.3cm}
    \tunit{}{\milli  m} \hspace{0.3cm}
\end{tasks}
}
{1}
\end{exop}

\begin{exop}
{Transforme dans l'unité indiquée.
\begin{tasks}(2)
    \task \tunit{60}{\milli m}= \dotfill \hspace{0.3cm}
    \tunit{}{\deca  m} \hspace{0.3cm}
    \task \tunit{77,4}{\kilo m} = \dotfill \hspace{0.3cm}
    \tunit{}{\m} \hspace{0.3cm}
  \task \tunit{0,9}{\deci m}= \dotfill \hspace{0.3cm}
    \tunit{}{\m} \hspace{0.3cm}
    \task \tunit{100}{\milli m}= \dotfill \hspace{0.3cm}
    \tunit{}{\deci m} \hspace{0.3cm}
     \task \tunit{4,2}{\hecto m}= \dotfill \hspace{0.3cm}
    \tunit{}{\kilo m} \hspace{0.3cm}
     \task \tunit{8}{\centi m}= \dotfill \hspace{0.3cm}
    \tunit{}{\m} \hspace{0.3cm}
     \task \tunit{61,5}{\m}= \dotfill \hspace{0.3cm}
    \tunit{}{\deca  m} \hspace{0.3cm}
     \task \tunit{9,09}{\kilo m}= \dotfill \hspace{0.3cm}
    \tunit{}{\milli  m} \hspace{0.3cm}
\end{tasks}
}
{1}
\end{exop}

\begin{exop}
{Indique l'unité correspondant à la transformation. 
\begin{tasks}(2)
    \task \tunit{6}{\centi m}= 60\dotfill \hspace{0.3cm}

    \task \tunit{8,4}{\m} = 0,83 \dotfill \hspace{0.3cm}
  \task \tunit{7,09}{\deca m}= 70,9\dotfill \hspace{0.3cm}
    \task \tunit{1,6}{\kilo m}= 16000\dotfill \hspace{0.3cm}
     \task \tunit{92,9}{\hecto m}= 9,29\dotfill \hspace{0.3cm}
     \task \tunit{8,88}{\deci m}= 0,00888\dotfill \hspace{0.3cm}
     \task \tunit{6,5}{\m}= 6500\dotfill \hspace{0.3cm}
     \task \tunit{9,9}{\deci m}= 0,0099\dotfill \hspace{0.3cm}
\end{tasks}
}
{1}
\end{exop}

\begin{exof}{GM13}{190}{1}
\end{exof}

%unité de d'aire

\begin{resolu}
{Unité d'aire avec des entiers}    
{À l'aide du tableau, transforme dans l'unité indiquée.

\begin{tasks}(2)
     \task \tunit{66}{\centi m^2}=6600
    \tunit{}{\milli  m^2} \hspace{0.3cm}
    \task \tunit{774}{\deca m^2} =7,74
    \tunit{}{\hecto m^2} \hspace{0.3cm}
\end{tasks}
\begin{center}


\begin{tabular}{*{14}{|x{0.7cm}}|}
 \hline
\multicolumn{2}{|c|}{\nunit{\kilo m^2}} & \multicolumn{2}{c|}{\nunit{\hecto m^2}} & \multicolumn{2}{c|}{\nunit{\deca m^2}}&\multicolumn{2}{c|}{\nunit{m^2}}&\multicolumn{2}{c|}{\nunit{\deci m^2}}&\multicolumn{2}{c|}{\nunit{\centi m^2}}&\multicolumn{2}{c|}{\nunit{\milli m^2}}\\
   \hhline{*{14}{|=}|}
 &   & & & & &  &   & & & 6& 6& 0&0\\ 
 \hline

 &   & &7 & 7& 4&  &   & & & & & &\\

\hline

\end{tabular}
\end{center}
}
{1}
\end{resolu}

\begin{exop}
{À l'aide du tableau, transforme dans l'unité indiquée.

\begin{tasks}(2)
    \task \tunit{60}{\milli m^2}= \dotfill \hspace{0.3cm}
    \tunit{}{\deca  m^2} \hspace{0.3cm}
    \task \tunit{74}{\kilo m^2} = \dotfill \hspace{0.3cm}
    \tunit{}{\m^2} \hspace{0.3cm}
  \task \tunit{9}{\deci m^2}= \dotfill \hspace{0.3cm}
    \tunit{}{\m^2} \hspace{0.3cm}
    \task \tunit{100}{\milli m^2}= \dotfill \hspace{0.3cm}
    \tunit{}{\deci m^2} \hspace{0.3cm}
     \task \tunit{42}{\hecto m^2}= \dotfill \hspace{0.3cm}
    \tunit{}{\kilo m^2} \hspace{0.3cm}
     \task \tunit{80}{\centi m^2}= \dotfill \hspace{0.3cm}
    \tunit{}{\m^2} \hspace{0.3cm}
     \task \tunit{615}{\m^2}= \dotfill \hspace{0.3cm}
    \tunit{}{\deca  m^2} \hspace{0.3cm}
     \task \tunit{9}{\kilo m^2}= \dotfill \hspace{0.3cm}
    \tunit{}{\milli m^2} \hspace{0.3cm}
\end{tasks}


\begin{center}
\begin{tabular}{*{14}{|x{0.7cm}}|}
 \hline
\multicolumn{2}{|c|}{\nunit{\kilo m^2}} & \multicolumn{2}{c|}{\nunit{\hecto m^2}} & \multicolumn{2}{c|}{\nunit{\deca m^2}}&\multicolumn{2}{c|}{\nunit{m^2}}&\multicolumn{2}{c|}{\nunit{\deci m^2}}&\multicolumn{2}{c|}{\nunit{\centi m^2}}&\multicolumn{2}{c|}{\nunit{\milli m^2}}\\
   \hhline{*{14}{|=}|}
 &   & & & & &  &   & & & & & &\\ 
 \hline

 &   & & & & &  &   & & & & & &\\

\hline
 &   & & & & &  &   & & & & & &\\

\hline
 &   & & & & &  &   & & & & & &\\

\hline
 &   & & & & &  &   & & & & & &\\

\hline
 &   & & & & &  &   & & & & & &\\

\hline
 &   & & & & &  &   & & & & & &\\

\hline
 &   & & & & &  &   & & & & & &\\

\hline

\end{tabular}
\end{center}
}
{1}
\end{exop}

\begin{exop}
{
Transforme dans l'unité indiquée.
\begin{tasks}(2)
    \task \tunit{132}{\centi m^2}= \dotfill \hspace{0.3cm}
    \tunit{}{\deca  m^2} \hspace{0.3cm}
    \task \tunit{4}{\kilo m^2} = \dotfill \hspace{0.3cm}
    \tunit{}{\deca m^2} \hspace{0.3cm}
  \task \tunit{129}{\milli m^2}= \dotfill \hspace{0.3cm}
    \tunit{}{\m^2} \hspace{0.3cm}
    \task \tunit{40}{\deci m^2}= \dotfill \hspace{0.3cm}
    \tunit{}{\deca m^2} \hspace{0.3cm}
     \task \tunit{432}{\hecto m^2}= \dotfill \hspace{0.3cm}
    \tunit{}{\kilo m^2} \hspace{0.3cm}
     \task \tunit{80}{\m^2}= \dotfill \hspace{0.3cm}
    \tunit{}{\deci m^2} \hspace{0.3cm}
     \task \tunit{98}{\deca m^2}= \dotfill \hspace{0.3cm}
    \tunit{}{\hecto m^2} \hspace{0.3cm}
     \task \tunit{9}{\hecto m^2}= \dotfill \hspace{0.3cm}
    \tunit{}{\deci m^2} \hspace{0.3cm}
\end{tasks} 
}
{1}
\end{exop}

\begin{resolu}
{Unité d'aire avec des décimaux}    
{À l'aide du tableau, transforme dans l'unité indiquée.

\begin{tasks}
   \task \tunit{6,6}{\centi m^2}=660
    \tunit{}{\milli  m^2} \hspace{0.3cm}
    \task \tunit{77,4}{\deca m^2} =0,774
    \tunit{}{\hecto m^2} \hspace{0.3cm}
\end{tasks}
\begin{center}


\begin{tabular}{*{14}{|x{0.7cm}}|}
 \hline
\multicolumn{2}{|c|}{\nunit{\kilo m^2}} & \multicolumn{2}{c|}{\nunit{\hecto m^2}} & \multicolumn{2}{c|}{\nunit{\deca m^2}}&\multicolumn{2}{c|}{\nunit{m^2}}&\multicolumn{2}{c|}{\nunit{\deci m^2}}&\multicolumn{2}{c|}{\nunit{\centi m^2}}&\multicolumn{2}{c|}{\nunit{\milli m^2}}\\
   \hhline{*{14}{|=}|}
 &   & & & & &  &   & & & & 6& 6&0\\ 
 \hline

 &   & &0 & 7& 7& 4 &   & & & & & &\\

\hline

\end{tabular}
\end{center}
}
{1}
\end{resolu}

\begin{exop}
{À l'aide du tableau, transforme dans l'unité indiquée.

\begin{tasks}(2)
    \task \tunit{3,63}{\centi m^2}= \dotfill \hspace{0.3cm}
    \tunit{}{\deca  m^2} \hspace{0.3cm}
    \task \tunit{7,4}{\kilo m^2} = \dotfill \hspace{0.3cm}
    \tunit{}{\m^2} \hspace{0.3cm}
  \task \tunit{9,99}{\deci m^2}= \dotfill \hspace{0.3cm}
    \tunit{}{\m^2} \hspace{0.3cm}
    \task \tunit{100,1}{\deca m^2}= \dotfill \hspace{0.3cm}
    \tunit{}{\deci m^2} \hspace{0.3cm}
     \task \tunit{4,2}{\hecto m^2}= \dotfill \hspace{0.3cm}
    \tunit{}{\kilo m^2} \hspace{0.3cm}
     \task \tunit{80,9}{\centi m^2}= \dotfill \hspace{0.3cm}
    \tunit{}{\m^2} \hspace{0.3cm}
     \task \tunit{6,15}{\m^2}= \dotfill \hspace{0.3cm}
    \tunit{}{\deca  m^2} \hspace{0.3cm}
     \task \tunit{9.7}{\kilo m^2}= \dotfill \hspace{0.3cm}
    \tunit{}{\milli m^2} \hspace{0.3cm}
\end{tasks}

\begin{center}
\bgroup
\def\arraystretch{0.9}
\begin{tabular}{*{14}{|x{0.7cm}}|}
 \hline
\multicolumn{2}{|c|}{\nunit{\kilo m^2}} & \multicolumn{2}{c|}{\nunit{\hecto m^2}} & \multicolumn{2}{c|}{\nunit{\deca m^2}}&\multicolumn{2}{c|}{\nunit{m^2}}&\multicolumn{2}{c|}{\nunit{\deci m^2}}&\multicolumn{2}{c|}{\nunit{\centi m^2}}&\multicolumn{2}{c|}{\nunit{\milli m^2}}\\
   \hhline{*{14}{|=}|}
 &   & & & & &  &   & & & & & &\\ 
 \hline

 &   & & & & &  &   & & & & & &\\

\hline
 &   & & & & &  &   & & & & & &\\

\hline
 &   & & & & &  &   & & & & & &\\

\hline
 &   & & & & &  &   & & & & & &\\

\hline
 &   & & & & &  &   & & & & & &\\

\hline
 &   & & & & &  &   & & & & & &\\

\hline
 &   & & & & &  &   & & & & & &\\

\hline

\end{tabular}
\egroup
\end{center}
}
{1}
\end{exop}

\begin{exop}
{
Transforme dans l'unité indiquée.
\begin{tasks}(2)
    \task \tunit{132}{\centi m^2}= \dotfill \hspace{0.3cm}
    \tunit{}{\deca  m^2} \hspace{0.3cm}
    \task \tunit{4}{\kilo m^2} = \dotfill \hspace{0.3cm}
    \tunit{}{\deca m^2} \hspace{0.3cm}
  \task \tunit{129}{\milli m^2}= \dotfill \hspace{0.3cm}
    \tunit{}{\m^2} \hspace{0.3cm}
    \task \tunit{40}{\deci m^2}= \dotfill \hspace{0.3cm}
    \tunit{}{\deca m^2} \hspace{0.3cm}
     \task \tunit{432}{\hecto m^2}= \dotfill \hspace{0.3cm}
    \tunit{}{\kilo m^2} \hspace{0.3cm}
     \task \tunit{80}{\m^2}= \dotfill \hspace{0.3cm}
    \tunit{}{\deci m^2} \hspace{0.3cm}
     \task \tunit{98}{\deca m^2}= \dotfill \hspace{0.3cm}
    \tunit{}{\hecto m^2} \hspace{0.3cm}
     \task \tunit{9}{\hecto m^2}= \dotfill \hspace{0.3cm}
    \tunit{}{\deci m^2} \hspace{0.3cm}
\end{tasks} 
}
{1}
\end{exop}


\begin{exop}
{
Transforme dans l'unité indiquée.
\begin{tasks}(2)
    \task \tunit{13,2}{\centi m^2}= \dotfill \hspace{0.3cm}
    \tunit{}{\deca  m^2} \hspace{0.3cm}
    \task \tunit{0,4}{\kilo m^2} = \dotfill \hspace{0.3cm}
    \tunit{}{\deca m^2} \hspace{0.3cm}
  \task \tunit{1,29}{\milli m^2}= \dotfill \hspace{0.3cm}
    \tunit{}{\m^2} \hspace{0.3cm}
    \task \tunit{40,4}{\deci m^2}= \dotfill \hspace{0.3cm}
    \tunit{}{\deca m^2} \hspace{0.3cm}
     \task \tunit{43,2}{\hecto m^2}= \dotfill \hspace{0.3cm}
    \tunit{}{\kilo m^2} \hspace{0.3cm}
     \task \tunit{0,07}{\m^2}= \dotfill \hspace{0.3cm}
    \tunit{}{\deci m^2} \hspace{0.3cm}
     \task \tunit{19,8}{\deca m^2}= \dotfill \hspace{0.3cm}
    \tunit{}{\hecto m^2} \hspace{0.3cm}
     \task \tunit{6,9}{\hecto m^2}= \dotfill \hspace{0.3cm}
    \tunit{}{\deci m^2} \hspace{0.3cm}
\end{tasks} 
}
{1}
\end{exop}


\begin{exop}
{
Transforme dans l'unité indiquée.
\begin{tasks}(2)
    \task \tunit{13}{\centi m^2}= \dotfill \hspace{0.3cm}
    \tunit{}{\deca  m^2} \hspace{0.3cm}
    \task \tunit{0,04}{\hecto m^2} = \dotfill \hspace{0.3cm}
    \tunit{}{\deca m^2} \hspace{0.3cm}
  \task \tunit{12,09}{\deci m^2}= \dotfill \hspace{0.3cm}
    \tunit{}{\deca m^2} \hspace{0.3cm}
    \task \tunit{4040}{\deci m^2}= \dotfill \hspace{0.3cm}
    \tunit{}{\kilo m^2} \hspace{0.3cm}
     \task \tunit{0,0432}{\hecto m^2}= \dotfill \hspace{0.3cm}
    \tunit{}{\kilo m^2} \hspace{0.3cm}
     \task \tunit{134,5}{\deca m^2}= \dotfill \hspace{0.3cm}
    \tunit{}{\deci m^2} \hspace{0.3cm}
     \task \tunit{132}{\centi m^2}= \dotfill \hspace{0.3cm}
    \tunit{}{\hecto m^2} \hspace{0.3cm}
     \task \tunit{9,9}{\hecto m^2}= \dotfill \hspace{0.3cm}
    \tunit{}{\deci m^2} \hspace{0.3cm}
\end{tasks} 
}
{1}
\end{exop}


\begin{exop}
{
	Indique l'unité qui correspond à la transformation.
\begin{tasks}(2)
    \task \tunit{73}{\milli m^2}= 0,73\dotfill \hspace{0.3cm}
    \task \tunit{7,6}{\deca m^2} = 760 \dotfill \hspace{0.3cm}

  \task \tunit{1,9}{\deci m^2}= 0,00019\dotfill \hspace{0.3cm}
    \task \tunit{4,09}{\m^2}= 0,00000409\dotfill \hspace{0.3cm}
     \task \tunit{0,032}{\hecto m^2}= 3,2\dotfill \hspace{0.3cm}
     \task \tunit{13,4}{\deca m^2}= 134000\dotfill \hspace{0.3cm}
     \task \tunit{1,2}{\centi m^2}= 0,0000012\dotfill \hspace{0.3cm}
     \task \tunit{99}{\hecto m^2}= 9900000000\dotfill \hspace{0.3cm}
\end{tasks} 
}
{1}
\end{exop}


\begin{exof}{GM3}{185}{1}
\end{exof}

\begin{exof}{GM15}{190}{1}
\end{exof}

\begin{exof}{GM18}{191}{1}
\end{exof}

\begin{exol}{GM14}{139}{1}
\end{exol}

\begin{exof}{GM16}{190}{1}
\end{exof}

\begin{exof}{GM17}{191}{1}
\end{exof}


\begin{exof}{GM19}{192}{2}
\end{exof}

%unité de masse
\begin{resolu}
    {Unité de masse avec des entiers}
    {À l'aide du tableau, transforme dans l'unité indiquée.

\begin{tasks}(2)
    \task $\tunit{35}{\kilo g}= \tunit{0,035}{\tonne}$
    \task $\tunit{35}{\hecto m} = \tunit{3500}{\g}$
    
\end{tasks}
\begin{center}

\begin{tabular}{*{10}{|x{0.9cm}}|}
 \hline
  t & & & kg & hg & dag &g &dg &cg&mg\\   
\hhline{*{10}{|=}|}
0 & 0  &3 & 5& & & & & & \\ 
 \hline

 &   & & 3&5 &0 &0 & & & \\ 
\hline

\end{tabular} 
\end{center}
    
    }
{1}
\end{resolu}

\begin{exop}
{À l'aide du tableau, transforme dans l'unité indiquée.
\begin{tasks}(2)
    \task \tunit{133}{\deci g}= \dotfill \hspace{0.3cm}
    \tunit{}{\deca g} \hspace{0.3cm}
    \task \tunit{37}{\hecto g} = \dotfill \hspace{0.3cm}
    \tunit{}{\kilo g} \hspace{0.3cm}
  \task \tunit{127}{\kilo g}= \dotfill \hspace{0.3cm}
    \tunit{}{\tonne} \hspace{0.3cm}
    \task \tunit{145}{\deci g}= \dotfill \hspace{0.3cm}
    \tunit{}{\deca g} \hspace{0.3cm}
     \task \tunit{12}{\deca g}= \dotfill \hspace{0.3cm}
    \tunit{}{\hecto g} \hspace{0.3cm}
     \task \tunit{1}{\g}= \dotfill \hspace{0.3cm}
    \tunit{}{\centi g} \hspace{0.3cm}
     \task \tunit{92}{\deca g}= \dotfill \hspace{0.3cm}
    \tunit{}{\milli g} \hspace{0.3cm}
     \task \tunit{467}{\g}= \dotfill \hspace{0.3cm}
    \tunit{}{\tonne} \hspace{0.3cm}
\end{tasks}
\begin{center}


\begin{tabular}{*{10}{|x{0.9cm}}|}
 \hline
  t & & & kg & hg & dag &g &dg &cg&mg\\   
\hhline{*{10}{|=}|}
 & & & & & & & & & \\ 
 \hline

 &   & & & & & & & & \\ 
\hline
 & & & & & & & & & \\ 
 \hline

 &   & & & & & & & & \\ 
\hline
 & & & & & & & & & \\ 
 \hline

 &   & & & & & & & & \\ 
\hline
 & & & & & & & & & \\ 
 \hline

 &   & & & & & & & & \\ 
\hline

\end{tabular}
\end{center}
}
{1}
\end{exop}

\begin{exop}
{
	Transforme dans l'unité indiquée.
\begin{tasks}(2)
    \task \tunit{13}{\centi g}= \dotfill \hspace{0.3cm}
    \tunit{}{\deca  g} \hspace{0.3cm}
    \task \tunit{4}{\hecto g} = \dotfill \hspace{0.3cm}
    \tunit{}{\deca g} \hspace{0.3cm}
  \task \tunit{1209}{\deci g}= \dotfill \hspace{0.3cm}
    \tunit{}{\tonne} \hspace{0.3cm}
    \task \tunit{44}{\deci g}= \dotfill \hspace{0.3cm}
    \tunit{}{\kilo g} \hspace{0.3cm}
     \task \tunit{432}{\kilo g}= \dotfill \hspace{0.3cm}
    \tunit{}{\tonne} \hspace{0.3cm}
     \task \tunit{1345}{\deca g}= \dotfill \hspace{0.3cm}
    \tunit{}{\deci g} \hspace{0.3cm}
     \task \tunit{132}{\centi g}= \dotfill \hspace{0.3cm}
    \tunit{}{\milli g} \hspace{0.3cm}
     \task \tunit{99}{\hecto g}= \dotfill \hspace{0.3cm}
    \tunit{}{\deci g} \hspace{0.3cm}
\end{tasks} 
}
{1}
\end{exop}

\begin{resolu}
    {Unité de masse avec des décimaux}
    {À l'aide du tableau, transforme dans l'unité indiquée.

\begin{tasks}(2)
   \task $\tunit{3,5}{\kilo g}= \tunit{0,0035}{\tonne}$
    \task $\tunit{0,35}{\hecto g} = \tunit{35}{\g}$
    
\end{tasks}
\begin{center}



\begin{tabular}{*{10}{|x{0.9cm}}|}
 \hline
  t & & & kg & hg & dag &g &dg &cg&mg\\   
\hhline{*{10}{|=}|}
0 & 0  &0 & 3&5 & & & & & \\ 
 \hline

 &   & & &0 &3 &5 & & & \\ 
\hline

\end{tabular}
\end{center}
    
    }
{1}
\end{resolu}


\begin{exop}
{À l'aide du tableau, transforme dans l'unité indiquée.
\begin{tasks}(2)
    \task \tunit{13,5}{\hecto g}= \dotfill \hspace{0.3cm}
    \tunit{}{\deca g} \hspace{0.3cm}
    \task \tunit{2,04}{\kilo g} = \dotfill \hspace{0.3cm}
    \tunit{}{\deci g} \hspace{0.3cm}
  \task \tunit{65,4}{\kilo g}= \dotfill \hspace{0.3cm}
    \tunit{}{\tonne} \hspace{0.3cm}
    \task \tunit{0,05}{\deci g}= \dotfill \hspace{0.3cm}
    \tunit{}{\deca g} \hspace{0.3cm}
     \task \tunit{10,2}{\g}= \dotfill \hspace{0.3cm}
    \tunit{}{\hecto g} \hspace{0.3cm}
     \task \tunit{1,01}{\g}= \dotfill \hspace{0.3cm}
    \tunit{}{\centi g} \hspace{0.3cm}
     \task \tunit{0,001}{\deca g}= \dotfill \hspace{0.3cm}
    \tunit{}{\milli g} \hspace{0.3cm}
     \task \tunit{4,7}{\g}= \dotfill \hspace{0.3cm}
    \tunit{}{\tonne} \hspace{0.3cm}
\end{tasks}
\begin{center}


\begin{tabular}{*{10}{|x{0.9cm}}|}
 \hline
  t & & & kg & hg & dag &g &dg &cg&mg\\   
\hhline{*{10}{|=}|}
 & & & & & & & & & \\ 
 \hline

 &   & & & & & & & & \\ 
\hline
 & & & & & & & & & \\ 
 \hline

 &   & & & & & & & & \\ 
\hline
 & & & & & & & & & \\ 
 \hline

 &   & & & & & & & & \\ 
\hline
 & & & & & & & & & \\ 
 \hline

 &   & & & & & & & & \\ 
\hline

\end{tabular}
\end{center}
}
{1}
\end{exop}

\begin{exop}
{
	Transforme dans l'unité indiquée.
\begin{tasks}(2)
    \task \tunit{4,5}{\deci g}= \dotfill \hspace{0.3cm}
    \tunit{}{\deca  g} \hspace{0.3cm}
    \task \tunit{0,07}{\hecto g} = \dotfill \hspace{0.3cm}
    \tunit{}{\deca g} \hspace{0.3cm}
  \task \tunit{120,9}{\centi g}= \dotfill \hspace{0.3cm}
    \tunit{}{\deci g} \hspace{0.3cm}
    \task \tunit{4,4}{\tonne}= \dotfill \hspace{0.3cm}
    \tunit{}{\kilo g} \hspace{0.3cm}
     \task \tunit{46,9}{\kilo g}= \dotfill \hspace{0.3cm}
    \tunit{}{\deca g} \hspace{0.3cm}
     \task \tunit{170,45}{\deca g}= \dotfill \hspace{0.3cm}
    \tunit{}{\deci g} \hspace{0.3cm}
     \task \tunit{1,02}{\deci g}= \dotfill \hspace{0.3cm}
    \tunit{}{\milli g} \hspace{0.3cm}
     \task \tunit{0,707}{\hecto g}= \dotfill \hspace{0.3cm}
    \tunit{}{\deci g} \hspace{0.3cm}
\end{tasks} 
}
{1}
\end{exop}

\begin{exop}
{
Transforme dans l'unité indiquée.
\begin{tasks}(2)
    \task \tunit{5}{\deci g}= \dotfill \hspace{0.3cm}
    \tunit{}{\milli  g} \hspace{0.3cm}
    \task \tunit{0,0007}{\hecto g} = \dotfill \hspace{0.3cm}
    \tunit{}{\deci g} \hspace{0.3cm}
  \task \tunit{109}{\centi g}= \dotfill \hspace{0.3cm}
    \tunit{}{\g} \hspace{0.3cm}
    \task \tunit{4,94}{\kilo g}= \dotfill \hspace{0.3cm}
    \tunit{}{\tonne} \hspace{0.3cm}
     \task \tunit{29}{\kilo g}= \dotfill \hspace{0.3cm}
    \tunit{}{\deca g} \hspace{0.3cm}
     \task \tunit{10,01}{\deca g}= \dotfill \hspace{0.3cm}
    \tunit{}{\milli g} \hspace{0.3cm}
     \task \tunit{2}{\deci g}= \dotfill \hspace{0.3cm}
    \tunit{}{\milli g} \hspace{0.3cm}
     \task \tunit{0,77}{\kilo g}= \dotfill \hspace{0.3cm}
    \tunit{}{\deci g} \hspace{0.3cm}
\end{tasks} 
}
{1}
\end{exop}

\begin{exop}
{
	Indique l'unité qui correspond à la transformation.
\begin{tasks}(2)
    \task \tunit{55}{\deci g}=  550 \dotfill \hspace{0.3cm}
    
    \task \tunit{0,03}{\deci g} = 3\dotfill \hspace{0.3cm}
  \task \tunit{10,9}{\centi g}= 0,109 \dotfill \hspace{0.3cm}
   
    \task \tunit{0,954}{\tonne}= 954 \dotfill \hspace{0.3cm}
    
     \task \tunit{2,9}{\g}= 0,29\dotfill \hspace{0.3cm}
     \task \tunit{1,11}{\hecto g}= 0,111\dotfill \hspace{0.3cm}
     \task \tunit{22}{\deci g}= 220\dotfill \hspace{0.3cm}
     \task \tunit{0,07}{\deca g}= 7 \dotfill \hspace{0.3cm}
\end{tasks} 
}
{1}
\end{exop}

\begin{resolu}
    {On mélange!}
    {Transforme dans l'unité indiquée.

\begin{tasks}(3)
    \task $\tunit{65}{\kilo m}= \tunit{65000}{\m}$
    \task $\tunit{0,5}{\hecto g} = \tunit{0,05}{\kilo g}$
    \task $\tunit{3,5}{\m^2}= \tunit{350}{\deci m^2}$
    
\end{tasks}   
    }
{1}
\end{resolu}


\begin{exop}
{
Transforme dans l'unité indiquée.
\begin{tasks}(2)
    \task \tunit{55}{\deci m^2}= \dotfill \hspace{0.3cm}
    \tunit{}{\deca m^2} \hspace{0.3cm}
    \task \tunit{4327}{\hecto g} = \dotfill \hspace{0.3cm}
    \tunit{}{\tonne} \hspace{0.3cm}
  \task \tunit{1,009}{\centi m}= \dotfill \hspace{0.3cm}
    \tunit{}{\milli m} \hspace{0.3cm}
    \task \tunit{0,0494}{\kilo m^2}= \dotfill \hspace{0.3cm}
    \tunit{}{\centi m^2} \hspace{0.3cm}
     \task \tunit{5,5}{\kilo g}= \dotfill \hspace{0.3cm}
    \tunit{}{\deca g} \hspace{0.3cm}
     \task \tunit{0,01}{\deca m^2}= \dotfill \hspace{0.3cm}
    \tunit{}{\kilo m^2} \hspace{0.3cm}
     \task \tunit{78}{\deci g}= \dotfill \hspace{0.3cm}
    \tunit{}{\milli g} \hspace{0.3cm}
     \task \tunit{5,5}{\kilo m^2}= \dotfill \hspace{0.3cm}
    \tunit{}{\deci m^2} \hspace{0.3cm}
\end{tasks} 
}
{1}
\end{exop}


\begin{exop}
{
Transforme dans l'unité indiquée.
\begin{tasks}[after-item-skip=0.3em](2)
    \task \tunit{3}{\tonne}= \dotfill \hspace{0.3cm}
    \tunit{}{\milli  g} \hspace{0.3cm}
    \task \tunit{0,009}{\hecto m^2} = \dotfill \hspace{0.3cm}
    \tunit{}{\deci m^2} \hspace{0.3cm}
  \task \tunit{10}{\centi m}= \dotfill \hspace{0.3cm}
    \tunit{}{\m} \hspace{0.3cm}
    \task \tunit{40,4}{\milli g}= \dotfill \hspace{0.3cm}
    \tunit{}{\tonne} \hspace{0.3cm}
     \task \tunit{290}{\kilo m^2}= \dotfill \hspace{0.3cm}
    \tunit{}{\m^2} \hspace{0.3cm}
     \task \tunit{0,01}{\deci g}= \dotfill \hspace{0.3cm}
    \tunit{}{\milli g} \hspace{0.3cm}
     \task \tunit{20}{\deci m^2}= \dotfill \hspace{0.3cm}
    \tunit{}{\milli m^2} \hspace{0.3cm}
     \task \tunit{0,7}{\kilo m}= \dotfill \hspace{0.3cm}
    \tunit{}{\deci m} \hspace{0.3cm}
\end{tasks} 
}
{1}
\end{exop}

%mélange


\begin{exof}{GM85}{207}{1}
\end{exof}


\begin{exof}{GM87}{208}{2}
\end{exof}


\begin{exof}{GM88}{208}{2}
\end{exof}

\begin{FLP}{195}{1}
\end{FLP}

\end{document}

