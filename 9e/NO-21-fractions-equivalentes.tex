\documentclass[a4paper,12pt]{report}
\usepackage[showexo=true,showcorr=false]{../packages/coursclasse}
%\usepackage{arev}
%\usepackage{sansmathfonts}
% - Simplification et amplification de fractions
% - Fractions irréductibles
% - Ecriture décimale à fraction irréductible
% - Fractions équivalentes
% - Comparaison de fractions

%Commenter ou enlever le commentaire sur la ligne suivante pour montrer le niveau
\toggletrue{montrerNiveaux}
%permet de gérer l'espacement entre les items des env enumerate et enumitem
\usepackage{enumitem}
\setlist[enumerate]{align=left,leftmargin=1cm,itemsep=10pt,parsep=0pt,topsep=0pt,rightmargin=0.5cm}
\setlist[itemize]{align=left,labelsep=1em,leftmargin=*,itemsep=0pt,parsep=0pt,topsep=0pt,rightmargin=0cm}
%permet de gerer l'espacement entre les colonnes de multicols
\setlength\columnsep{35pt}

\begin{document}
%%%%%%%%%%%%%%%%% À MODIFIER POUR CHAQUE SERIE %%%%%%%%%%%%%%%%%%%%%%%%%%%%%
\newcommand{\chapterName}{Nombres et opérations}
\newcommand{\serieName}{Fractions équivalentes}


%%%%%%%%%%%%%%%%%% PREMIERE PAGE NE PAS MODIFER %%%%%%%%%%%%%%%%%%%%%%%%
% le chapitre en cours, ne pas changer au cours d'une série
\chapter*{\chapterName}
\thispagestyle{empty}

%%%%% LISTE AIDE MEMOIRE %%%%%%
\begin{amL}{\serieName}{
\item Définition d'une fraction (page 27)
\item Passage d'une écriture fractionnaire à une écriture décimale (page 28)
\item Passer d'une écriture décimale à une écriture fractionnaire (page 28)
	\item Amplification et simplification d'une fraction (page 29)
}
\end{amL}
%%%%%%%%%%%%%%% DEBUT DE LA SERIE NE PAS MODIFIER %%%%%%%%%%%%%%%%%%%%%%%%%%%%%
\section*{\serieName}
\setcounter{page}{1}
\thispagestyle{firstPage}



%%%%%%%%%%% LES EXERCICES %%%%%%%%%%%%%%%%%%%%%%%%%%%%%%%%%%%


\begin{exop}{
Complète en fonction des représentations graphiques.

\vspace{-1cm}
\begin{tasks}[after-item-skip = -0.4em](3)
	\task $\begin{matrix}[2]\vfracrect[xscale=3]{3}{4} && \vfracrect[xscale=3]{9}{12}\\
	\emptyfrac&=&\emptyfrac 
\end{matrix}$
	\task $\begin{matrix}[2]\vfracrect[xscale=3]{10}{20} && \vfracrect[xscale=3]{1}{2}\\
	\emptyfrac&=&\emptyfrac 
\end{matrix}$
	\task $\begin{matrix}[2]\vfracrect[xscale=3]{3}{9} && \vfracrect[xscale=3]{1}{3}\\
	\emptyfrac&=&\emptyfrac 
\end{matrix}$
	\task $\begin{matrix}[2]\vfracrect[xscale=3]{4}{4} && \vfracrect[xscale=3]{16}{16}\\
	\emptyfrac&=&\emptyfrac
\end{matrix}$
	\task $\begin{matrix}[2]\vfracrect[xscale=3]{1}{5} && \vfracrect[xscale=3]{3}{15}\\
	\emptyfrac&=&\emptyfrac
\end{matrix}$
	\task $\begin{matrix}[2]\vfracrect[xscale=3]{6}{24} && \vfracrect[xscale=3]{1}{4}\\
	\emptyfrac&=&\emptyfrac 
\end{matrix}$
\end{tasks}
}{1}\end{exop}



\begin{exof}{NO173}{59}{1}
\end{exof}
\begin{exof}{NO174}{60}{1}
\end{exof}

\begin{exop}{
Complète afin d'obtenir une égalité.
\begin{tasks}[after-item-skip = 0.2em, after-skip=-1em](3)
    \task $\dfrac{2}{7}=\dfrac{\ldots\ldots}{14}$\\
    \task $\dfrac{6}{8}=\dfrac{24}{\ldots\ldots}$\\
    \task $\dfrac{11}{10}=\dfrac{\ldots\ldots}{70}$\\
    \task $\dfrac{45}{30}=\dfrac{15}{\ldots\ldots}$\\
    \task $\dfrac{33}{11}=\dfrac{\ldots\ldots}{1}$\\
    \task $\dfrac{21}{8}=\dfrac{42}{\ldots\ldots}$\\
\end{tasks}
}{1}\end{exop}

\begin{exop}{
Complète afin d'obtenir une égalité.
\begin{tasks}[after-item-skip = 0.2em, after-skip=-1em](3)
    \task $\dfrac{5}{4}=\dfrac{\ldots\ldots}{72}$\\
    \task $\dfrac{105}{30}=\dfrac{21}{\ldots\ldots}$\\
    \task $\dfrac{26}{10}=\dfrac{\ldots\ldots}{15}$\\
    \task $\dfrac{14}{28}=\dfrac{35}{\ldots\ldots}$\\
    \task $\dfrac{16}{24}=\dfrac{\ldots\ldots}{9}$\\
    \task $\dfrac{51}{17}=\dfrac{15}{\ldots\ldots}$\\
\end{tasks}
}{3}\end{exop}


\begin{exof}{NO181}{65}{1}
\end{exof}
\begin{exof}{NO182}{65}{2}
\end{exof}


\begin{exo}{
Ecris quatre fractions équivalentes à:
\begin{tasks}[after-item-skip = 0.2em, after-skip=-0.5em](5)
\task $\dfrac{1}{4}$
\task $\dfrac{2}{5}$
\task $\dfrac{3}{4}$
\task $\dfrac{1}{2}$
\task $\dfrac{6}{10}$
\end{tasks}
 \vspace{1pt}
}{1}\end{exo}

\begin{exo}{
Parmi les fractions suivantes, lesquelles sont équivalentes à $\dfrac{2}{3}$ ?
\begin{tasks}[after-item-skip = 0.2em](4)
\task $\dfrac{8}{12}$
\task $\dfrac{12}{18}$
\task $\dfrac{5}{9}$
\task $\dfrac{9}{12}$
\task $\dfrac{6}{9}$
\task $\dfrac{14}{21}$
\task $\dfrac{14}{24}$
\end{tasks}
 \vspace{1pt}
}{1}\end{exo}

\begin{exo}{
Parmi les fractions suivantes, lesquelles sont équivalentes à $\dfrac{1}{2} ?$
\begin{tasks}(3)
\task $\dfrac{6}{12}$
\task $\dfrac{48}{96}$
\task $\dfrac{52}{102}$
\task $\dfrac{70}{140}$
\task $\dfrac{2048}{4096}$
\end{tasks}
 \vspace{1pt}
}{1}\end{exo}

\begin{exo}{
Parmi les fractions suivantes, lesquelles sont équivalentes à $\dfrac{1}{5}$ ?
\begin{tasks}(3)
\task $\dfrac{3}{15}$
\task $\dfrac{4}{25}$
\task $\dfrac{100}{500}$
\task $\dfrac{50}{250}$
\task $\dfrac{6}{30}$
\task $\dfrac{24}{120}$
\end{tasks}
 \vspace{1pt}
}{1}\end{exo}

\begin{exo}{
Parmi les fractions suivantes, lesquelles sont équivalentes à $\dfrac{3}{9}$ ?
\begin{tasks}(4)
\task $\dfrac{2}{6}$
\task $\dfrac{8}{21}$
\task $\dfrac{6}{18}$
\task $\dfrac{5}{25}$
\task $\dfrac{8}{24}$
\task $\dfrac{12}{36}$
\task $\dfrac{10}{30}$
\end{tasks}
 \vspace{1pt}
}{2}\end{exo}

\begin{exop}{
Identifie, parmi les fractions suivantes, les fractions équivalentes.
\vspace{10pt}
\begin{itemize}
    \item[] à $\dfrac{1}{2}:\hrulefill$\\
    \item[] à $\dfrac{2}{3}:\hrulefill$\\
    \item[] à $\dfrac{3}{2}:\hrulefill$\\
\end{itemize}
\vspace{0.5cm}

 \begin{center}
\begin{tabular}{cccccccc}
     $\dfrac{20}{30}$;&$\dfrac{6}{2}$;&$\dfrac{6}{8}$;&$\dfrac{10}{5}$;&$\dfrac{4}{8}$;&$\dfrac{70}{105}$;&$\dfrac{12}{9}$;&$\dfrac{7}{14}$;\\
		      &&&&&&&\\
     $\dfrac{18}{9}$;&$\dfrac{41}{82}$;&$\dfrac{8}{16}$;&$\dfrac{25}{5}$;&$\dfrac{15}{10}$;&$\dfrac{72}{48}$;&$\dfrac{21}{7}$;& $\dfrac{15}{5}$\\
\end{tabular}
\end{center}
}{2}\end{exop}


\begin{exol}{NO180}{51}{1}
\end{exol}

\begin{exop}{
Simplifie les fractions suivantes par le nombre indiqué.
\begin{tasks}(2)
    \task $\dfrac{4}{10}$ simplifiée par $2: \hrulefill$
    \task $\dfrac{15}{21}$ simplifiée par $3: \hrulefill$
    \task $\dfrac{32}{12}$ simplifiée par $2: \hrulefill$
    \task $\dfrac{9}{81}$ simplifiée par $9: \hrulefill$
    \task $\dfrac{50}{25}$ simplifiée par $5: \hrulefill$
    \task $\dfrac{300}{500}$ simplifiée par $10: \hrulefill$
\end{tasks}
 \vspace{1pt}

}{1}\end{exop}

\begin{exop}{
Simplifie les fractions suivantes par le nombre indiqué.
\begin{tasks}(2)
    \task $\dfrac{8}{20}$ simplifiée par $4: \hrulefill$
    \task $\dfrac{60}{40}$ simplifiée par $20: \hrulefill$
    \task $\dfrac{49}{147}$ simplifiée par $7: \hrulefill$
    \task $\dfrac{125}{500}$ simplifiée par $25: \hrulefill$
    \task $\dfrac{39}{169}$ simplifiée par $13: \hrulefill$
    \task $\dfrac{126}{252}$ simplifiée par $9: \hrulefill$
\end{tasks}
 \vspace{1pt}

}{1}\end{exop}

\begin{exop}{
Amplifie les fractions suivantes par le nombre indiqué.
\begin{tasks}[after-item-skip = 0.5em, after-skip=-0.5em](2)
    \task $\dfrac{10}{5}$ amplifiée par $5: \hrulefill$
    \task $\dfrac{7}{1}$ amplifiée par $7: \hrulefill$
    \task $\dfrac{4}{6}$ amplifiée par $3: \hrulefill$
    \task $\dfrac{3}{7}$ amplifiée par $2: \hrulefill$
    \task $\dfrac{3}{8}$ amplifiée par $5: \hrulefill$
    \task $\dfrac{9}{4}$ amplifiée par $11: \hrulefill$
\end{tasks}
}{1}\end{exop}

\begin{exop}{
Amplifie les fractions suivantes par le nombre indiqué.
\begin{tasks}[after-item-skip = 0.5em, after-skip=-0.5em](2)
    \task $\dfrac{2}{7}$ amplifiée par $9: \hrulefill$
    \task $\dfrac{4}{9}$ amplifiée par $12: \hrulefill$
    \task $\dfrac{5}{3}$ amplifiée par $6: \hrulefill$
    \task $\dfrac{10}{12}$ amplifiée par $8: \hrulefill$
    \task $\dfrac{7}{4}$ amplifiée par $15: \hrulefill$
    \task $\dfrac{6}{2}$ amplifiée par $16: \hrulefill$
\end{tasks}
}{1}\end{exop}

\begin{exop}{
Rends les fractions suivantes irréductibles.
\begin{tasks}[after-item-skip = 0.4em, after-skip=-0.5em](3)
    \task $\dfrac{5}{15}$
    \task $\dfrac{20}{35}$
    \task $\dfrac{98}{28}$
    \task $\dfrac{32}{12}$
    \task $\dfrac{16}{28}$
    \task $\dfrac{54}{108}$
\end{tasks}
}{1}\end{exop}

\begin{exop}{
Rends les fractions suivantes irréductibles.
\begin{tasks}[after-item-skip = 0.4em, after-skip=-0.5em](3)
    \task $\dfrac{144}{135}$
    \task $\dfrac{120}{280}$
    \task $\dfrac{82}{205}$
    \task $\dfrac{68}{76}$
    \task $\dfrac{51}{93}$
    \task $\dfrac{363}{66}$
\end{tasks}
}{2}\end{exop}

\begin{exo}{
Rends les fractions suivantes irréductibles.
\begin{tasks}[after-item-skip = 0.4em, after-skip=-0.5em](5)
\task $\dfrac{17}{68}$
\task $\dfrac{26}{65}$
\task $\dfrac{72}{24}$
\task $\dfrac{3}{51}$
\task $\dfrac{18}{81}$
\end{tasks}
}{1}\end{exo}

\begin{exo}{
Rends les fractions suivantes irréductibles.
\begin{tasks}[after-item-skip = 0.4em, after-skip=-0.5em](5)
\task $\dfrac{63}{84}$
\task $\dfrac{25}{75}$
\task $\dfrac{46}{69}$
\task $\dfrac{101}{63}$
\task $\dfrac{77}{121}$
\end{tasks}
}{1}\end{exo}


\begin{exof}{NO183}{66}{1}
\end{exof}
\begin{exof}{NO184}{67}{2}
\end{exof}

\begin{exo}{
Écris les nombres suivants sous forme d'un entier ou d'une fraction irréductible.
\begin{tasks}(6)
	\task $4,5$
	\task $0,1$
	\task $0,3$
	\task $67$
	\task $3,2$
	\task $12,5$
	\task $4$
	\task $1,7$
	\task $1,5$
	\task $0,\overline{6}$
	\task $0,\overline{1}$
	\task $0,\overline{8}$
\end{tasks}
}{1}\end{exo}

\begin{exo}{
Écris les nombres suivants sous forme d'un entier ou d'une fraction irréductible.
\begin{tasks}(6)
	\task $5,42$
	\task $6,50$
	\task $0,44$
	\task $65,1$
	\task $301,2$
	\task $10,04$
	\task $41,45$
	\task $2,75$
	\task $0,25$
	\task $0,\overline{2}$
	\task $0,\overline{3}$
\end{tasks}
}{1}\end{exo}

\begin{exo}{
Écris les nombres suivants sous forme d'un entier ou d'une fraction irréductible.
\begin{tasks}(3)
	\task $3,25$
	\task $0,062$
	\task $34,65$
	\task $1,364$
	\task $0,02$
	\task $9$
\end{tasks}
}{2}\end{exo}


\begin{exol}{NO193}{53}{1}
\end{exol}
\begin{exol}{NO194}{53}{2}
\end{exol}
\begin{exol}{NO195}{53}{3}
\end{exol}
\begin{exof}{NO201}{70}{1}
\end{exof}
\begin{exof}{NO205}{72}{1}
\end{exof}

\begin{resolu}{Comparaison de fractions}{Laquelle des deux fractions est la plus grand? Justifie ta réponse.

	{\color{blue} Il y a deux méthodes pour résoudre cet exercice:\begin{enumerate}
			\item[1.] Afin de comparer des fractions, on doit les mettre sur un dénominateur commun. La fraction qui a ensuite le plus grand numérateur est la plus grande.
			\item[2.] On transforme la fraction en nombre décimal et on compare les deux nombres obtenus.
\end{enumerate}}
\begingroup
\setlength{\lineskip}{4pt}
\begin{tasks}(1)
	\task $\dfrac{1}{4}$ ou $\dfrac{3}{4}$? Les deux fractions ont le même dénominateur. On compare les numérateurs et on a que $\dfrac{1}{4}<\dfrac{3}{4}$.
	\task $\dfrac{3}{5}$ ou $\dfrac{3}{7}$? Le plus petit commun multiple de $5$ et $7$ est $\text{ppmc}(5;7)=35$. Il faut donc amplifier $\dfrac{3}{5}$ par $7$ pour obtenir $\dfrac{21}{35}$ et amplifier $\dfrac{3}{7}$ par $5$ pour obtenir $\dfrac{15}{35}$. $\dfrac{3}{5}=\dfrac{21}{35}$ et $\dfrac{3}{7}=\dfrac{15}{35}$, donc $\dfrac{3}{5}>\dfrac{3}{7}$.
	\task $\dfrac{5}{6}$ ou $\dfrac{2}{4}$? En utilisant la deuxième méthode, $\dfrac{5}{6}=0,8\overline{3}$ et $\dfrac{2}{4}=0,5$. Ainsi $\dfrac{5}{6}>\dfrac{2}{4}$.
\end{tasks}
\endgroup
}{1}
\end{resolu}

\newpage

\begin{exo}{
Laquelle des deux fractions est la plus grande? Justifie ta réponse.
\begin{tasks}(3)
\task $\dfrac{1}{8}$ ou $\dfrac{1}{6}$
\task $\dfrac{7}{4}$ ou $\dfrac{5}{4}$
\task $\dfrac{1}{2}$ ou $\dfrac{2}{3}$
\task $\dfrac{2}{5}$ ou $\dfrac{3}{5}$
\task $\dfrac{2}{5}$ ou $\dfrac{2}{7}$
\task $\dfrac{3}{5}$ ou $\dfrac{3}{4}$
\end{tasks}
}{1}\end{exo}

\begin{exo}{
Laquelle des deux fractions est la plus petite? Justifie ta réponse.

\begin{tasks}(3)
\task $\dfrac{5}{12}$ ou $\dfrac{7}{12}$
\task $\dfrac{1}{2}$ ou $\dfrac{6}{11}$
\task $\dfrac{6}{5}$ ou $\dfrac{11}{10}$
\task $\dfrac{8}{7}$ ou $\dfrac{4}{3}$
\task $\dfrac{3}{7}$ ou $\dfrac{9}{21}$
\task $\dfrac{4}{6}$ ou $\dfrac{6}{9}$
\end{tasks}
}{1}\end{exo}


\begin{exof}{NO185}{68}{1}
\end{exof}
\begin{exol}{NO178}{51}{1}
\end{exol}
\begin{exol}{NO179}{51}{2}
\end{exol}

\begin{exo}{
Ecris les nombres suivants par ordre croissant.
\begin{tasks}(2)
\task $\dfrac{1}{2} ;\quad \dfrac{1}{3} ;\quad \dfrac{1}{4} ;\quad \dfrac{1}{5} ;\quad \dfrac{1}{6}$
\task $\dfrac{2}{5} ;\quad \dfrac{5}{2} ;\quad \dfrac{2}{3} ;\quad \dfrac{3}{2}$
\task $\dfrac{1}{5} ;\quad \dfrac{2}{5} ;\quad \dfrac{3}{5} ;\quad \dfrac{4}{5} ;\quad \dfrac{5}{5}$
\task $\dfrac{3}{8} ;\quad \dfrac{8}{3} ;\quad \dfrac{1}{5} ;\quad \dfrac{5}{2}$
\end{tasks}
}{1}\end{exo}

\begin{exo}{
Ecris les nombres suivants par ordre décroissant.
\begin{tasks}(3)
\task $\dfrac{1}{3} ;\quad 0,3 ;\quad 0,4$
\task $\dfrac{1}{3} ;\quad 0,33 ;\quad 0,34$
\task $\dfrac{1}{3} ;\quad 0,333 ;\quad 0,05$
\end{tasks}
}{2}\end{exo}


\exol{NO196}{53}{2}

\begin{exo}{
Ecris les nombres suivants par ordre croissant.
	\begin{tasks}(3)
\task $1,04 ;\quad 1,044 ;\quad \dfrac{7}{5}$
\task $6,20 ;\quad \dfrac{25}{4} ;\quad 6,3$
\task $0,04 ;\quad \dfrac{3}{100} ;\quad 0,05$
\end{tasks}
}{2}
\end{exo}
%\exol{NO197}{53}{1}




\begin{qmoodle}{Fractions équivalentes
}{3}{
	\begin{center}	
		Q1

\includegraphics[scale=1]{media/qr/feq1}

\tiny{{https://edu.ge.ch/qr/feq1}}
\end{center}
	\begin{center}	
		Q2

\includegraphics[scale=1]{media/qr/feq2}

\tiny{{https://edu.ge.ch/qr/feq2}}
\end{center}
	\begin{center}	
		Q3

\includegraphics[scale=1]{media/qr/feq3}

\tiny{{https://edu.ge.ch/qr/feq3}}
\end{center}
}
\end{qmoodle}


%\begin{exopn}
%Une consigne {\cursive{{\color{blue}un texte}}}
%\begin{enumerate}
%\medskip\item Le $\;$\hrulefill$\;$ 432 est composé des $\;$\hrulefill$\;$ 2, 3 et 4.
%\end{enumerate}
%\end{exopn}

%\begin{exop}{
%Consigne
%}{0}\end{exop}


\end{document}
