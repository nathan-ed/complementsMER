\documentclass[a4paper,11pt]{report}
\usepackage[showexo=true,showcorr=false]{../packages/coursclasse}
%Commenter ou enlever le commentaire sur la ligne suivante pour montrer le niveau
\toggletrue{montrerNiveaux}
%permet de gérer l'espacement entre les items des env enumerate et enumitem
\usepackage{enumitem}
\setlist[enumerate]{align=left,leftmargin=1cm,itemsep=10pt,parsep=0pt,topsep=0pt,rightmargin=0.5cm}
\setlist[itemize]{align=left,labelsep=1em,leftmargin=*,itemsep=0pt,parsep=0pt,topsep=0pt,rightmargin=0cm}
%permet de gerer l'espacement entre les colonnes de multicols
\setlength\columnsep{35pt}
\usetikzlibrary{arrows.meta}



\begin{document}
%%%%%%%%%%%%%%%%% À MODIFIER POUR CHAQUE SERIE %%%%%%%%%%%%%%%%%%%%%%%%%%%%%
\newcommand{\chapterName}{Nombres et opérations}
\newcommand{\serieName}{Encadrer et arrondir}


%%%%%%%%%%%%%%%%%% PREMIERE PAGE NE PAS MODIFER %%%%%%%%%%%%%%%%%%%%%%%%
% le chapitre en cours, ne pas changer au cours d'une série
\chapter*{\chapterName}
\thispagestyle{empty}

%%%%% LISTE AIDE MEMOIRE %%%%%%
\begin{amL}{\serieName}{
\item  Approximation d’un nombre décimal (page 22)
}
\end{amL}
%%%%%%%%%%%%%%% DEBUT DE LA SERIE NE PAS MODIFIER %%%%%%%%%%%%%%%%%%%%%%%%%%%%%
\section*{\serieName}
\setcounter{page}{1}
\thispagestyle{firstPage}



%%%%%%%%%%% LES EXERCICES %%%%%%%%%%%%%%%%%%%%%%%%%%%%%%%%%%%

\begin{resolu}{Encadrer le nombre 235,867}{Encadre le nombre 235,867 à la précision demandée en t'aidant des droites graduées ci-dessous~:
\begin{tasks}
\task Encadre le nombre {\bf 235,867} à l'unité près.
\begin{center}
\begin{numberlined}{230}{240}{1}{0.1}{230/230, 231/231, 232/232, 233/233, 234/234, 235/235, 236/236, 237/237, 238/238, 239/239, 240/240}{0.5}{0}{235.867/{235,867}}\end{numberlined}
\end{center}
{\bf Réponse~:} Voici l'encadrement à l'unité près de {\bf 235,867}~: 

$235  < \bm{235,867} < 236.$
\task Encadre le nombre {\bf 235,867} à la dizaine près.
\begin{center}
\begin{numberlined}{200}{250}{10}{0.1}{200/200, 210/210, 220/220, 230/230, 240/240, 250/250}{0.5}{0}{235.867/{235,867}}\end{numberlined}
\end{center}
{\bf Réponse~:} Voici l'encadrement à la dizaine près de {\bf 235,867}~: 

$230  < \bm{235,867} <  240.$
\task Encadre le nombre {\bf 235,867} à la centaine près.
\begin{center}
\begin{numberlined}{100}{500}{100}{0.1}{100/100,200/200,300/300,400/400,500/500}{0.5}{0}{235.867/{235,867}}\end{numberlined}
\end{center}
{\bf Réponse~:} Voici l'encadrement à la centaine près de {\bf 235,867}~: 

$200  < \bm{235,867} <  300.$
\task Encadre le nombre {\bf 235,867} au dixième près.
\begin{center}
\begin{numberlined}{232}{238}{1}{0.1}{232/232,233/233,234/234,235/235,236/236,237/237,238/238}{0.5}{1}{235.867/{235,867}}\end{numberlined}
\end{center}
{\bf Réponse~:} Voici l'encadrement au dixième près de {\bf 235,867}~: 

$235,8  < \bm{235,867} <  235,9.$
\end{tasks}
}{1}
\end{resolu}

\begin{exop}{
Encadre le nombre 82,576 à la précision demandée en t'aidant des droites graduées ci-dessous : 
\begin{tasks}
\task Encadre le nombre {\bf 82,576} à l'unité près.
\begin{center}
\begin{numberlined}{80}{90}{1}{0.1}{80/80, 81/81, 82/82, 83/83, 84/84, 85/85, 86/86, 87/87, 88/88, 89/89, 90/90}{0.5}{0}{82.576/{82,576}}\end{numberlined}
\end{center}
{\bf Réponse~:} Voici l'encadrement à l'unité près de {\bf 82,576}~: 

\makebox[.8in]{\hrulefill} $ < \bm{82,576} < $ \makebox[.8in]{\hrulefill} .
\task Encadre le nombre {\bf 82,576} à la dizaine près.
\begin{center}
\begin{numberlined}{50}{100}{10}{0.1}{50/50, 60/60, 70/70, 80/80, 90/90, 100/100}{0.5}{0}{82.576/{82,576}}\end{numberlined}
\end{center}
{\bf Réponse~:} Voici l'encadrement à la dizaine près de {\bf 82,576}~: 

\makebox[.8in]{\hrulefill} $  < \bm{82,576} <  $ \makebox[.8in]{\hrulefill} .
\task Encadre le nombre {\bf 82,576} au dixième près.
\begin{center}
\begin{numberlined}{80}{86}{1}{0.1}{80/80,81/81,82/82,83/83,84/84,85/85,86/86}{0.5}{1}{82.576/{82,576}}\end{numberlined}
\end{center}

{\bf Réponse~:} Voici l'encadrement au dixième près de {\bf 82,576}~: 

 \makebox[.8in]{\hrulefill} $ < \bm{82,576} < $  \makebox[.8in]{\hrulefill} .
\end{tasks}
}{1}
\end{exop}

\begin{exop}
{Encadre chacun des nombres suivants à l'unité près, en t'aidant de la droite numérique ci-dessous si nécessaire~:
\begin{center}
\vspace*{0.2cm}
\begin{numberlined}{20}{30}{1}{0.1}{20/20,21/21,22/22,23/23,24/24,25/25,26/26,27/27,28/28,29/29,30/30}{0.5}{0}{}\end{numberlined}
\vspace*{0.2cm}
\end{center}
\begin{tasks}[after-item-skip = 0.1em, after-skip=-2em](2)
\task \makebox[.8in]{\hrulefill} $<$ 26,18 $<$ \makebox[.8in]{\hrulefill}\\
\task \makebox[.8in]{\hrulefill} $<$ 27,9 $<$ \makebox[.8in]{\hrulefill}\\
\task \makebox[.8in]{\hrulefill} $<$ 29,83 $<$ \makebox[.8in]{\hrulefill}\\
\task \makebox[.8in]{\hrulefill} $<$ 28,29 $<$ \makebox[.8in]{\hrulefill}\\
\task \makebox[.8in]{\hrulefill} $<$ 21,4 $<$ \makebox[.8in]{\hrulefill}\\
\task \makebox[.8in]{\hrulefill} $<$ 24,65 $<$ \makebox[.8in]{\hrulefill}
\end{tasks}
}{1}
\end{exop}



\begin{exop}{
Encadre chacun des nombres suivants à l'unité près, en t'aidant d'une droite numérique si nécessaire~:
\begin{tasks}[after-item-skip = 0.15em, after-skip=-2em](2)
\task \makebox[.8in]{\hrulefill} $<$ 6,58 $<$ \makebox[.8in]{\hrulefill}\\
\task \makebox[.8in]{\hrulefill} $<$ 17,9 $<$ \makebox[.8in]{\hrulefill}\\
\task \makebox[.8in]{\hrulefill} $<$ 29,83 $<$ \makebox[.8in]{\hrulefill}\\
\task \makebox[.8in]{\hrulefill} $<$ 108,49 $<$ \makebox[.8in]{\hrulefill}\\
\task \makebox[.8in]{\hrulefill} $<$ 999,4 $<$ \makebox[.8in]{\hrulefill}\\
\task \makebox[.8in]{\hrulefill} $<$ 0,65 $<$ \makebox[.8in]{\hrulefill}
\end{tasks}
}{1}\end{exop}


\begin{exop}{
Encadre chacun des nombres suivants à la dizaine près, en t'aidant d'une droite numérique si nécessaire~:
\begin{tasks}[after-item-skip = 0.15em, after-skip=-2em](2)
\task \makebox[.8in]{\hrulefill} $<$ 18 $<$ \makebox[.8in]{\hrulefill}\\
\task \makebox[.8in]{\hrulefill} $<$ 42,8 $<$ \makebox[.8in]{\hrulefill}\\
\task \makebox[.8in]{\hrulefill} $<$ 1094 $<$ \makebox[.8in]{\hrulefill}\\
\task \makebox[.8in]{\hrulefill} $<$ 4,02 $<$ \makebox[.8in]{\hrulefill}\\
\task \makebox[.8in]{\hrulefill} $<$ 193,4 $<$ \makebox[.8in]{\hrulefill}\\
\task \makebox[.8in]{\hrulefill} $<$ 0,6 $<$ \makebox[.8in]{\hrulefill}
\end{tasks}
}{1}\end{exop}
\begin{exop}{
Encadre chacun des nombres suivants au dixième près, en t'aidant d'une droite numérique si nécessaire~:
\begin{tasks}[after-item-skip = 0.15em, after-skip=-2em](2)
\task \makebox[.7in]{\hrulefill} $<$ 6,58 $<$ \makebox[.7in]{\hrulefill}\\
\task \makebox[.7in]{\hrulefill} $<$ 0,503 $<$ \makebox[.7in]{\hrulefill}\\
\task \makebox[.7in]{\hrulefill} $<$ 29,83 $<$ \makebox[.7in]{\hrulefill}\\
\task \makebox[.7in]{\hrulefill} $<$ 11,93 $<$ \makebox[.7in]{\hrulefill}\\
\task \makebox[.7in]{\hrulefill} $<$ 1,49999 $<$ \makebox[.7in]{\hrulefill}\\
\task \makebox[.7in]{\hrulefill} $<$ 0,04 $<$ \makebox[.7in]{\hrulefill}\\
\end{tasks}
}{1}\end{exop}
\begin{exop}{
Encadre chacun des nombres suivants à la centaine près, en t'aidant d'une droite numérique si nécessaire~:
\begin{tasks}[after-item-skip = 0.15em, after-skip=-2em](2)
\task \makebox[.7in]{\hrulefill} $<$ 284 $<$ \makebox[.7in]{\hrulefill}\\
\task \makebox[.7in]{\hrulefill} $<$ 42,3 $<$ \makebox[.7in]{\hrulefill}\\
\task \makebox[.7in]{\hrulefill} $<$ 183,7 $<$ \makebox[.7in]{\hrulefill}\\
\task \makebox[.7in]{\hrulefill} $<$ 1903 $<$ \makebox[.7in]{\hrulefill}\\
\task \makebox[.7in]{\hrulefill} $<$ 836'780 $<$ \makebox[.7in]{\hrulefill}\\
\task \makebox[.7in]{\hrulefill} $<$ 0,8 $<$ \makebox[.7in]{\hrulefill}\\
\end{tasks}
}{1}\end{exop}
\begin{exop}{
Encadre chacun des nombres suivants au centième près, en t'aidant d'une droite numérique si nécessaire~:
\begin{tasks}[after-item-skip = 0.15em, after-skip=-2em](2)
\task \makebox[.7in]{\hrulefill} $<$ 6,58 $<$ \makebox[.7in]{\hrulefill}\\
\task \makebox[.7in]{\hrulefill} $<$ 0,503 $<$ \makebox[.7in]{\hrulefill}\\
\task \makebox[.7in]{\hrulefill} $<$ 3,0048 $<$ \makebox[.7in]{\hrulefill}\\
\task \makebox[.7in]{\hrulefill} $<$ 11,993 $<$ \makebox[.7in]{\hrulefill}\\
\task \makebox[.7in]{\hrulefill} $<$ 0,0073 $<$ \makebox[.7in]{\hrulefill}\\
\task \makebox[.7in]{\hrulefill} $<$ 26,548 $<$ \makebox[.7in]{\hrulefill}\\
\end{tasks}
}{1}\end{exop}


\begin{exo}{
Encadre chacun des nombres suivants par deux entiers successifs, en t'aidant d'une droite numérique si nécessaire~:
\begin{tasks}[after-item-skip = 0.2em, after-skip=-0.5em](3)
\task 10,8
\task 0,72
\task 1,08
\task 4,25
\task 99,3
\task1309,5
\end{tasks}
}{1}\end{exo}


\begin{exo}{
Encadre chacun des nombres suivants à la dizaine près, en t'aidant d'une droite numérique si nécessaire~:
\begin{tasks}[after-item-skip = 0.2em, after-skip=-0.5em](3)
\task 27
\task 31,6
\task 708,2
\task 134
\task 96,5
\task 0,45
\end{tasks}
}{1}\end{exo}
\begin{exo}{
Encadre chacun des nombres suivants au dixième près, en t'aidant d'une droite numérique si nécessaire~:
\begin{tasks}[after-item-skip = 0.2em, after-skip=-0.5em](3)
\task 8,12
\task 0,65
\task 0,07
\task 14,909
\task 3,94
\task 200,21
\end{tasks}
}{1}\end{exo}
\begin{exo}{
Encadre chacun des nombres suivants à la centaine près, en t'aidant d'une droite numérique si nécessaire~:
\begin{tasks}(3)
\task 6728
\task 394,52
\task 14,5
\task 8428
\task4 232 390
\task 6934,2
\end{tasks}
}{1}\end{exo}
\begin{exo}{
Encadre chacun des nombres suivants à la centième près, en t'aidant d'une droite numérique si nécessaire~:
\begin{tasks}(3)
\task 4,252
\task 6,694
\task 3,0085
\task 2,092
\task 0,0056
\task 3,283
\end{tasks}
}{1}\end{exo}
\begin{exo}{
Encadre chacun des nombres suivants au dixième près, en t'aidant d'une droite numérique si nécessaire~:
\begin{tasks}(3)
\task 623,48
\task 0,0475
\task 52,625
\task 82,98
\task 23,0052
\task 5,208
\end{tasks}
}{1}\end{exo}
\begin{exo}{
Encadre chacun des nombres suivants par deux entiers successifs, en t'aidant d'une droite numérique si nécessaire~:
\begin{tasks}(3)
\task 6 289,3
\task 9 999,9
\task 100,04
\task 620,92
\task 319,57
\task 889,3
\end{tasks}
}{1}\end{exo}


\begin{resolu}{Arrondir le nombre 235,867}{Arrondis le nombre 235,867 à la précision demandée en t'aidant des droites graduées ci-dessous~:
\begin{tasks}
\task Arrondis le nombre {\bf 235,867} à l'unité près.
\begin{center}
\begin{numberlined}{230}{240}{1}{0.1}{230/230, 231/231, 232/232, 233/233, 234/234, 235/235, 236/236, 237/237, 238/238, 239/239, 240/240}{0.5}{0}{235.867/{235,867}}\end{numberlined}
\end{center}
%\begin{center}
%\begin{DroiteGraduee}{15}{230}{240}{1}{1}{0}{0}
%\AfficheFleche{235.867}{.5}[\bm{235,867}]
%\end{DroiteGraduee} 
%\end{center}
{\bf Réponse~:} Voici l'arrondi à l'unité près de {\bf 235,867}~: 

$\bm{235,867 } \approx 236.$
\task Arrondis le nombre {\bf 235,867} à la dizaine près.
\begin{center}
\begin{numberlined}{200}{250}{10}{0.1}{200/200, 210/210, 220/220, 230/230, 240/240, 250/250}{0.5}{0}{235.867/{235,867}}\end{numberlined}
\end{center}
%\begin{center}
%\begin{DroiteGraduee}{15}{200}{250}{1}{10}{0}{0}
%\AfficheFleche{235.867}{.5}[\bm{235,867}]
%\end{DroiteGraduee} 
%\end{center}
{\bf Réponse~:} Voici l'arrondi à la dizaine près de {\bf 235,867}~: 

$\bm{235,867 } \approx 240.$
\task Arrondis le nombre {\bf 235,867} à la centaine près.
\begin{center}
\begin{numberlined}{100}{500}{100}{0.1}{100/100,200/200,300/300,400/400,500/500}{0.5}{0}{235.867/{235,867}}\end{numberlined}
\end{center}
%\begin{center}
%\begin{DroiteGraduee}{15}{100}{500}{1}{100}{0}{0}
%\AfficheFleche{235.867}{.5}[\bm{235,867}]
%\end{DroiteGraduee} 
%\end{center}
{\bf Réponse~:} Voici l'arrondi à la centaine près de {\bf 235,867}~: 

$\bm{235,867 } \approx 200.$
\task Arrondis le nombre {\bf 235,867} au dixième près.
\begin{center}
\begin{numberlined}{232}{238}{1}{0.1}{232/232,233/233,234/234,235/235,236/236,237/237,238/238}{0.5}{1}{235.867/{235,867}}\end{numberlined}
\end{center}
%\begin{center}
%\begin{DroiteGraduee}{15}{232}{238}{10}{1}{0}{0}
%\AfficheFleche{235.867}{.5}[\bm{235,867}]
%\end{DroiteGraduee} 
%\end{center}
{\bf Réponse~:} Voici l'arrondi au dixième près de {\bf 235,867}~: 

$\bm{235,867 } \approx 235,9.$
\end{tasks}
}{1}
\end{resolu}


\begin{exop}{
Arrondis le nombre 82,576 à la précision demandée en t'aidant des droites graduées ci-dessous~: 
\begin{tasks}
\task Arrondis le nombre {\bf 82,576} à l'unité près.
\begin{center}
\begin{numberlined}{80}{90}{1}{0.1}{80/80, 81/81, 82/82, 83/83, 84/84, 85/85, 86/86, 87/87, 88/88, 89/89, 90/90}{0.5}{0}{82.576/{82,576}}\end{numberlined}
\end{center}
{\bf Réponse~:} Voici l'arrondi à l'unité près de {\bf 82,576}~: 

{\bf 82,576} $\approx$ \makebox[.8in]{\hrulefill} .
\task Arrondis le nombre {\bf 82,576} à la dizaine près.
\begin{center}
\begin{numberlined}{50}{100}{10}{0.1}{50/50, 60/60, 70/70, 80/80, 90/90, 100/100}{0.5}{0}{82.576/{82,576}}\end{numberlined}
\end{center}
{\bf Réponse~:} Voici l'arrondi à la dizaine près de {\bf 82,576}~: 

{\bf 82,576} $\approx$ \makebox[.8in]{\hrulefill} .
\task Arrondis le nombre {\bf 82,576} au dixième près.
\begin{center}
\begin{numberlined}{80}{86}{1}{0.1}{80/80,81/81,82/82,83/83,84/84,85/85,86/86}{0.5}{1}{82.576/{82,576}}\end{numberlined}
\end{center}

{\bf Réponse~:} Voici l'arrondi au dixième près de {\bf 82,576}~: 

{\bf 82,576} $\approx$ \makebox[.8in]{\hrulefill} .
\end{tasks}
}{1}
\end{exop}


\begin{exop}
{Arrondis chacun des nombres suivants à l'unité près, en t'aidant de la droite numérique ci-dessous si nécessaire~:
\begin{center}
\vspace*{0.1cm}
\begin{numberlined}{20}{30}{1}{0.1}{20/20,21/21,22/22,23/23,24/24,25/25,26/26,27/27,28/28,29/29,30/30}{0.5}{0}{}\end{numberlined}
\vspace*{0.1cm}
\end{center}
\begin{tasks}[after-item-skip = 0em, after-skip=-2em](2)
\task  26,18 $\approx$ \hrulefill \\
\task   27,9 $\approx$ \hrulefill \\
\task   29,83 $\approx$ \hrulefill \\
\task   28,29 $\approx$ \hrulefill \\
\task   21,4 $\approx$ \hrulefill \\
\task   24,65 $\approx$ \hrulefill \\
\end{tasks}
}{1}
\end{exop}


\begin{exop}{
Arrondis à l'unité la plus proche les nombres suivants, en t'aidant d'une droite numérique si nécessaire~:
\begin{tasks}[after-item-skip = 0.12em, after-skip=-1.5em](2)
\bigskip\task 109,75 $\approx$\hrulefill 
\bigskip\task 46,8 $\approx$ \hrulefill 
\bigskip\task 1,3 $\approx$ \hrulefill 
\bigskip\task 0,09 $\approx$\hrulefill 
\bigskip\task 234,08 $\approx$\hrulefill 
\bigskip\task 0,7 $\approx$\hrulefill 
\bigskip\task 3,14 $\approx$\hrulefill 
\bigskip\task 4087,63 $\approx$ \hrulefill 
\end{tasks}

}{1}\end{exop}

\begin{exop}{
Arrondis à la dizaine  la plus proche les nombres suivants, en t'aidant d'une droite numérique si nécessaire~:
\begin{tasks}[after-item-skip = 0.2em, after-skip=-1.5em](2)
\bigskip\task 109,75 $\approx$ \hrulefill 
\bigskip\task 47,8 $\approx$ \hrulefill 
\bigskip\task 1,3 $\approx$ \hrulefill 
\bigskip\task 0,09 $\approx$ \hrulefill 
\bigskip\task 234,2 $\approx$ \hrulefill 
\bigskip\task 0,7 $\approx$ \hrulefill 
\bigskip\task 3,14 $\approx$\hrulefill 
\bigskip\task 4087,63 $\approx$\hrulefill 
\end{tasks}
}{1}\end{exop}

\begin{exop}{
Arrondis au dixième le plus proche les nombres suivants, en t'aidant d'une droite numérique si nécessaire~:
\begin{tasks}[after-item-skip = 0.2em, after-skip=-1.5em](2)
\bigskip\task 8,372$\approx$ \hrulefill 
\bigskip\task 50,64 $\approx$\hrulefill 
\bigskip\task 3000,088 $\approx$\hrulefill 
\bigskip\task 43,725 $\approx$ \hrulefill 
\bigskip\task 0,02 $\approx$ \hrulefill
\bigskip\task 1,09 $\approx$ \hrulefill 
\bigskip\task 0,98 $\approx$ \hrulefill 
\bigskip\task 78,66 $\approx$ \hrulefill 
\end{tasks}
}{1}\end{exop}


\begin{exop}{
Arrondis à l'unité la plus proche les nombres suivants, en t'aidant d'une droite numérique si nécessaire~:
\begin{tasks}[after-item-skip = 0.2em, after-skip=-1.5em](2)
\bigskip\task 149,75 $\approx$\hrulefill  
\bigskip\task 446,8 $\approx$ \hrulefill  
\bigskip\task 13,3 $\approx$ \hrulefill  
\bigskip\task 3,09 $\approx$\hrulefill  
\bigskip\task 264,08 $\approx$\hrulefill  
\bigskip\task 0,88 $\approx$\hrulefill  
\bigskip\task 3,15 $\approx$\hrulefill  
\bigskip\task 4787,43 $\approx$ \hrulefill  
\end{tasks}
}{1}\end{exop}

\begin{exop}{
Arrondis à la dizaine  la plus proche les nombres suivants, en t'aidant d'une droite numérique si nécessaire~:
\begin{tasks}[after-item-skip = 0.2em, after-skip=-1.5em](2)
\bigskip\task 139,75 $\approx$ \hrulefill  
\bigskip\task 447,8 $\approx$ \hrulefill  
\bigskip\task 44,3 $\approx$ \hrulefill  
\bigskip\task 191,09 $\approx$ \hrulefill  
\bigskip\task 232,2 $\approx$ \hrulefill  
\bigskip\task 42,997 $\approx$ \hrulefill  
\bigskip\task 423,14 $\approx$\hrulefill  
\bigskip\task 4437,63 $\approx$\hrulefill  
\end{tasks}
}{1}\end{exop}

\begin{exop}{
Arrondis au dixième le plus proche les nombres suivants, en t'aidant d'une droite numérique si nécessaire~:
\begin{tasks}(2)
\bigskip\task 148,482 $\approx$ \hrulefill  
\bigskip\task 55,94 $\approx$\hrulefill  
\bigskip\task 35,179 $\approx$\hrulefill  
\bigskip\task 45,745 $\approx$ \hrulefill  
\bigskip\task 0,92 $\approx$ \hrulefill  
\bigskip\task 1,99 $\approx$ \hrulefill  
\bigskip\task 0,18 $\approx$ \hrulefill  
\bigskip\task 78,76 $\approx$ \hrulefill  
\end{tasks}
}{1}\end{exop}

\newpage 

\begin{qmoodle}{Arrondir un nombre décimal}{2}{
	\begin{center}	
		Q-1

\includegraphics[scale=1]{media/qr/ndeea1}

\tiny{{https://edu.ge.ch/qr/ndeea1}}
\end{center}
	\begin{center}	
		Q-2

\includegraphics[scale=1]{media/qr/ndeea2}

\tiny{{https://edu.ge.ch/qr/ndeea2}}
\end{center}
}
\end{qmoodle}

\begin{resolu}
{Le plus proche}{Pour chaque question ci-dessous, justifie tes réponses par des calculs.
\begin{tasks}[after-item-skip = 0.2em, after-skip=-0.5em]
\task Le nombre le plus proche de 1 est-il 1,3 ou 0,8~?

{\bf Réponse :} On calcule les écarts entre 1,3 et 1 et entre 1 et 0,8. $$1,3-1=0,3\qquad 1-0,8 = 0,2.$$
Ainsi le nombre le plus proche est le nombre qui a l'écart le plus petit de 1. Donc le nombre le plus proche de 1 est 0,8.
 
\task Le nombre le plus proche de 5 est-il 4,73 ou 5,265~?

{\bf Réponse :} On calcule les écarts entre 5 et 4,73 et entre 5,265 et 5. $$5-4,73=0,27 \qquad 5,265-5=0,265.$$

Ainsi le nombre le plus proche est le nombre qui a l'écart le plus petit. Donc le nombre le plus proche de 5 est 5,265.
\end{tasks}
}{1}
\end{resolu}

\begin{exo}{
Pour chaque question ci-dessous, justifie tes réponses par des calculs.
\begin{tasks}[after-item-skip = 0.2em, after-skip=-0.5em,before-skip=-0.5em]

\task Le nombre le plus proche de 6 est-il 5,64 ou 5,642~?
\task Le nombre le plus proche de 8 est-il 8,28 ou 8,32~?
\task Le nombre le plus proche de 11 est-il 11,001 ou 10,99~?
\end{tasks}
}{1}\end{exo}


\begin{exo}{
Pour chaque question ci-dessous, justifie tes réponses par des calculs.
\begin{tasks}[after-item-skip = 0.2em, after-skip=-0.5em, before-skip=-0.5em]

\task Le nombre le plus proche de 4 est-il 3,75 ou 3,82~?
\task Le nombre le plus proche de 9 est-il 9,58 ou 9,43~?
\task Le nombre le plus proche de 14 est-il 13,459 ou 14,54~?
\end{tasks}
}{1}\end{exo}

\end{document}
