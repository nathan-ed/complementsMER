\documentclass[a4paper,11pt]{report}
\usepackage[showexo=true,showcorr=false,showdegree=true]{../packages/coursclassed}
%Commenter ou enlever le commentaire sur la ligne suivante pour montrer le niveau
\toggletrue{montrerNiveaux}
%permet de gérer l'espacement entre les items des env enumerate et enumitem
\usepackage{enumitem}
\setlist[enumerate]{align=left,leftmargin=1cm,itemsep=10pt,parsep=0pt,topsep=0pt,rightmargin=0.5cm}
\setlist[itemize]{align=left,labelsep=1em,leftmargin=*,itemsep=0pt,parsep=0pt,topsep=0pt,rightmargin=0cm}
%permet de gerer l'espacement entre les colonnes de multicols
\setlength\columnsep{35pt}

\usepackage{numprint}

\begin{document}

%%%%%%%%%%%%%%%%% À MODIFIER POUR CHAQUE SERIE %%%%%%%%%%%%%%%%%%%%%%%%%%%%%
\newcommand{\chapterName}{Nombres et opérations}
\newcommand{\serieName}{Nombres premiers}

%%%%%%%%%%%%%%%%%% PREMIERE PAGE NE PAS MODIFER %%%%%%%%%%%%%%%%%%%%%%%%
% le chapitre en cours, ne pas changer au cours d'une série
\chapter*{\chapterName}
\thispagestyle{empty}

%%%%% LISTE AIDE MEMOIRE %%%%%%
\begin{amL}{\serieName}{
\item Nombre premier (page 13)
\item Décomposition en produit de facteurs premiers (page 14)
}\end{amL}

%%%%%%%%%%%%%%% DEBUT DE LA SERIE NE PAS MODIFIER %%%%%%%%%%%%%%%%%%%%%%%%%%%%%
\section*{\serieName}
\setcounter{page}{1}

%%%%%%%%%%% LES EXERCICES %%%%%%%%%%%%%%%%%%%%%%%%%%%%%%%%%%%%

\begin{exo}{
    Énumère les diviseurs des entiers suivants. 
    \begin{tasks}(4)
    \task $7$
    \task $13$
    \task $31$
    \task $3$
    \task $51$
    \task $23$
    \task $47$
    \task $2$
\end{tasks}
}{1}\end{exo}

\begin{exof}{NO14}{9}{1}
\end{exof}

\begin{exol}{NO15}{14}{1}
\end{exol}

\begin{exol}{NO16}{14}{1}
\end{exol}



\begin{exo}{ %seymath
    Parmi les nombres suivants, lesquels ne sont pas premiers et pourquoi~? Utilise les critères de divisibilité.
    \begin{tasks}(5)
    \task $58$
    \task $19$
    \task $60$
    \task $741$
    \task $549$
\end{tasks}
}{1}\end{exo}    


\begin{exo}{ %seymath
    Parmi les nombres suivants, lesquels ne sont pas premiers et pourquoi~? Utilise les critères de divisibilité.
    \begin{tasks}(5)
    \task $258$
    \task $109$
    \task $605$
    \task $4731$
    \task $654219$
\end{tasks}
}{1}\end{exo}

\begin{exol}{NO17}{14}{1}
\end{exol}

\begin{exol}{NO18}{15}{1}
\end{exol}

\begin{exo}{ %seymath
    Écris chaque nombre sous la forme d'une multiplication de deux facteurs.
    \begin{tasks}(4)
    \task $12$
    \task $30$
    \task $39$
    \task $60$
    \task $27$
    \task $111$
    \task $180$
    \task $200$
\end{tasks}
}{1}\end{exo}

\begin{exo}{ %seymath
    Écris chaque nombre sous la forme d'une multiplication de deux facteurs.
    \begin{tasks}(4)
    \task $48$
    \task $32$
    \task $68$
    \task $72$
    \task $14$
    \task $117$
    \task $56$
    \task $33$
\end{tasks}
}{1}\end{exo}




\begin{exol}{NO20}{15}{1}
\end{exol}



%-------------------------------------------------------------
\begin{exo}{
    Décompose les nombres suivants en produit de facteurs premiers.
    \begin{tasks}(5)
    \task $12$
    \task $49$
    \task $54$
    \task $84$
    \task $150$
    \task $180$
    \task $525$
    \task $528$
    \task $600$
    \task $630$
    \end{tasks}
}{1}\end{exo}



%-------------------------------------------------------------
\begin{exo}{ %seymath
    Décompose les nombres suivants en produit de facteurs premiers.
    \begin{tasks}(3)
    \task $4200$
    \task $4700$
    \task $3600$
    \task $150000$
    \task $10200$
    \task $81900$
    \end{tasks}
}{1}\end{exo}


%------------------------------------------------------






%------------------------------------------------------
\begin{exo}{
    Calcule le pgdc des entiers suivants en utilisant la décomposition en produit de facteurs premiers.

    \begin{tasks}(2)
    \task $45$ et $60$
    \task $36$ et $42$
    \task $126$ et $180$
    \task $176$, $112$ et $208$
    \task $90$, $125$ et $605$
    \task $\begin{aligned}
		    A&=2^2\cdot3^3\cdot5^3 \text{ et}\\
		    B&= 2\cdot3^2\cdot5\cdot7^2
	\end{aligned}$
\end{tasks}
}{1}\end{exo}


%------------------------------------------------------
\begin{exo}{
Calcule le ppmc des entiers suivants en utilisant la décomposition en produit de facteurs premiers.



\begin{tasks}(2)
    \task $500$ et $625$
    \task $180$ et $84$
    \task $214$ et $216$
    \task $256$ et $1024$
    \task $135$, $675$ et $63$
    \task $\begin{aligned}A &= 5^2\cdot7^5\cdot11^3 \text{ et}\\ B&= 3^2\cdot7\cdot11^3\cdot13^2
	    \end{aligned}$
\end{tasks}
}{1}\end{exo}


\begin{FLP}{11}{1}
\end{FLP}


\end{document}
