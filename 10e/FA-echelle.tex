\documentclass[a4paper,11pt]{report}
\usepackage[showexo=true,showcorr=false,showdegree=true]{../packages/coursclassed}
%Commenter ou enlever le commentaire sur la ligne suivante pour montrer le niveau
\toggletrue{montrerNiveaux}
%permet de gérer l'espacement entre les items des env enumerate et enumitem
\usepackage{enumitem}
\setlist[enumerate]{align=left,leftmargin=1cm,itemsep=10pt,parsep=0pt,topsep=0pt,rightmargin=0.5cm}
\setlist[itemize]{align=left,labelsep=1em,leftmargin=*,itemsep=0pt,parsep=0pt,topsep=0pt,rightmargin=0cm}
%permet de gerer l'espacement entre les colonnes de multicols
\setlength\columnsep{35pt}

\usepackage{numprint}
\usepackage{tabularx}

\begin{document}

%%%%%%%%%%%%%%%%% À MODIFIER POUR CHAQUE SERIE %%%%%%%%%%%%%%%%%%%%%%%%%%%%%
\newcommand{\chapterName}{Fonctions et algèbre}
\newcommand{\serieName}{Proportionnalité : Les échelles}

%%%%%%%%%%%%%%%%%% PREMIERE PAGE NE PAS MODIFER %%%%%%%%%%%%%%%%%%%%%%%%
% le chapitre en cours, ne pas changer au cours d'une série
\chapter*{\chapterName}
\thispagestyle{empty}

%%%%% LISTE AIDE MEMOIRE %%%%%%
\begin{amL}{\serieName}{
\item Proportionnalité - Généralités (page 55)
\item Résoudre un problème de proportionnalité (page 57)
\item Échelle (page 59)
\item Déterminer l'échelle d'une reproduction de la réalité (page 59)
}\end{amL}

%%%%%%%%%%%%%%% DEBUT DE LA SERIE NE PAS MODIFIER %%%%%%%%%%%%%%%%%%%%%%%%%%%%%
\section*{\serieName}
\setcounter{page}{1}

%%%%%%%%%%% LES EXERCICES %%%%%%%%%%%%%%%%%%%%%%%%%%%%%%%%%%%%


\begin{resolu}{Calculer l'échelle d'une réduction (carte, plan, maquette...)}{
\tunit{20}{\kilo m} en réel sont représentés sur une carte par \tunit{5}{\centi m}. Quelle est l'échelle de cette carte ?

\begin{tasks}(1)
    \task Exprime les distances dans la même unité.

    $\tunit{20}{\kilo m} = \numprint{2000000}$ cm

    \task Applique la formule et exprime l'échelle sous la forme d'une fraction dont le numérateur est égal à $1$.
    
    \begin{align*}        \textrm{Échelle}&=\dfrac{\textrm{Longueur mesurée sur la carte}}{\textrm{Longueur réelle}} \\
    &=\dfrac{5}{\numprint{2000000}} \\
    &=\dfrac{1}{\numprint{400000}} 
    \end{align*}
    
    \task L'échelle de la carte est $1:\numprint{400000}$.
\end{tasks}
}{1}    
\end{resolu}

\begin{resolu}{Calculer l'échelle d'un agrandissement}{
\tunit{2}{\milli m} en réel sont représentés sur un dessin par \tunit{14}{\centi m}. Quelle est l'échelle de ce dessin ?

\begin{tasks}(1)
    \task Exprime les distances dans la même unité.

    $\tunit{14}{\centi m} = \numprint{140}$ mm

    \task Applique la formule et exprime l'échelle sous la forme d'une fraction dont le dénominateur est égal à $1$.
    
    \begin{align*}        \textrm{Échelle}&=\dfrac{\textrm{Longueur mesurée sur le dessin}}{\textrm{Longueur réelle}} \\
    &=\dfrac{140}{2} \\
    &=\dfrac{70}{1} 
    \end{align*}
    
    \task L'échelle de la carte est $70:1$.
\end{tasks}
}{1}    
\end{resolu}


\begin{exo}{
\begin{tasks}(1)
    \task Quelle est l'échelle d'une carte sur laquelle \tunit{28}{\kilo m} sont représentés par \tunit{0,56}{\kilo m} ?

    \task Quelle est l'échelle d'un dessin sur laquelle \tunit{96}{\milli m} sont représentés par \tunit{576}{\milli m} ?

    \task Quelle est l'échelle d'une maquette sur laquelle \tunit{42}{m} sont représentés par \tunit{0,14}{m} ?

    \task Quelle est l'échelle d'une image sur laquelle \tunit{32,5}{\centi m} sont représentés par \tunit{130}{\centi m} ?
\end{tasks}
}{1}    
\end{exo}

\begin{exo}{
\begin{tasks}(1)
    \task Quelle est l'échelle d'une carte sur laquelle \tunit{35}{\kilo m} sont représentés par \tunit{2,5}{\centi m} ?

    \task Quelle est l'échelle d'un dessin sur laquelle \tunit{1,5}{\milli m} sont représentés par \tunit{3,75}{\centi m} ?

    \task Quelle est l'échelle d'une maquette sur laquelle \tunit{2720}{m} sont représentés par \tunit{68}{\centi m} ?

    \task Quelle est l'échelle d'une image sur laquelle \tunit{12}{\centi m} sont représentés par \tunit{6}{m} ?
\end{tasks}
}{1}    
\end{exo}

\begin{exol}{FA53}{79}{1} % échelle coccinelle
\end{exol}

\begin{exo}{
    Lola souhaite acheter une parcelle rectangulaire dont les dimensions sont \tunit{70}{m} par \tunit{45}{m}. Son architecte lui montre un plan mesurant \tunit{28}{\centi m} sur \tunit{18}{\centi m}.

    Quelle est l'échelle de ce plan ?
}{1}    
\end{exo}

\begin{exo}{
    Un mille-pattes mesure \tunit{48}{\milli m}. Sur le dessin, il mesure \tunit{20,4}{\centi m}.

    Quelle est l'échelle de ce dessin ?
}{1}    
\end{exo}

\begin{resolu}{Calculer une longueur à l'aide d'une échelle (réduction)}{Sur une maquette de la ville de Genève, on indique l'échelle $1:500$.

\begin{tasks}(1)
    
\task Calcule la distance sur la maquette entre le Musée d'Histoire des Sciences et le Musée d'ethnographie, en sachant que ces deux établissements sont éloignés de $\tunit{2,8}{\kilo m}$ en réalité.

    L'échelle indique que les distances sur la maquette sont $500$ fois plus petites qu'en réalité. 
    
       \[2,8:500=\tunit{0,0056}{\kilo m}\]
        
    Convertis dans l'unité adaptée.

    \[\tunit{0,0056}{\kilo m}=\tunit{5,6}{m}\]

    \underline{Réponse} : Les deux musées sont éloignés de \tunit{5,6}{m} sur la maquette.
    
\task Sur la maquette, on mesure une distance de \tunit{6,8}{\centi m} entre l'aéroport et la gare Cornavin. Quelle distance cela représente-t-il en réalité ?

    L'échelle indique que les distances en réalité sont $500$ fois plus grandes que sur la maquette. 
    
    \[6,8\cdot500=\tunit{3400}{m}\]
        
    Convertis dans l'unité adaptée.

    \[\tunit{3400}{m}=\tunit{3,4}{\kilo m}\]  

    \underline{Réponse} : L'aéroport et la gare Cornavin sont éloignés de \tunit{3,4}{\kilo m} en réalité.
\end{tasks}
}{1}    
\end{resolu}








\begin{resolu}{Calculer une longueur à l'aide d'une échelle (agrandissement)}{On dessine un insecte à l'échelle $20:1$.

\begin{tasks}(1)
    
\task L'insecte mesure environ \tunit{9}{\milli m} de long.  Quelle sera la longueur du dessin de cet insecte ?

    L'échelle indique que les longueurs sur le dessin sont $20$ fois plus grande qu'en réalité. 
    
       \[9\cdot20=\tunit{180}{\milli m}\]
        
    Convertis dans l'unité adaptée.

    \[\tunit{180}{\milli m}=\tunit{18}{\centi m}\]

    \underline{Réponse} : L'insecte mesure \tunit{18}{\centi m} sur le dessin.
    
\task Sur le dessin, une patte de l'insecte mesure \tunit{3,2}{\centi m}. Quelle longueur cela représente-t-il en réalité ?

    L'échelle indique que les distances en réalité sont $20$ fois plus petites que sur le dessin. 
    
    \[3,2:20=\tunit{0,16}{\centi m}\]
        
    Convertis dans l'unité adaptée.

    \[\tunit{0,16}{\centi m}=\tunit{1,6}{\milli m}\]

    \underline{Réponse} : Cette patte mesure \tunit{1,6}{\milli m} en réalité.
\end{tasks}
}{0}    
\end{resolu}



\begin{exol}{FA51}{78}{1} % échelle VW
\end{exol}

\begin{exol}{FA52}{79}{1} % fermes carte
\end{exol}

\begin{exol}{FA54}{79}{1} %échelle insecte 8:1
\end{exol}

\begin{exof}{FA55}{89}{1} %vol oiseau
\end{exof}


\begin{exo}{
Sur le plan d'une maison à l'échelle $1:50$, on représente des murs de \tunit{5,1}{m} et \tunit{7,2}{m} de long. Quelle est la longueur des murs sur le plan ?
}{1}    
\end{exo}

\begin{exo}{
Sur une carte, on indique $1:\numprint{70000}$. Tu y mesures une longueur de \tunit{17,5}{\centi m}. Que représente cette distance en réalité ?
}{1}    
\end{exo}

\begin{exo}{
Fanta dessine une vis à l'échelle $2,5:1$.
\begin{tasks}
    \task Quelle sera la longueur d'un clou sur ce même dessin s'il mesure réellement \tunit{45}{\milli m} ?
    \task La longueur de la vis sur le dessin est de \tunit{75}{\centi m}. Quelle est sa longueur en réalité ?
\end{tasks}
}{1}
\end{exo}









%%%%%% AJOUTER SERIES DEXERCICES VOIR EXO RESOLU POUR UN CHANGEMENT DECHELLE

\begin{exo}{
Sur une carte au $1:\numprint{40000}$, la distance entre deux point est de \tunit{68}{\centi m}.

Quelle est la distance en centimètres entre ces deux points sur une carte au $1:\numprint{1000000}$ ?
}{2}    
\end{exo}


\begin{exo}{
Deux villages sont à une distance de \tunit{15}{\kilo m}. Quelle serait la distance entre ces deux villes sur une carte à l'échelle $1:\numprint{100000}$ ?
}{1}
\end{exo}

\begin{exo}{
Sur un plan à l'échelle $1:50$ la longueur d'une maison est de \tunit{40}{\centi m} alors que sur un plan à l'échelle $1:100$ sa largeur est de \tunit{1,5}{\deci m}. Quelles sont les dimensions réelles de la maison ?
}{2}
\end{exo}
\begin{exol}{FA56}{79}{1} %échelle carte
\end{exol}

\begin{exol}{FA57}{80}{1} % plan, maquette, à mesurer
\end{exol}

\begin{exof}{FA88}{93}{2} %dinosaure
\end{exof}


\begin{exol}{FA77}{87}{1} % pente + échelle
\end{exol}



\begin{exo}{
Sur une carte, tu mesures une distance de \tunit{5}{cm} entre la station de départ d'un télésiège et la station d'arrivée. La pente moyenne de ce télésiège est de $20\%$.
La station supérieure est à \tunit{2000}{m} d'altitude et le départ se situe à \tunit{1500}{m}.

Quelle est l'échelle de cette carte ?
}{1}
\end{exo}


\begin{exol}{FA75}{87}{1}
\end{exol}

\end{document}
