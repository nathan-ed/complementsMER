\documentclass[a4paper,11pt]{report}
\usepackage[showexo=true,showcorr=false,showdegree=true]{../packages/coursclassed}
%Commenter ou enlever le commentaire sur la ligne suivante pour montrer le niveau
\toggletrue{montrerNiveaux}
%permet de gérer l'espacement entre les items des env enumerate et enumitem
\usepackage{enumitem}
\setlist[enumerate]{align=left,leftmargin=1cm,itemsep=10pt,parsep=0pt,topsep=0pt,rightmargin=0.5cm}
\setlist[itemize]{align=left,labelsep=1em,leftmargin=*,itemsep=0pt,parsep=0pt,topsep=0pt,rightmargin=0cm}
%permet de gerer l'espacement entre les colonnes de multicols
\setlength\columnsep{35pt}

\begin{document}

%%%%%%%%%%%%%%%%% À MODIFIER POUR CHAQUE SERIE %%%%%%%%%%%%%%%%%%%%%%%%%%%%%
\newcommand{\chapterName}{Nombres et opérations}
\newcommand{\serieName}{Multiplications et divisions de  relatifs}

%%%%%%%%%%%%%%%%%% PREMIERE PAGE NE PAS MODIFER %%%%%%%%%%%%%%%%%%%%%%%%
% le chapitre en cours, ne pas changer au cours d'une série
\chapter*{\chapterName}
\thispagestyle{empty}

%%%%% LISTE AIDE MEMOIRE %%%%%%
\begin{amL}{\serieName}{
\item Distance à zéro (page 17)
\item Multiplication de nombres relatifs (page 20)
\item Division de nombres relatifs (page 20)
}\end{amL}

%%%%%%%%%%%%%%% DEBUT DE LA SERIE NE PAS MODIFIER %%%%%%%%%%%%%%%%%%%%%%%%%%%%%
\section*{\serieName}
\setcounter{page}{1}

%%%%%%%%%%% LES EXERCICES %%%%%%%%%%%%%%%%%%%%%%%%%%%%%%%%%%%%

\resolu{Multiplication de deux nombres relatifs}{Observe les 8 multiplications suivantes de deux nombres relatifs afin de pouvoir proposer une règle de calcul.

\begin{tasks}(2)
\task $(+5)\cdot(+7)=+35$
\task $(-5)\cdot(-7)=+35$
\task $(+5)\cdot(-7)=-35$
\task $(-5)\cdot(+7)=-35$
\task $(+6)\cdot(+9)=+54$
\task $(-6)\cdot(-9)=+54$
\task $(+6)\cdot(-9)=-54$
\task $(-6)\cdot(+9)=-54$
\end{tasks}

{\bf  Règle de de calcul pour la multiplication de deux nombres relatifs :} Pour multiplier deux nombres relatifs, on multiplie leurs distances à zéro et on donne au produit:
 \begin{itemize}
\item le signe $+$ si les deux nombres sont de même signe. 
\item le signe $-$ si les deux nombres sont de signes différents.
 \end{itemize}
}{2}


\exop{Observe les 8 multiplications suivantes de deux nombres relatifs afin de pouvoir compléter la règle de calcul.

\begin{tasks}(2)
\task $(+11)\cdot(+12)=+132$
\task $(-11)\cdot(-12)=+132$
\task $(+11)\cdot(-12)=-132$
\task $(-11)\cdot(+12)=-132$
\task $(+7)\cdot(+8)=+56$
\task $(-7)\cdot(-8)=+56$
\task $(+7)\cdot(-8)=-56$
\task $(-7)\cdot(+8)=-56$
\end{tasks}

{\bf Règle de de calcul pour la multiplication de deux nombres relatifs :} Pour multiplier deux nombres relatifs, on multiplie leurs \ligne{6} et on donne au \ligne{5}:
\begin{itemize}
\item le signe \ligne{1} si les deux nombres sont de \ligne{5}. 
\item le signe \ligne{1} si les deux nombres sont de \ligne{5}.
\end{itemize}
}{2}

\exop{Effectue les multiplications de nombres relatifs ci-dessous.
\begin{tasks}(2)
\task $(-3) \cdot (-4) = $
\task $(+7) \cdot (+2) = $
\task $(-5) \cdot (-3) = $
\task $(+6) \cdot (+1) = $
\task $(-4) \cdot (-5) = $
\task $(+2) \cdot (+6) = $
\end{tasks}
}{2}

\exop{Calcule les produits de nombres relatifs ci-dessous.
\begin{tasks}(2)
\task $(+8) \cdot (-3) = $
\task $(-7) \cdot (+2) = $
\task $(+6) \cdot (-5) = $
\task $(-4) \cdot (+1) = $
\task $(+3) \cdot (-2) = $
\task $(-10) \cdot (+4) = $
\end{tasks}
}{2}


\exop{Complète par un entier relatif.
\begin{tasks}(2)
\task $(+8) \cdot \ligne{2}= -32$
\task $\ligne{2} \cdot (+2) = -14$
\task $(-6) \cdot \ligne{2} =  -42$
\task $\ligne{2} \cdot (+11) = -77$
\task $(+3) \cdot \ligne{2} =+21$
\task $\ligne{2} \cdot (-4) = -20$
\end{tasks}
}{2}

\exop{Complète par un entier relatif.
\begin{tasks}(2)
\task $(+6) \cdot \ligne{2} = -36$
\task $\ligne{2} \cdot (-3) = 12$
\task $(-5) \cdot \ligne{2} =  50$
\task $\ligne{2} \cdot (+7) = -21$
\task $(+4) \cdot \ligne{2} = -28$
\task $\ligne{2} \cdot (-2) = 10$
\end{tasks}
}{2}

\exop{Complète par un entier relatif.
\begin{tasks}(2)
\task $(+7) \cdot \ligne{2} = \phantom{-}63$
\task $\ligne{2} \cdot (-5) = \phantom{-}35$
\task $(-9) \cdot \ligne{2} = -99$
\task $\ligne{2} \cdot (+6) = -66$
\task $(+12) \cdot \ligne{2} = -84$
\task $\ligne{2} \cdot (-8) = \phantom{-}56$
\end{tasks}
}{2}

\exol{NO65}{27}{2}
\exol{NO66}{27}{2}
\exof{NO67}{20}{2}
\exof{NO68}{20}{2}

\resolu{Division de deux nombres relatifs}{
\begin{enumerate}
\item Calcule chaque quotient ci-dessous en t'aidant de la multiplication d'à côté.

\begin{tabular}{clcl}
a) & $(+45) : (+3) = +15 $ & car &  $(+3) \cdot (+15) = (+45)$. \\
b) & $(-45) : (-3) = +15 $ & car &  $(-3) \cdot (+15) = (-45)$. \\
c) & $(-45) : (+3) = -15 $ & car &  $(+3) \cdot (-15) = (-45)$. \\
d) & $(+45) : (-3) = -15 $ & car &  $(-3) \cdot (-15) = (+45)$.
\end{tabular}

\item Propose une règle de calcul pour la division de deux nombres relatifs.

{\bf  Règle de de calcul pour la division de deux nombres relatifs :} Pour diviser deux nombres relatifs, on divise leurs distances à zéro et on donne au quotient:
 \begin{enumerate}
\item le signe $+$ si les deux nombres sont de même signe. 
\item le signe $-$ si les deux nombres sont de signes différents.
 \end{enumerate}
\end{enumerate}}{2}


\exop{
\begin{enumerate}
\item Calcule chaque quotient ci-dessous en t'aidant de la multiplication d'à côté.

\begin{tabular}{clcl}
a) & $(+65) : (+5) =$ \ligne{2} & car &  $(+5) \cdot (+13) = (+65)$. \\
b) & $(-65) : (-5) =$ \ligne{2} & car &  $(-5) \cdot (+13) = (-65)$. \\
c) & $(-65) : (+5) =$ \ligne{2} & car &  $(+5) \cdot (-13) = (-65)$. \\
d) & $(+65) : (-5) =$ \ligne{2} & car &  $(-5) \cdot (-13) = (+65)$.
\end{tabular}
\item Complète la règle de calcul pour la division de deux nombres relatifs.

{\bf  Règle de de calcul pour la division de deux nombres relatifs :} Pour diviser deux nombres relatifs, on divise leurs \ligne{6} et on donne au \ligne{4}:
 \begin{enumerate}
\item le signe \ligne{1} si les deux nombres sont de \ligne{5}. 
\item le signe \ligne{1} si les deux nombres sont de \ligne{5}.
 \end{enumerate}
\end{enumerate}}{2}

\exop{Calcule chaque quotient ci-dessous en t'aidant de la multiplication d'à côté. 

%\begin{flushleft}
\begin{tabular}{clcl}
a) & $(-99) : (+11) =$ \ligne{2} & car &  $(+11) \cdot (-9) = (-99)$. \\
b) & $(+77) : (-7) =$ \ligne{2} & car &  $(-7) \cdot (-11) = (+77)$. \\
c) & $(+48) : (+6) =$ \ligne{2} & car &  $(+6) \cdot (+8) = (+48)$. \\
d) & $(-84) : (-12) =$ \ligne{2} & car &  $(-12) \cdot (+7) = (-84)$. \\
e) & $(-63) : (+9) =$ \ligne{2} & car &  $(+9) \cdot (-7) = (-63)$. \\
f) & $(+30) : (-5) =$ \ligne{2} & car &  $(-5) \cdot (-6) = (+30)$.
\end{tabular}
%\end{flushleft}

}{2}

\exop{Effectue les divisions de deux nombres relatifs ci-dessous. 

\begin{tasks}(2)
\task $(-35) : (-7) =$
\task $(-88) : (+11) =$
\task $(+40) : (-5) =$
\task $(+120) : (-12) =$
\task $(-49) : (+7) =$
\task $(-64) : (-8) =$
\end{tasks}
}{2}

\exop{Effectue les divisions de deux nombres relatifs ci-dessous.
\begin{tasks}(2)
\task $(-28) : (+4) =$
\task $(+63) : (-9) =$
\task $(-72) : (-8) =$
\task $(+90) : (+10) =$
\task $(-56) : (+8) =$
\task $(+36) : (-6) =$
\end{tasks}
}{2}


\exop{Effectue les divisions de deux nombres relatifs ci-dessous.
\begin{tasks}(2)
\task $(-45) : (-9) =$
\task $(+77) : (+11) =$
\task $(+56) : (-7) =$
\task $(-96) : (+8) =$
\task $(-63) : (-3) =$
\task $(+30) : (-6) =$
\end{tasks}
}{2}

\exop{Effectue les divisions de deux nombres relatifs ci-dessous.
\begin{tasks}(2)
\task $(-27) : (-3) =$
\task $(+55) : (-11) =$
\task $(-36) : (-6) =$
\task $(+72) : (+8) =$
\task $(-42) : (+7) =$
\task $(+45) : (-5) =$
\end{tasks}
}{2}

\exop{Effectue les divisions de deux nombres relatifs ci-dessous.
\begin{tasks}(2)
\task $(-2,4) : (-3) =$
\task $(+3,5) : (-7) =$
\task $(-4,8) : (-8) =$
\task $(+5,4) : (+9) =$
\task $(-2,8) : (+0,4) =$
\task $(+4,5) : (-0,5) =$
\end{tasks}
}{3}

\exop{Complète par un entier relatifs
\begin{tasks}(2)
\task $(-18) : \ligne{2} = 6$
\task $(+55) : \ligne{2} = -11$
\task $\ligne{2} : (-8) = -4$
\task $\ligne{2} : (+8) = -8$
\task $\ligne{2} : (+10) = -5$
\task $(+36) : \ligne{2} = (-6)$
\end{tasks}
}{2}

\exop{Complète par un entier relatif.
\begin{tasks}(2)
\task $(-27) : \ligne{2} = 9$
\task $(+66) : \ligne{2} = -22$
\task $\ligne{2} : (-6) = 4$
\task $\ligne{2} : (+2) = 8$
\task $\ligne{2} : (+6) = -5$
\task $(+48) : \ligne{2} = 6$
\end{tasks}
}{2}

\exop{Complète par un entier relatif.
\begin{tasks}(2)
\task $\ligne{2} : (-3) = 7$
\task $66 : \ligne{2} = -6$
\task $(-48) : \ligne{2} = -8$
\task $(-72) : \ligne{2} = -8$
\task $(-45) : \ligne{2} = -5$
\task $(+54) : \ligne{2} = -6$
\end{tasks}
}{2}

\exol{NO69}{27}{2}
\exol{NO70}{28}{2}
\exol{NO71}{28}{2}
\exof{NO72}{20}{2}
\exof{NO73}{20}{2}
\resolu{Quel est le signe ?}{Pour déterminer le signe quand on multiplie plusieurs nombres relatifs différents de zéro :
\begin{enumerate}
\item si le nombre de facteurs négatifs est pair, le produit est positif ;
\item si le nombre de facteurs négatifs est impair, le produit est négatif.
\end{enumerate}
Le nombre de facteurs positifs n'influence pas le signe du produit.
A l'aide de cette règle, détermine le signe des produits suivants en justifiant ta réponse.
\begin{tasks}(1)
\task $(+5)\cdot(+4)\cdot(+3)=+60$ car le nombre de facteurs négatifs (0) est pair. 
\task $(+5)\cdot(-4)\cdot(+3)=-60$ car le nombre de facteurs négatifs (1) est impair.
\task $(-5)\cdot(-4)\cdot(+3)=+60$ car le nombre de facteurs négatifs (2) est pair.
\task $(-5)\cdot(-4)\cdot(-3)=-60$ car le nombre de facteurs négatifs (3) est impair.
\task $(+5)\cdot(+4)\cdot(+3)\cdot(+2)=+120$ car le nombre de facteurs négatifs (0) est pair.
\end{tasks}
}{2}


\end{document}