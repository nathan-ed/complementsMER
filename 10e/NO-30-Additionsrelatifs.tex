\documentclass[a4paper,11pt]{report}
\usepackage[showexo=true,showcorr=false,showdegree=true]{../packages/coursclassed}
%Commenter ou enlever le commentaire sur la ligne suivante pour montrer le niveau
\toggletrue{montrerNiveaux}
%permet de gérer l'espacement entre les items des env enumerate et enumitem
\usepackage{enumitem}
\setlist[enumerate]{align=left,leftmargin=1cm,itemsep=10pt,parsep=0pt,topsep=0pt,rightmargin=0.5cm}
\setlist[itemize]{align=left,labelsep=1em,leftmargin=*,itemsep=0pt,parsep=0pt,topsep=0pt,rightmargin=0cm}
%permet de gerer l'espacement entre les colonnes de multicols
\setlength\columnsep{35pt}

\begin{document}

%%%%%%%%%%%%%%%%% À MODIFIER POUR CHAQUE SERIE %%%%%%%%%%%%%%%%%%%%%%%%%%%%%
\newcommand{\chapterName}{Nombres et opérations}
\newcommand{\serieName}{Additions de nombres relatifs}

%%%%%%%%%%%%%%%%%% PREMIERE PAGE NE PAS MODIFER %%%%%%%%%%%%%%%%%%%%%%%%
% le chapitre en cours, ne pas changer au cours d'une série
\chapter*{\chapterName}
\thispagestyle{empty}

%%%%% LISTE AIDE MEMOIRE %%%%%%
\begin{amL}{\serieName}{
\item Distance à zéro (page17)
\item Addition de nombres relatifs (page 18)
}\end{amL}

%%%%%%%%%%%%%%% DEBUT DE LA SERIE NE PAS MODIFIER %%%%%%%%%%%%%%%%%%%%%%%%%%%%%
\section*{\serieName}
\setcounter{page}{1}

%%%%%%%%%%% LES EXERCICES %%%%%%%%%%%%%%%%%%%%%%%%%%%%%%%%%%%%


\begin{resolu}
{De même signe}{Observe les 8 additions suivantes et rédige une règle permettant d'additionner deux nombres relatifs de même signe.
\begin{tasks}(2)
\task $(+5)+(+3)= +8$
\task $(-5)+(-3)= -8$
\task $(+13)+(+19)= +32$
\task $(-13)+(-19)= -32$
\task $(+27)+(+30)= +57$
\task $(-27)+(-30)= -57$
\task $(+18)+(+9)= +27$
\task $(-18)+(-9)= -27$
\end{tasks}

{\bf Règle pour l'addition de deux nombres relatifs de même signe:} Pour additionner deux nombres de même signe, on additionne les distances à zéro et on donne au résultat le signe commun.
}{1}
\end{resolu}

\begin{exop}
{Observe les 8 additions suivantes et rédige une règle permettant d'additionner deux nombres relatifs de même signe.
\begin{tasks}(2)
\task $(+8)+(+4)= +12$
\task $(-8)+(-4)= -12$
\task $(+15)+(+16)= +31$
\task $(-15)+(-16)= -31$
\task $(+23)+(+40)= +63$
\task $(-23)+(-40)= -63$
\task $(+16)+(+8)= +24$
\task $(-16)+(-8)= -24$
\end{tasks}

{\bf Règle pour l'addition de deux nombres relatifs de même signe:} 

\hrulefill

\hrulefill

}{1}
\end{exop}


\exo{Effectue les additions suivantes de deux nombres relatifs de même signe.

\begin{tasks}(2)
\task $(-3)+(-7)=$
\task $(-12)+(-18)=$
\task $(-9)+(-6)=$
\task $(+3)+(+7)=$
\task $(+13)+(+12)=$
\task $(+10)+(+14)=$
\task $(+4)+(+11)=$
\task $(-6)+(-4)=$
\task $(-8)+(-8)=$
\task $(+5)+(+17)=$
\end{tasks}}{1}



\begin{resolu}
{De signes différents}{Observe les 8 additions suivantes et rédige une règle permettant d'additionner deux nombres relatifs de signes différents.
\begin{tasks}(2)
\task $(+5)+(-3)= +2$
\task $(-5)+(+3)= -2$
\task $(+13)+(-19)= -6$
\task $(-13)+(+19)= +6$
\task $(+27)+(-30)= -3$
\task $(-27)+(+30)= +3$
\task $(+18)+(-9)= +9$
\task $(-18)+(+9)= -9$
\end{tasks}

{\bf Règle pour l'addition de deux nombres relatifs de signes différents:} Pour additionner deux nombres de signes différents, on soustrait la plus petite distance à zéro de la plus grande et on donne au résultat le signe du nombre qui a la plus grande distance à zéro.
}{1}
\end{resolu}

\begin{exop}
{Observe les 8 additions suivantes et rédige une règle permettant d'additionner deux nombres relatifs de signes différents.
\begin{tasks}(2)
\task $(+8)+(-4)= +4$
\task $(-8)+(+4)= -4$
\task $(-15)+(+16)= +1$
\task $(+15)+(-16)= -1$
\task $(-23)+(+40)= +17$
\task $(+23)+(-40)= -17$
\task $(-16)+(+8)= -8$
\task $(+16)+(-8)= +8$
\end{tasks}

{\bf Règle pour l'addition de deux nombres relatifs de signes différents:} 

\hrulefill

\hrulefill

}{1}
\end{exop}



\exop{Effectue les additions suivantes de deux nombres relatifs de signes différents.

\begin{tasks}(2)
\task $(-3)+(+7)=$
\task $(+12)+(-18)=$
\task $(+9)+(-6)=$
\task $(+3)+(-7)=$
\task $(+13)+(-12)=$
\task $(-10)+(+14)=$
\task $(-4)+(+11)=$
\task $(-6)+(+4)=$
\task $(-8)+(+9)=$
\task $(+5)+(-17)=$
\end{tasks}}{1}


\resolu{Additionner les opposés}{Observe les additions suivantes et complète la phrase :
\begin{tasks}(2)
\task $(-5)+(+5) = 0$
\task $(+7)+(-7) = 0$
\task $(-13)+(+13) = 0$
\task $(+18)+(-18) = 0$
\task $(-24)+(+24) = 0$
\task $(+32)+(-32) = 0$
\task $(-55)+(+55) = 0$
\task $(+72)+(-72) = 0$
\end{tasks}
Lorsqu'on additionne un nombre et son opposé, la somme vaudra toujours \bf{zéro}.
}{1}

\exop{Observe les additions suivantes et complète la phrase :
\begin{tasks}(2)
\task $(-4)+(+4) = 0$
\task $(+8)+(-8) = 0$
\task $(-12)+(+12) = 0$
\task $(+15)+(-15) = 0$
\task $(-22)+(+22) = 0$
\task $(+42)+(-42) = 0$
\task $(-59)+(+59) = 0$
\task $(+77)+(-77) = 0$
\end{tasks}
Lorsqu'on additionne un nombre et son opposé, la somme vaudra toujours \ligne{2}.}{1}

\resolu{Même signe ?}{Indique si les deux nombres qu'on souhaite additionner ont le même signe ou sont de signes différents, puis effectue cette addition.
\begin{center}
{\renewcommand{\arraystretch}{2}
\begin{tabular}{|l|l|l|}\hline
Addition & Même signe ? & Calcul \\\hline
$(+2)+(+6)$ & Même signe & $(+2)+(+6)=+8$ \\\hline
$(-8)+(+6)$ & Signes différents & $(-8)+(+6)=-2$ \\\hline
$(+2)+(-8)$ & Signes différents & $(+2)+(-8)=-6$ \\\hline
$(+10)+(+5)$ & Même signe & $(+10)+(+5)=+15$ \\\hline
$(-11)+(-7)$ & Même signe & $(-11)+(-7)=-18$ \\\hline
$(-7)+(+11)$ & Signes différents & $(-7)+(+11)=+4$ \\\hline
\end{tabular}}
\end{center}
}{1}

\exop{Indique si les deux nombres qu'on souhaite additionner ont le même signe ou sont de signes différents, puis effectue cette addition.
\begin{center}
{\renewcommand{\arraystretch}{2}
\begin{tabular}{|l|p{5cm}|p{5cm}|}\hline
Addition & Même signe ? & Calcul \\\hline
$(-7)+(+9)$ &  & $(-7)+(+9)=$ \\\hline
$(+3)+(+11)$ &  & $(+3)+(+11)=$  \\\hline
$(-4)+(-15)$ &  & $(-4)+(-15)=$  \\\hline
$(-11)+(+16)$ &  & $(-11)+(+16)=$  \\\hline
$(+17)+(-4)$ &  & $(+17)+(-4)=$ \\\hline
$(+13)+(+9)$ & & $(+13)+(+9)=$ \\\hline
\end{tabular}}
\end{center}
}{1}

\exo{Effectue les additions de deux nombres relatifs ci-dessous. Pense à vérifier si les deux nombres ont le même signe.
\begin{tasks}(2)
\task $(-11)+(+7) = $
\task $(+13)+(-9) = $
\task $(-4)+(-4) = $
\task $(+15)+(+8) = $
\task $(-19)+(-11) =$
\task $(+21)+(-19)=$
\task $(+23)+(+19)=$
\task $(+14)+(-13)=$
\end{tasks}
}{1}

\exo{Effectue les additions de deux nombres relatifs ci-dessous. Pense à vérifier si les deux nombres ont le même signe.
\begin{tasks}(2)
\task $(-7)+(+5) = $
\task $(+6)+(-8) = $
\task $(-10)+(-3) = $
\task $(+14)+(+9) = $
\task $(-17)+(-10) = $
\task $(+20)+(-15) = $
\task $(-23)+(+20) = $
\task $(+12)+(-11) =$
\task $(-4)+(+2) = $
\task $(+16)+(-14) = $
\task $(-9)+(+7) = $
\task $(+11)+(-13) = $
\end{tasks}
}{1}

\exo{Effectue les additions de deux nombres relatifs ci-dessous. Pense à vérifier si les deux nombres ont le même signe.
\begin{tasks}(2)
\task $(-4)+(+2) = $
\task $(-14)+(+5) = $
\task $(+16)+(-14) = $
\task $(-24)+(+12) = $
\task $(-9)+(+7) = $
\task $(-9)+(+9) = $
\task $(+11)+(-13) = $
\task $(-32)+(+13) = $
\end{tasks}
}{1}

\exo{Effectue les additions de deux nombres relatifs ci-dessous. Pense à vérifier si les deux nombres ont le même signe.
\begin{tasks}(2)
\task $(-37)+(+42) = $
\task $(+45)+(-31) = $
\task $(-33)+(-40) = $
\task $(+30)+(+48) = $
\task $(-50)+(-36) = $
\task $(+40)+(-35) = $
\task $(+44)+(+31) = $
\task $(-38)+(+44) =$
\end{tasks}
}{1}

\exo{Effectue les additions de deux nombres relatifs ci-dessous. Pense à vérifier si les deux nombres ont le même signe.
\begin{tasks}(2)
\task $(-25)+(+62) = $
\task $(+33)+(-41) = $
\task $(-70)+(-38) = $
\task $(+50)+(+27) = $
\task $(-20)+(-40) = $
\task $(+60)+(-35) = $
\task $(+44)+(+21) = $
\task $(-28)+(+65) =$
\end{tasks}
}{1}


\resolu{3 ou 4 nombres}{Effectue l'addition des nombres relatifs ci-dessous.
\begin{tasks}(1)
\task $(+5)+(-3)+(+12)=\underbrace{(+5)+(+12)}_{\mbox{même signe}}+(-3)=(+17)+(-3)=(+14)$
\task $(-5)+(+8)+(-14)=(+8)+\underbrace{(-5)+(-14)}_{\mbox{même signe}}=(+8)+(-19)=(-11)$
\task $(+17)+(-13)+(+7)+(-21)=\underbrace{(+17)+(+7)}_{\mbox{même signe}}+\underbrace{(-13)+(-21)}_{\mbox{même signe}}=(+24)+(-34)=(-10)$
\task $(-6)+(+11)+(+8)+(-20)=\underbrace{(+11)+(+8)}_{\mbox{même signe}}+\underbrace{(-6)+(-20)}_{\mbox{même signe}}=(+19)+(-26)=(-7)$
\end{tasks}
}{1}

\exo{Effectue l'addition des nombres relatifs ci-dessous.
\begin{tasks}(1)
\task $(+8) + (-5) + (+12) = $
\task $(-10) + (-3) + (+18) = $
\task $(+15) + (+1) + (-9) = $
\task $(-7) + (+6) + (-14) = $
\task $(+4) + (+11) + (-2) = $
\task $(-16) + (-8) + (+19) = $
\end{tasks}
}{1}

\exo{Effectue l'addition des nombres relatifs ci-dessous.
\begin{tasks}(1)
\task $(+25) + (-30) + (-12) + (+18) = $
\task $(-10) + (+20) + (-35) + (+30) = $
\task $(-30) + (-12) + (+15) + (-18) = $
\task $(+18) + (+22) + (-10) + (-35) = $
\task $(+30) + (-10) + (-35) + (+22) = $
\task $(-40) + (+10) + (-12) + (+15) = $
\end{tasks}
}{1}


\exo{Effectue l'addition des nombres relatifs ci-dessous.
\begin{tasks}(1)
\task $(+20) + (-15) + (+10) + (-12) + (+18) = $
\task $(-10) + (+30) + (+25) + (-35) + (-20) = $
\task $(-30) + (-12) + (+15) + (+25) + (-18) = $
\task $(+25) + (-20) + (+35) + (-30) + (-12) = $
\task $(+18) + (+22) + (-10) + (-35) + (+30) = $
\task $(-40) + (+10) + (-12) + (-25) + (+15) = $
\end{tasks}
}{1}


\exol{NO42}{22}{1}
\exol{NO43}{22}{1}
\exol{NO44}{23}{1}
\exof{NO45}{14}{1}
\exof{NO46}{14}{1}
\exof{NO47}{15}{1}

\resolu{Un ascenseur}{Dans un immeuble de 9 étages et 4 sous-sols, il y a un ascenseur. Il monte de 3 étages, descend ensuite de 4 étages, remonte de 9 étages, et finalement redescend de 7 étages. Après ces quatre mouvements, l'ascenseur se retrouve au deuxième sous-sol. À partir de quel étage l'ascenseur a-t-il commencé son trajet?

A l'aide des nombres relatifs, on peut traduire l'énoncé par le calcul suivant, où le premier nombre n'est pas connu : 
\begin{center}
$($ \ligne{1} $)+(+3)+(-4)+(+9)+(-7)=(-2)$
\end{center}

En calculant le membre de gauche de l'égalité, on obtient : 
\begin{center}
$($\ligne{1} $)+(+1)=(-2) $
\end{center}
Ainsi, on trouve : $\bf{(-3)} +(+1)=(-2)$

Donc on peut conclure que l'ascenseur a commencé son trajet depuis le troisième sous-sol.
}{1}

\exo{Dans un immeuble de 12 étages et 6 sous-sols, il y a un ascenseur. Il monte de 4 étages, descend ensuite de 3 étages, remonte de 8 étages, et finalement redescend de 5 étages. Après ces quatre mouvements, l'ascenseur se retrouve au troisième sous-sol. À partir de quel étage l'ascenseur a-t-il commencé son trajet?}{1}


\begin{exo}
{Genève connaît des variations de température au cours d'une semaine. Le lundi, la température monte de 5$^{o}$C, le mardi elle descend de 3 $^{o}$C, le mercredi elle remonte de 8 $^{o}$C, et le jeudi elle redescend de 6 $^{o}$C. Après ces quatre fluctuations de température, la semaine se termine avec une température finale de 18 $^{o}$C. Quelle était la température de départ le lundi?}{1}
\end{exo}

\exo{Vernier connaît des fluctuations de température au cours d'une semaine hivernale. Le lundi, la température chute de 7$^{o}$C, le mardi elle remonte de 4 $^{o}$C, le mercredi elle plonge de nouveau de 9 $^{o}$C, et le jeudi elle remonte de 5 $^{o}$C. Après ces quatre changements de température, la semaine se termine avec une température finale de -3$^{o}$C. Quelle était la température de départ le lundi?}{1}

\exo{Une région connaît des fluctuations de température au cours d'une semaine froide. Le lundi, la température chute de 9$^{o}$C, le mardi elle remonte de 7$^{o}$C, le mercredi elle plonge de nouveau de 8$^{o}$C, et le jeudi elle remonte de 3$^{o}$C. Après ces quatre changements de température, la semaine se termine avec une température finale de -5$^{o}$C. Quelle était la température de départ le lundi?}{1}

\exo{\begin{tasks}(1)
\task Lundi, Iris a joué deux parties de billes. A la première partie, elle a perdu 6 billes. En tout, elle a perdu 9 billes.
Que s'est-il passé à la deuxième partie ?
\task Mardi, Iris a joué deux parties de billes. A la première partie, elle a gagné 9 billes. En tout, elle a gagné 2 billes.
Que s'est-il passé à la deuxième partie ?
\task Mercredi, Iris a joué deux parties de billes. A la deuxième partie, elle a gagné 3 billes.
En tout, elle a gagné 8 billes.
Que s'est-il passé à la première partie ?
\task Jeudi, Iris a joué deux parties de billes. A la première partie, elle a perdu 12 billes.
En tout, elle a perdu 3 billes.
Que s'est-il passé à la deuxième partie ?
\task Vendredi, Iris a joué deux parties de billes. A la deuxième partie, elle a gagné 4 billes.
En tout, elle a perdu 5 billes.
Que s'est-il passé à la première partie ?
\task Samedi, Iris a encore joué deux parties de billes. A la première partie, elle a perdu 4 billes.
En tout, elle a gagné 5 billes.
Que s'est-il passé à la deuxième partie ?
Combien de billes Iris a-t-elle perdues ou gagnées au cours de la semaine ?
\end{tasks}}{1}


\exol{NO48}{24}{1}
\exol{NO49}{24}{1}

\end{document}
