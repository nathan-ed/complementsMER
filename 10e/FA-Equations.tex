\documentclass[a4paper,11pt]{report}
\usepackage[showexo=true,showcorr=false,showdegree=true]{../packages/coursclassed}
%Commenter ou enlever le commentaire sur la ligne suivante pour montrer le niveau
\toggletrue{montrerNiveaux}
%permet de gérer l'espacement entre les items des env enumerate et enumitem
\usepackage{enumitem}
\setlist[enumerate]{align=left,leftmargin=1cm,itemsep=10pt,parsep=0pt,topsep=0pt,rightmargin=0.5cm}
\setlist[itemize]{align=left,labelsep=1em,leftmargin=*,itemsep=0pt,parsep=0pt,topsep=0pt,rightmargin=0cm}
%permet de gerer l'espacement entre les colonnes de multicols
\setlength\columnsep{35pt}

\begin{document}

%%%%%%%%%%%%%%%%% À MODIFIER POUR CHAQUE SERIE %%%%%%%%%%%%%%%%%%%%%%%%%%%%%
\newcommand{\chapterName}{Fonctions et Algèbre}
\newcommand{\serieName}{Les équations}

%%%%%%%%%%%%%%%%%% PREMIERE PAGE NE PAS MODIFER %%%%%%%%%%%%%%%%%%%%%%%%
% le chapitre en cours, ne pas changer au cours d'une série
\chapter*{\chapterName}
\thispagestyle{empty}

%%%%% LISTE AIDE MEMOIRE %%%%%%

\begin{amL}{\serieName}{
\item Expression littéral (page 67)
\item Égalité de deux expressions littérales (page 68)
\item Calculer la valeur numérique d'une expression littérale connaissant la valeur de la lettre (page 74)
\item Équation (page 76)
\item Équations équivalentes (page 76)
\item Équation du premier degré à une inconnue (page 76)
\item Règles d'équivalence (page 77)
\item Résoudre une équation du premier degré à une inconnue (page 77)

}
\end{amL}



%%%%%%%%%%%%%%% DEBUT DE LA SERIE NE PAS MODIFIER %%%%%%%%%%%%%%%%%%%%%%%%%%%%%
\section*{\serieName}
\setcounter{page}{1}

%%%%%%%%%%% LES EXERCICES %%%%%%%%%%%%%%%%%%%%%%%%%%%%%%%%%%%%

\begin{QSJ}{118}{2}
\end{QSJ}

%intro
\begin{exol}{FA180}{116}{2}
\end{exol}

\begin{exof}{FA181}{119}{2}
\end{exof}

\begin{resolu}
{Intuition}
    {Détermine la longueur d'un triangle équilatéral ABC dont le périmètre vaut 12cm.
    \newline

   Dans un triangle équilatéral, les 3 côtés sont isométriques. Ainsi le périmètre se calcule par la formule suivante:
\[3*\text{côté}=\text{périmètre}
\]
Ainsi, périmètre/3=côté. 12/3=4.\\
Le côté mesure 4 cm.
    }
    {2}
\end{resolu}

\begin{exo}
{Le périmètre d'un losange ABCD est 36 cm. Quelle est la longueur de son côté?}
    {2}
\end{exo}


%solution d'une équation 

\begin{resolu}
    {Est solution de?}
    {Est-ce que 5 est solution de l'équation:
    \[4x+3=12
    \]
    \newline
\begin{solve}
    4*5&+3&\text{on substitue}\\
 20&+3 &\text{on calcule}\\
 23&\neq 12
\end{solve}
23 n'est pas égale à 12 donc 5 n'est pas solution de l'équation.    }
{2}
\end{resolu}


\begin{exo}
{Est-ce que 4 est solution de l'équation suivante?
\[
2x+5=3x-1
\]}
    {2}
\end{exo}

\begin{exo}
{Est-ce que 2 est solution de l'équation suivante?
\[
10+3x=14+x
\]}
    {2}
\end{exo}

\begin{exo}
{Est-ce que -5 est solution de l'équation suivante?
\[
7x-3=3x-1
\]}
    {2}
\end{exo}

\begin{exo}
{Est-ce que 0 est solution de l'équation suivante?
\[
2x^2+5x-\dfrac{3}{4}=2x^2-1
\]}
    {3}
\end{exo}

\begin{exo}
{Est-ce que $\dfrac{1}{2}$ est solution de l'équation suivante?
\[
4x+\dfrac{1}{2}=\dfrac{1}{4}x-1
\]}
    {3}
\end{exo}

\begin{exol}{FA193}{118}{2}
\end{exol}
%résolution 

\begin{resolu}{Résolution d'équations}{Résous les équations suivantes: 
\begin{enumerate}
    \item $x-2=4$
    \begin{solve}
     x-2& =4& +2\\
     x& = 6
     
\end{solve}

\item $x+1 = -2$

\[\begin{array}{rrcr}
    x+1 & = -2 \\
  -1 & -1 \\ \hline\\
   x&  = -3
    
    
\end{array}
\]
\end{enumerate}
}
{2}
\end{resolu}

\begin{exo}
    {Résous les équations suivantes en utilisant la présentation de ton choix.
\begin{tasks}(3)
\task $x+8=-4$
\task $x+7=3$
\task $x-11=10$
\task $x-9=10$
\task $x+5=-3$
\task $x+7=-7$
\end{tasks} }
{2}
\end{exo}

\begin{exo}
    {Résous les équations suivantes: 
\begin{tasks}(3)
\task $x+\dfrac{1}{2}=4$
\task $x+\dfrac{1}{3}=\dfrac{4}{5}$
\task $x-1+4=4$
\task $x-\dfrac{3}{2}=10$
\task $x-1+\dfrac{1}{2}=4$
\task $x-\dfrac{1}{2}+\dfrac{3}{4}=\dfrac{1}{4}$

\end{tasks} }
{3}
\end{exo}

\begin{resolu}{Résolution d'équations}{Résous les équations suivantes: 
\begin{enumerate}
    \item $2x=4$
    \begin{solve}
     2x& =4& \div 2\\
     x& = 2
     
\end{solve}

\item $3x = -2$

\[\begin{array}{rcr}
    3x & = -2 \\
  \div3& \div3\\ \hline\\
   x&  = \dfrac{-2}{3}
    
    
\end{array}
\]
\end{enumerate}
}
{2}
\end{resolu}

\begin{exo}
    {Résous les équations suivantes en utilisant la présentation de ton choix.
\begin{tasks}(3)
\task $3x=15$
\task $6x=42$
\task $14x=28$
\task $-5x=-15$
\task $-2x=8$
\task $12x=0$
\end{tasks} }
{2}
\end{exo}

\begin{exo}
    {Résous les équations suivantes en utilisant la méthode de ton choix.
\begin{tasks}(3)
\task $\dfrac{1}{3}x=15$
\task $6x=\dfrac{4}{5}$
\task $\dfrac{2}{6}x=\dfrac{5}{2}$
\task $-\dfrac{5}{4}x=-15$
\task $-2x=0$
\task $\dfrac{9}{7}x=0$
\end{tasks} }
{3}
\end{exo}

\begin{resolu}{Résolution d'équations}{Résous les équations suivantes: 
\begin{enumerate}
    \item $2x+1=4$
    \begin{solve}
     2x+1&=4&-1\\
     2x& =3& \div 2\\
     x& = \dfrac{3}{2}
     
\end{solve}

\item $3x  +5  = -2x-2$

\[\begin{array}{rrcr}
    3x & +5 & = -2 \\
    & -5 & =  &-5\\ \hline\\
    3x& & =    -7\\ 
  \div3& & \div3\\ \hline\\
   x&&  = \dfrac{-7}{3}
    
    
\end{array}
\]
\end{enumerate}
}
{2}
\end{resolu}

\begin{exo}
    {Résout les équations suivantes en utilisant la méthode de ton choix.
\begin{tasks}(3)
\task $2x+8=-4$
\task $5x+7=3$
\task $12x-11=10$
\task $3x-9=10$
\task $2x+5=-3$
\task $7x+7=-7$
\task $15x+3=-27$
\task $2x-5=7$
\task $6x+9=14$
\end{tasks} }
{2}
\end{exo}

\begin{exo}
    {Résout les équations suivantes en utilisant la méthode de ton choix.
\begin{tasks}(3)
\task $x+\dfrac{1}{2}=4$
\task $\dfrac{3}{4}x+2=3$
\task $12x-11=-11$
\task $\dfrac{2}{3}x-\dfrac{1}{6}=\dfrac{5}{2}$
\task $x+\dfrac{1}{3}=-3$
\task $7x+7=-\dfrac{1}{7}$
\task $\dfrac{2}{15}x+\dfrac{2}{5}=-\dfrac{1}{3}$
\task $\dfrac{2}{4}x-\dfrac{1}{2}=4$
\task $6x+\dfrac{2}{7}=14$
\end{tasks} }
{2}
\end{exo}


\begin{resolu}{Résolution d'équations}{Résous les équations suivantes: 
\begin{enumerate}
    \item $2x+1=4-3x$
    \begin{solve}
     2x+1&=4-3x&-1\\
     2x& =3-3x& +3x\\
     5x& = 3& \div5\\
     x&=\dfrac{3}{5}
     
\end{solve}

\item $3x  +5  = -2x-2$

\[\begin{array}{rrcrr}
    3x & +5 & = -2x &-2 \\
    & -5  & &-5\\ \hline\\
    3x& & =   -2x& -7\\ 
    +2x&& +2x&\\ \hline\\
    5x & &=-7\\
  \div5& & \div5\\ \hline\\
   x&&  = \dfrac{-7}{5}   
\end{array}
\]
\end{enumerate}
}
{2}
\end{resolu}

\begin{exo}
    {Résous les équations suivantes en utilisant la présentation de ton choix.
\begin{tasks}(3)
\task $8x-11=2x+5$
\task $18+2x=10x-22$
\task $3x+7=2x+5$
\task $4x+8=8+10x$
\task $13-10x=21-8x$
\task $12x+5=3x-4$
\task $3-5x=-2-3x$
\task $6x+13=-7x-5$
\task $-7-9x=-4-5x$
\end{tasks} }
{2}
\end{exo}

\begin{exo}
    {Résous les équations suivantes en utilisant la présentation de ton choix.
\begin{tasks}(2)
\task $2x-11=2x+5$
\task $18+2x=10x+18$
\task $\dfrac{1}{3}x+7=x+5$
\task $\dfrac{2}{3}x+\dfrac{1}{3}=\dfrac{5}{3}+x$
\task $1-\dfrac{1}{2}x=1-8x$
\task $12x+5+3x=3x-4-6$
\task $3-\dfrac{1}{5}x=-\dfrac{1}{25}-3x$
\task $\dfrac{6}{3}x+\dfrac{36}{2}=-24x-12$
\task $-\dfrac{2}{7}-\dfrac{35}{49}x=-4-x$
\end{tasks} }
{3}
\end{exo}

\begin{exo}
    {Résous les équations suivantes en utilisant la présentation de ton choix.
\begin{tasks}(2)
\task $2(x-11)=9$
\task $-5(3x+7)=5x-3$
\task $\dfrac{1}{3}(x-4)=3x$
\task $-5(x-4)=6(x+2)$
\task $1-(4-2x)=2-(2-x)$
\task $2x+2=x-(3-x)$
\task $5+(x+2)=(4+1)+2$
\task $3(x+2)=4(x+3)$
\task $x-(7x-2)=5(2-x)$
\end{tasks} }
{3}
\end{exo}

\begin{exof}{FA196}{122}{2}  
\end{exof}

\begin{exof}{FA201}{123}{2}
\end{exof}


\begin{exol}{FA190}{117}{2}
\end{exol}

\begin{exol}{FA197}{119}{2}
\end{exol}

\begin{exol}{FA198}{120}{2}
\end{exol}
\begin{exol}{FA199}{120}{2}
\end{exol}

%mise en équation

\begin{resolu}{Mise en équation}{Marie et Nathan ont dépensé 45,50.- en bonbons. Sachant que Marie a acheté pour 15.- de bonbons, combien a dépensé Nathan? 
Exprime la dépense de Nathan en fonction de celle de Marie, puis résous l'équation.\\

\begin{enumerate}
    \item $x$ est l'inconnue et représente la dépense de Nathan.
    \item On met en équation. $x+15=45,50$.
    \item On résout. $45,50-15=x$\\
    $30,50=x$
    \item La solution: Nathan a dépensé 30,50.-
    \item On vérifie: $15 + 30,50=45,50$. La réponse est correcte
\end{enumerate}
}
{2}
\end{resolu}

\begin{exol}{FA187}{117}{2}
\end{exol}

\begin{exo}
  {Je pense à un nombre. Je le multiplie par 3 puis lui ajoute 4. Le résultat me donne 40. Quel est mon nombre de départ?
\begin{enumerate}
    \item Quelle est l'inconnue?
    \item Pose une équation te permettant de résoudre le problème.
    \item Résous l'équation.
    \item Vérifie ta réponse.
\end{enumerate}
}
 {2} 
\end{exo}

\begin{exo}
  {Je pense à un nombre. Je lui ajoute 6, puis multiplie le résultat par 4. Je trouve 124. Quel est mon nombre de départ?
\begin{enumerate}
    \item Quelle est l'inconnue?
    \item Pose une équation te permettant de résoudre le problème.
    \item Résous l'équation.
    \item Vérifie ta réponse.
\end{enumerate}
}
 {2} 
\end{exo}

\begin{exol}{FA204}{121}{2}
\end{exol}
\begin{exol}{FA207}{122}{2}
\end{exol}
\begin{exol}{FA208}{122}{2}
\end{exol}
\begin{exol}{FA209}{122}{2}
\end{exol}
\begin{exol}{FA210}{123}{2}
\end{exol}
\begin{exol}{FA211}{123}{2}
\end{exol}
\begin{exol}{FA212}{123}{2}
\end{exol}
\begin{exol}{FA213}{124}{2}
\end{exol}
\begin{exol}{FA214}{124}{2}
\end{exol}
\begin{exol}{FA215}{124}{2}
\end{exol}
\begin{exol}{FA216}{124}{2}
\end{exol}
\begin{exol}{FA217}{124}{2}
\end{exol}
\begin{exol}{FA218}{124}{2}
\end{exol}
\begin{exol}{FA219}{124}{2}
\end{exol}


\begin{exo}
{L'aire du rectangle EFGH est 50 $cm^2$. Sachant que sa longueur est deux fois plus grande que sa largeur, quelles sont les mesures de ses côtés?
\begin{enumerate}
    \item Quelle est l'inconnue?
    \item Détermine l'équation.
    \item Résous l'équation.
    \item Vérifie ta réponse.
\end{enumerate}}
  {2}
\end{exo}

\begin{exo}
{Le carré EFGH à un périmètre égal à son aire. Quelle est la longueur de son côté?
\begin{enumerate}
    \item Quelle est l'inconnue?
    \item Détermine l'équation.
    \item Résous l'équation.
    \item Vérifie ta réponse.
\end{enumerate}}
 {2}
\end{exo}

\begin{exo}
{ Je choisis un nombre. Je lui ajoute son triple. J'obtiens 44. Quel est ce nombre?\begin{enumerate}
    \item Quelle est l'inconnue?
    \item Détermine l'équation.
    \item Résous l'équation.
    \item Vérifie ta réponse.
\end{enumerate}}
{2}
\end{exo}

\begin{exo}
{ Je pense à un nombre. J’ajoute 21 au double de ce nombre. Je trouve le même résultat qu’en retranchant 16 au triple de ce nombre. A quel nombre je pense ?
\begin{enumerate}
    \item Quelle est l'inconnue?
    \item Détermine l'équation.
    \item Résous l'équation.
    \item Vérifie ta réponse.
\end{enumerate}}
{3}
\end{exo}

\begin{exo}
{ Je fais la somme de 3 nombres entiers consécutifs. J'obtiens 2424. Quels sont ces trois nombres?\begin{enumerate}
    \item Quelle est l'inconnue?
    \item Détermine l'équation.
    \item Résous l'équation.
    \item Vérifie ta réponse.
\end{enumerate}}
{3}
\end{exo}

\begin{exo}
{Je prends 3 nombres entiers consécutifs. Si je fais la somme du triple du premier, du deuxième et du double du troisième j'obtiens 125. Quels sont ces 3 nombres?\begin{enumerate}
    \item Quelle est l'inconnue?
    \item Détermine l'équation.
    \item Résous l'équation.
    \item Vérifie ta réponse.
\end{enumerate}}
{3}
\end{exo}

\begin{exo}
  {Avec 25 pièces, toutes de 1 franc et 2 francs, j'ai une somme de 38 francs.
Combien ai-je de pièces de chaque sorte ?
\begin{enumerate}
    \item Quelle est l'inconnue?
    \item Exprime le nombre de pièces de 2.- en fonction du nombre de pièces de 1.-
    \item utilise le point b) pour poser une équation te permettant de résoudre le problème.
    \item Résous l'équation.
\end{enumerate}

}
 {3} 
\end{exo}


\begin{exo}
  {Manu a deux filles, Audrey et Chrisitne. Manu décide de donner de l'argent de poche pour les vacances à ses filles. Comme Christine est plus âgée, elle recevra le double d'Audrey. De plus, pour la récompenser de son aide précieuse dans les tâches ménagères, Christine recevra 5.- de plus d'argent de poche. 
  La somme totale est de 65.-
  Quelle somme Audrey et Christine ont-elles reçu?
\begin{enumerate}
 \item Quelle est l'inconnue?
    \item Détermine l'équation.
    \item Résous l'équation.
    \item Vérifie ta réponse.
\end{enumerate}

}
 {3} 
\end{exo}


%problèmes:

\end{document}

