\documentclass[a4paper,11pt]{report}
\usepackage[showexo=true,showcorr=false,showdegree=true]{../packages/coursclassed}
%Commenter ou enlever le commentaire sur la ligne suivante pour montrer le niveau
\toggletrue{montrerNiveaux}
%permet de gérer l'espacement entre les items des env enumerate et enumitem
\usepackage{enumitem}
\setlist[enumerate]{align=left,leftmargin=1cm,itemsep=10pt,parsep=0pt,topsep=0pt,rightmargin=0.5cm}
\setlist[itemize]{align=left,labelsep=1em,leftmargin=*,itemsep=0pt,parsep=0pt,topsep=0pt,rightmargin=0cm}
%permet de gerer l'espacement entre les colonnes de multicols
\setlength\columnsep{35pt}

\usepackage{numprint}

\begin{document}

%%%%%%%%%%%%%%%%% À MODIFIER POUR CHAQUE SERIE %%%%%%%%%%%%%%%%%%%%%%%%%%%%%
\newcommand{\chapterName}{Nombres et opérations}
\newcommand{\serieName}{Les puissances}

%%%%%%%%%%%%%%%%%% PREMIERE PAGE NE PAS MODIFER %%%%%%%%%%%%%%%%%%%%%%%%
% le chapitre en cours, ne pas changer au cours d'une série
\chapter*{\chapterName}
\thispagestyle{empty}

%%%%% LISTE AIDE MEMOIRE %%%%%%
\begin{amL}{\serieName}{
\item Puissance d'exposant positif (page 32)
\item Propriétés des puissances (page 34)
}\end{amL}

%%%%%%%%%%%%%%% DEBUT DE LA SERIE NE PAS MODIFIER %%%%%%%%%%%%%%%%%%%%%%%%%%%%%
\section*{\serieName}
\setcounter{page}{1}

%%%%%%%%%%% LES EXERCICES %%%%%%%%%%%%%%%%%%%%%%%%%%%%%%%%%%%%


%PER

%Comparaison, approximation, encadrement, représentation sur une droite et ordre de grandeur de nombres écrits sous la forme de puissances $a^b$ avec $a$ sous forme décimale dans $\mathbb{Q}$ et $b$ dans $\mathbb{N}$. 1-2-3

%Connaissance et utilisation des propriétés des opérations pour organiser et effectuer des calculs de manière efficace et pour donner des estimations : quatre propriétés des puissances dans $\mathbb{N}$. 2-3

%Utilisation de procédures de calcul réfléchi ou de calcul mental avec des carrés parfaits pour en extraire la racine. 1-2-3


\begin{QSJ}{57}{1}
\end{QSJ}

\begin{exol}{NO192}{50}{1} %volume d'un cube
\end{exol}

\begin{exop}{
Développe et calcule comme dans l'exemple.

Exemple : $2^3=\underline{~2\cdot2\cdot2~}=\underline{~8~}$
\begin{tasks}(2)
    \task $5^2=~\hrulefill~=$ \ligne{1.5}~
    \task $4^3=~\hrulefill~=$ \ligne{1.5}~
    \task $1^6=~\hrulefill~=$ \ligne{1.5}~
    \task $2^4=~\hrulefill~=$ \ligne{1.5}~
    \task $11^2=~\hrulefill~=$ \ligne{1.5}~
    \task $43^1=~\hrulefill~=$ \ligne{1.5}~
\end{tasks}
}{1}
\end{exop}

\begin{exop}{
Développe et calcule comme dans l'exemple.

Exemple : $3^4=\underline{~3\cdot3\cdot3\cdot3~}=\underline{~81~}$
\begin{tasks}(2)
    \task $3^3=~\hrulefill~=$ \ligne{1.5}~
    \task $25^1=~\hrulefill~=$ \ligne{1.5}~
    \task $2^6=~\hrulefill~=$ \ligne{1.5}~
    \task $1^4=~\hrulefill~=$ \ligne{1.5}~
    \task $4^4=~\hrulefill~=$ \ligne{1.5}~
    \task $12^2=~\hrulefill~=$ \ligne{1.5}~
\end{tasks}
}{1}
\end{exop}

\begin{exop}{
Développe et calcule comme dans l'exemple.

Exemple : $20^4=\underline{~20\cdot20\cdot20\cdot20~}=\underline{~\numprint{160000}~}$
\begin{tasks}(2)
    \task $110^2=~\hrulefill~=$ \ligne{1.5}~
    \task $900^2=~\hrulefill~=$ \ligne{1.5}~
    \task $30^3=~\hrulefill~=$ \ligne{1.5}~
    \task $10^4=~\hrulefill~=$ \ligne{1.5}~
    \task $400^2=~\hrulefill~=$ \ligne{1.5}~
    \task $1200^2=~\hrulefill~=$ \ligne{1.5}~
\end{tasks}
}{1}
\end{exop}

\begin{exop}{
Calcule.

\begin{tasks}(2)
    \task $100^4=~\hrulefill$
    \task $50^2=~\hrulefill$
    \task $600^2=~\hrulefill$
    \task $200^3=~\hrulefill$
    \task $10^5=~\hrulefill$
    \task $7000^2=~\hrulefill$
\end{tasks}
}{1}
\end{exop}

\begin{exop}{
Complète.
\begin{tasks}(2)
    \task $4\cdot4\cdot4\cdot4\cdot4\cdot4\cdot4\cdot4=4^{\ligne{0.5}}$
    \task $125=5^{\ligne{0.5}}$
    \task $1=60^{\ligne{0.5}}$
    \task $7^2=\ligne{1.5}$
    \task $2^{\ligne{0.5}}=32$
    \task $9^0=\ligne{1.5}$
    \task $2,5\cdot2,5\cdot2,5\cdot2,5=\ligne{1.5}^4$
    \task $0^{\ligne{0.5}}=0$
\end{tasks}
}{1}
\end{exop}

\begin{exop}{
Complète.
\begin{tasks}(2)
    \task $\ligne{1.5}^5=0$
    \task $1=17^{\ligne{0.5}}$
    \task $3,1\cdot3,1\cdot3,1=\ligne{1.5}^3$
    \task $7\cdot7\cdot7\cdot7\cdot7\cdot7=7^{\ligne{0.5}}$
    \task $64=4^{\ligne{0.5}}$
    \task $9^2=\ligne{1.5}$
    \task $2^{\ligne{0.5}}=128$
    \task $4,5^0=\ligne{1.5}$    
\end{tasks}
}{1}
\end{exop}


\begin{exop}{
Complète
\begin{tasks}(2)
    \task $(-3)\cdot(-3)\cdot(-3)\cdot(-3)=(-3)^{\ligne{0.5}}$
    \task $\numprint{1000}=10^{\ligne{0.5}}$
    \task $1=\ligne{1.5}^0$
    \task $12^2=\ligne{1.5}$
    \task $128=2^{\ligne{0.5}}$
    \task $-4^0=\ligne{1.5}$
    \task $-2\cdot2\cdot2\cdot2=\ligne{1.5}^4$
    \task $10^{\ligne{0.5}}=\numprint{100000}$
\end{tasks}
}{1}
\end{exop}

\begin{exop}{
Complète
\begin{tasks}(2)
    \task $(-3)^{\ligne{0.5}}=81$
    \task $\numprint{10000}=10^{\ligne{0.5}}$
    \task $\ligne{1.5}^7=1$
    \task $90^2=\ligne{1.5}$
    \task $2=2^{\ligne{0.5}}$
    \task $-\ligne{1.5}^7=0$
    \task $-30\cdot30\cdot30=\ligne{1.5}^3$
    
    \task $60^{\ligne{0.5}}=\numprint{3600}$
\end{tasks}
}{1}
\end{exop}

\begin{exof}{NO190}{58}{1} %op à trou
\end{exof}

\begin{resolu}{Calculer une puissance d'un nombre décimal}{
Développe et calcule.
\begin{tasks}(2)
    \task $0,1^2=\underline{~0,1\cdot0,1~}=\underline{~0,01~}$
    \task $0,5^3=\underline{0,5\cdot0,5\cdot0,5}=\underline{0,125}$
    \task $1,3^2=\underline{~1,3\cdot1,3~}=\underline{~1,69~}$
    \task $4,7^0=\underline{~1~}$
    \task* $0,1^6=\underline{0,1\cdot0,1\cdot0,1\cdot0,1\cdot0,1\cdot0,1}=\underline{\numprint{0,000001}}$
    \task* $0,02^4=\underline{0,02\cdot0,02\cdot0,02\cdot0,02}=\underline{\numprint{0,00000016}}$
\end{tasks}
\bigskip  
}{2}
\end{resolu}


\begin{exop}{
Développe et calcule. %fractions, décimaux
\begin{tasks}(2)
    \task $1,1^2=~\hrulefill~=$ \ligne{1.5}~
    \task $0,9^2=~\hrulefill~=$ \ligne{1.5}~
    \task $0,03^3=~\hrulefill~=$ \ligne{1.5}~
    \task $0,1^4=~\hrulefill~=$ \ligne{1.5}~
    \task $0,8^2=~\hrulefill~=$ \ligne{1.5}~
    \task $1,2^2=~\hrulefill~=$ \ligne{1.5}~
\end{tasks}
}{2}
\end{exop}


\begin{exop}{
Développe et calcule. %fractions, décimaux
\begin{tasks}(2)
    \task $0,1^3=~\hrulefill~=$ \ligne{1.5}~
    \task $0,03^2=~\hrulefill~=$ \ligne{1.5}~
    \task $0,2^5=~\hrulefill~=$ \ligne{1.5}~
    \task $0,2^4=~\hrulefill~=$ \ligne{1.5}~
    \task $0,8^2=~\hrulefill~=$ \ligne{1.5}~
    \task $0,01^2=~\hrulefill~=$ \ligne{1.5}~
\end{tasks}
}{2}
\end{exop}

\begin{resolu}{Déterminer le signe d'une puissance d'un nombre relatif}{

\faExclamationTriangle~ Une \textbf{puissance paire} d'un nombre négatif donne un \textbf{résultat positif} : \[ (-4)^{16}=\underbrace{\underbrace{(-4)\cdot(-4)}_{\textrm{positif}}\cdot\underbrace{(-4)\cdot(-4)}_{\textrm{positif}}\cdot\ldots\cdot\underbrace{(-4)\cdot(-4)}_{\textrm{positif}}}_{\textrm{positif}} \]

\faExclamationTriangle~ Une \textbf{puissance impaire} d'un nombre négatif donne un \textbf{résultat négatif} : \[ (-4)^{15}=\underbrace{\underbrace{(-4)\cdot(-4)}_{\textrm{positif}}\cdot\underbrace{(-4)\cdot(-4)}_{\textrm{positif}}\cdot\ldots\cdot\underbrace{(-4)\cdot(-4)}_{\textrm{positif}}\cdot(-4)}_{\textrm{négatif}} \]

\bigskip

Détermine si le résultat sera positif ($>0$) ou négatif ($<0)$.

\begin{tasks}(1)
    \task $(-5)^3$\underline{~$<0$~} car $(-5)\cdot(-5)\cdot(-5)=(+25)\cdot(-5)=-125$
    \task $(-5)^2$\underline{~$>0$~} car $(-5)\cdot(-5)=25$
    \task $-5^3$\underline{~$<0$~} car $-5\cdot5\cdot5=-25\cdot5=-125$
    \task $-5^2$\underline{~$<0$~} car $-5\cdot5=-25$

\end{tasks}

}{2}
\end{resolu}

\begin{exop}{
Détermine si le résultat sera positif ($>0$) ou négatif ($<0)$.

\begin{tasks}(3)
    \task $(-5)^2$ \hrulefill~
    \task $(-7)^3$ \hrulefill~
    \task $-1^6$ \hrulefill~
    \task $-1,5^{17}$ \hrulefill~
    \task $(-6)^{22}$ \hrulefill~
    \task $-5^1$ \hrulefill~
\end{tasks}
 }{2}
\end{exop}

\begin{exop}{
Détermine si le résultat sera positif ($>0$) ou négatif ($<0)$.

\begin{tasks}(3)
    \task $(-3)^0$ \hrulefill~
    \task $(+10)^8$ \hrulefill~
    \task $(-3,4)^3$ \hrulefill~
    \task $(-1)^{1}$ \hrulefill~
    \task $(+6,2)^{13}$ \hrulefill~
    \task $(-9)^4$ \hrulefill~
\end{tasks}
 }{2}
\end{exop}

\begin{exop}{
Détermine si le résultat sera positif ($>0$) ou négatif ($<0)$.

\begin{tasks}(3)
    \task $-4^1$ \hrulefill~
    \task $(+9)^7$ \hrulefill~
    \task $-(-4,28)^5$ \hrulefill~
    \task $-2^0$ \hrulefill~
    \task $-5,14^{21}$ \hrulefill~
    \task $-(-10)^{12}$ \hrulefill~
\end{tasks}
 }{2}
\end{exop}



\begin{resolu}{Calculer une puissance d'un nombre relatif}{
Développe et calcule.
\begin{tasks}(2)
    \task $(-9)^2=\underline{~(-9)\cdot(-9)~}=\underline{~81~}$
    \task $(-1)^3=\underline{(-1)\cdot(-1)\cdot(-1)}=\underline{-1}$
    \task $-2^4=\underline{-2\cdot2\cdot2\cdot2~}=\underline{~-16~}$
    \task $-11^2=\underline{~-11\cdot11~}=\underline{~-121~}$
    \task $(-43)^1=\underline{~-43~}$
    \task $-6^2=\underline{~-6\cdot6~}=\underline{~-36~}$
\end{tasks}  
}{2}
\end{resolu}


\begin{exop}{
Développe et calcule.
\begin{tasks}(2)
    \task $(-5)^2=~\hrulefill~=$ \ligne{1.5}~
    \task $(-1)^9=~\hrulefill~=$ \ligne{1.5}~
    \task $(-1)^6=~\hrulefill~=$ \ligne{1.5}~
    \task $-2^3=~\hrulefill~=$ \ligne{1.5}~
    \task $-10^2=~\hrulefill~=$ \ligne{1.5}~
    \task $(-15)^1=~\hrulefill~=$ \ligne{1.5}~    
\end{tasks}
}{2}
\end{exop}

\begin{exop}{
Développe et calcule.
\begin{tasks}(2)
    \task $-4^1=~\hrulefill~=$ \ligne{1.5}~
    \task $(+10)^6=~\hrulefill~=$ \ligne{1.5}~
    \task $(-1,5)^2=~\hrulefill~=$ \ligne{1.5}~
    \task $-2^0=~\hrulefill~=$ \ligne{1.5}~
    \task $-4^3=~\hrulefill~=$ \ligne{1.5}~
    \task $(-11,5)^1=~\hrulefill~=$ \ligne{1.5}~
\end{tasks}
}{2}
\end{exop}

\begin{exop}{
Développe et calcule.
\begin{tasks}(2)
    \task $-(-3)^0=~\hrulefill~=$ \ligne{1.5}~
    \task $(-1)^{13}=~\hrulefill~=$ \ligne{1.5}~
    \task $-(-4)^2=~\hrulefill~=$ \ligne{1.5}~
    \task $2^5=~\hrulefill~=$ \ligne{1.5}~
    \task $-8^2=~\hrulefill~=$ \ligne{1.5}~
    \task $-(-10)^5=~\hrulefill~=$ \ligne{1.5}~
\end{tasks}
}{2}
\end{exop}










\begin{exop}{
Complète
\begin{tasks}(2)
    
    \task $(-2)\cdot(-2)\cdot(-2)=\ligne{1.5}^3$
    \task $\ligne{1.5}^5=\numprint{0,00001}$
    \task $-9\cdot9\cdot9\cdot9\cdot9\cdot9=\ligne{1.5}^6$
    \task $0,49=(-0,7)^{\ligne{0.5}}$
    \task $\ligne{1.5}=1,2^2$
    \task $0,3^{\ligne{0.5}}=\numprint{0,027}$
    \task $1=1,3^{\ligne{0.5}}$
    \task $(-0,2)^3=\ligne{1.5}$    
\end{tasks}
}{2}
\end{exop}


\begin{exop}{
Complète
\begin{tasks}(2)
    \task $\numprint{0,000001}=0,01^{\ligne{0.5}}$
    \task $-1=\ligne{1.5}^5$
    \task $0,12^2=\ligne{1.5}$
    \task $-32=-2^{\ligne{0.5}}$
    \task $-4^0=\ligne{1.5}$
    \task $-2\cdot2\cdot2\cdot2=\ligne{1.5}^4$
    \task $10^{\ligne{0.5}}=\numprint{100000}$
    \task $-(-3)\cdot(-3)\cdot(-3)=\ligne{1.5}$
\end{tasks}
}{2}
\end{exop}

\begin{exop}{
Complète %relatifs, décimaux, rationnels
\begin{tasks}(2)
    \task $\dfrac{2}{3}\cdot\dfrac{2}{3}={\left(\dfrac{2}{3}\right)}^{\ligne{0.5}}$
    \task $0,64=0,4^{\ligne{0.5}}$
    \task $1=0,2^{\ligne{0.5}}$
    \task $\dfrac{4}{5}^2=\ligne{1.5}$
    \task ${\left(\dfrac{1}{2}\right)}^{\ligne{0.5}}=\dfrac{1}{32}$
    \task $\dfrac{9^0}{10}=\ligne{1.5}$
    \task $2,5\cdot2,5\cdot2,5\cdot2,5=\ligne{1.5}^4$
    \task $0,4^{\ligne{0.5}}=0$
\end{tasks}
}{3}
\end{exop}

\begin{exof}{NO191}{58}{1} %calcul, aussi relatifs et rationnels
\end{exof}




%\begin{exol}{NO194}{50}{1} %situation probleme, grains de riz
%\end{exol}
%\begin{exol}{NO195}{51}{1} %comparer diff écritures (intro aux propriétés)
%\end{exol}
\begin{exol}{NO196}{51}{2} %propriétés, trouver la règle
\end{exol}
\begin{exol}{NO197}{52}{2} %appliquer propriétés
\end{exol}

\begin{exop}{%biceps
Donne la réponse sous la forme d'une puissance $a^n$ chaque fois que c'est possible; sinon, effectue.
\begin{tasks}(2)
    \task $3^4\cdot3^2=$ \hrulefill \quad
    \task ${(2^7)}^3=$ \hrulefill \quad
    \task $10^5:10^2=$ \hrulefill \quad
    \task $1^8\cdot1^9=$ \hrulefill \quad
    \task $4^6\cdot4^2\cdot4^3=$ \hrulefill \quad
    \task $2^2+2^3=$ \hrulefill \quad
    \task $5^3\cdot3^3=$ \hrulefill \quad
    \task $7^3\cdot7^2:7^4=$ \hrulefill \quad
    \task $5^7\cdot2^7=$ \hrulefill \quad
    \task ${(3^2)}^4=$ \hrulefill \quad
\end{tasks}
}{2}    
\end{exop}

\begin{exop}{%biceps
Donne la réponse sous la forme d'une puissance $a^n$ chaque fois que c'est possible; sinon, effectue.
\begin{tasks}(2)
    \task $5^6\cdot5^2\cdot5^1=$ \hrulefill \quad
    \task ${(2^3)}^4=$ \hrulefill \quad
    \task $10^3\cdot10^{12}:10^5=$ \hrulefill \quad
    \task $2^3\cdot5^3 =$ \hrulefill \quad
    \task ${(8^2)}^3\cdot{(8^3)}^4 =$ \hrulefill \quad
    \task $3^{10}:3^7 =$ \hrulefill \quad
    \task ${(8^0\cdot8^3)}^4 =$ \hrulefill \quad
    \task $6^3\cdot2^3 =$ \hrulefill \quad
    \task $5^7:5^2\cdot5^3 =$ \hrulefill \quad
    \task ${(3^2\cdot3^3)}^4 =$ \hrulefill \quad
\end{tasks}
}{2}    
\end{exop}

\begin{exop}{%biceps
Donne la réponse sous la forme d'une puissance $a^n$ chaque fois que c'est possible; sinon, effectue.
\begin{tasks}(2)
    \task ${(5^7\cdot5^2)}^3=$ \hrulefill \quad
    \task $2^7\cdot2^3:2^4=$ \hrulefill \quad
    \task $5^2\cdot2^2=$ \hrulefill \quad
    \task ${(3^7)}^3:{(3^2)}^8=$ \hrulefill \quad
    \task $10^6\cdot10^3:10^5 =$ \hrulefill \quad
    \task $3^2\cdot3^2\cdot3^2\cdot3^2 =$ \hrulefill \quad
    \task $3^4-3^2 =$ \hrulefill \quad
    \task $10^7\cdot{(10^2)}^3 =$ \hrulefill \quad
    \task $2^{15}:2^7\cdot2^3 =$ \hrulefill \quad
    \task ${(3^3)}^2\cdot3^5=$ \hrulefill \quad
\end{tasks}
}{2}    
\end{exop}



\begin{exo}{%biceps
Donne la réponse sous la forme d'une puissance $a^n$ chaque fois que c'est possible; sinon, effectue.
\begin{tasks}(2)
    \task ${(3^2\cdot3^7)}^2\cdot{(3^4:3^1)}^3=$
    \task ${(2^4)}^3\cdot{(2^7:2^3)}^4=$
    \task $10^7\cdot10^5:{(10^4)}^3=$
    \task $\big(4^1\cdot{(4^4)}^4\big):{(4^3)}^2=$
    \task ${(2^3\cdot5^3)}^2\cdot10^5=$
    \task $8^7:{(8^2)}^3\cdot{(8^4)}^3=$
    \task ${(10^1\cdot10^2\cdot10^3)}^4=$
    \task ${(3^2\cdot3^3:3^3)}^2=$
    \task $\big(10^2\cdot{(10^3)}^3\big):10^4=$
    \task $(2^5:2^2)\cdot(10^3:5^3)=$
\end{tasks}
}{2}    
\end{exo}



\begin{exo}{%biceps
Donne la réponse sous la forme d'une puissance $a^n$ chaque fois que c'est possible; sinon, effectue.
\begin{tasks}(2)
    \task ${(12^7)}^2:{(12^3)}^4\cdot12^0=$
    \task $\big(9^2\cdot{(9^3)}^5\big):{(9^2)}^4=$
    \task $(2^3\cdot7^3):14^2=$
    \task ${(5^2\cdot5^7)}^3:{(5^3\cdot5^2)}^4=$
    \task $2^4+{(2^2)}^2+{(2^3)}^2 =$
    \task ${\big(10^7:(10^5:10^2)\big)}^3=$
    \task ${(7^7\cdot7^2)}^3\cdot(14^3:2^3)=$
    \task ${\big({(2^{10})}^3:{(2^3)}^5\big)}^2=$
    \task $\big({(11^2\cdot11^4)}^8:11^5\big):11^{42}=$
    \task ${(3^8)}^0\cdot(3^3:3^3)\cdot{(3^0)}^7=$ 
\end{tasks}
}{2}    
\end{exo}


\begin{exop}{%biceps
Donne la réponse sous la forme d'une puissance $a^n$ chaque fois que c'est possible; sinon, effectue.
\begin{tasks}(2)
    \task $(-2)^4\cdot(-2)^1=$
    \task $(-6)^7\cdot(-6)^1=$
    \task $\big((-4)^3\big)^2=$
    \task $(-1)^2\cdot(-1)^6=$
    \task $\big((-7)^4\big)^0=$
    \task $(-5)^2\cdot(-3)^2=$
    \task $(-5)^7:(-5)^2=$
    \task $\big((-3)^2\big)^5=$
    \task $(-6)^7:(-6)^2\cdot(-6)^1=$
    \task $(-8)^2\cdot(-8)^3\cdot(-8)^4=$
\end{tasks}
}{2}    
\end{exop}


\begin{exop}{%biceps
Donne la réponse sous la forme d'une puissance $a^n$ chaque fois que c'est possible; sinon, effectue.
\begin{tasks}(2)
    \task $(-2)^7\cdot(-2)^4\cdot(-2)^3=$
    \task $(-3)^4\cdot(-3)^{10}:(-3)^6=$
    \task $\big((-5)^2\big)^3=$
    \task $\big((-2)^4\big)^3:\big((-2)^2\big)^3=$
    \task $(-5)^2-(-5)^3=$
    \task $(-2)^7\cdot(-5)^7=$
    \task $\big((-8)^6\big)^2\cdot\big((-8)^3\big)^3=$
    \task $(-15)^3:(+5)^3=$
    \task $\big((-3)\cdot(+4)\big)^3=$
    \task $(-5)^3\cdot(-5)^3\cdot(-5)^3=$
\end{tasks}
}{2}    
\end{exop}

\begin{exof}{NO198}{58}{2} %corriger égalités
\end{exof}
\begin{exof}{NO199}{59}{2} %calcul (ou complète)
\end{exof}
\begin{exol}{NO200}{52}{2} %ordre croissant
\end{exol}
\begin{exol}{NO201}{52}{2} %par quel chiffre se termine 2^100
\end{exol}

\begin{exop}{%biceps
Compare avec les signes $<$ ; $=$ et $>$.
\begin{tasks}(2)
    \task $3^4$ \ligne{1} $3\cdot4$
    \task $4+4+4$ \ligne{1} $4^3$
    \task $100\cdot100$ \ligne{1} $10^5$
    \task $7^2$ \ligne{1} $2^7$
    \task $100^3$ \ligne{1} $10^5$
    \task $36^2$ \ligne{1} $6^4$
    \task $9^2$ \ligne{1} $12^4$
    \task $4^5$ \ligne{1} $16^2$
    \task $13^2$ \ligne{1} $11^2$
    \task $2^4\cdot2^3$ \ligne{1} $16\cdot16$
\end{tasks}
}{2}    
\end{exop}

\begin{exop}{%biceps
Compare avec les signes $<$ ; $=$ et $>$.
\begin{tasks}(2)
    \task $9^2$ \ligne{1} $16^4$
    \task $1^9$ \ligne{1} $14^0$
    \task $4^3\cdot4^2$ \ligne{1} $16\cdot16$
    \task $100^3$ \ligne{1} $100\cdot10^4$
    \task $0^{22}$ \ligne{1} $1^{22}$
    \task $27$ \ligne{1} $3^4$
    \task $25\cdot5^3$ \ligne{1} $5^5$
    \task $7^4$ \ligne{1} $4^7$
    \task $2^5$ \ligne{1} $2+2+2+2+2$
    \task $13^3$ \ligne{1} $9^3$
\end{tasks}
}{2}    
\end{exop}

\begin{exop}{%biceps
Compare avec les signes $<$ ; $=$ et $>$.
\begin{tasks}(2)
    \task $10^4$ \ligne{1} $100^2$
    \task $8\cdot9$ \ligne{1} $2^4\cdot3^2$
    \task $2^7$ \ligne{1} $7^2$
    \task $25\cdot25$ \ligne{1} $5^3\cdot5$
    \task $36^3$ \ligne{1} $6^6$
    \task $11^3$ \ligne{1} $9^3$
    \task $3^4$ \ligne{1} $9^2$
    \task $10^2\cdot100$ \ligne{1} $10^5$
    \task $5^3$ \ligne{1} $5\cdot3$
    \task $8^4$ \ligne{1} $8+8+8+8$
\end{tasks}
}{2}    
\end{exop}

\begin{exop}{ %biceps
Compare avec les signes $<$ ; $=$ et $>$.
\begin{tasks}(2)
    \task $12^3$ \ligne{1} $144\cdot12$
    \task $3^3\cdot3$ \ligne{1} $75$
    \task $10^5$ \ligne{1} $\numprint{100000}$
    \task $3^5$ \ligne{1} $27^2$
    \task $10^4$ \ligne{1} $10+10+10+10$
    \task $3^5$ \ligne{1} $5^3$
    \task $49\cdot49$ \ligne{1} $7^5$
    \task $12^4$ \ligne{1} $11^4$
    \task $7\cdot8$ \ligne{1} $8^7$
    \task $1000^2$ \ligne{1} $10^5$
\end{tasks}
}{2}    
\end{exop}

\begin{resolu}{Classer des nombres écrits sous forme de puissance}
{
Classe les nombres de chaque ligne dans l'ordre croissant.
\begin{tasks}(1)
    \task $3^2\qquad3^1\qquad3^8\qquad3^3\qquad3^5\qquad3^{11}$
    
    Réponse : Puisque les puissances sont de même base, il suffit de comparer les exposants.
    
    $3^1<3^2<3^3<3^5<3^8<3^{11} $
    
    \task $8^8\qquad7^8\qquad13^8\qquad2^8\qquad4^8\qquad20^8$

    Réponse : Puisque les exposants sont les mêmes, il suffit de comparer les bases.

    $2^8<4^8<7^8<8^8<13^8<20^8$
    
    \task $3^8\qquad8^1\qquad1^{22}\qquad5^8\qquad4^8\qquad8^2$

    Réponse : Compare les puissances de même base entre elles, et de même exposant entre elles. Effectue quelques calculs si nécessaire.

    $1^{22}<8^1<8^2<3^8<4^8<5^8$
    
    \task $2^{10}\qquad4^2\qquad3^2\qquad5^2\qquad10^3\qquad4^1$

    Réponse : Compare les puissances de même base entre elles, et de même exposant entre elles. Effectue quelques calculs si nécessaire.

    $4^1<3^2<4^2<5^2<10^3<2^{10}$
\end{tasks}

}{1}    
\end{resolu}


\begin{exo}{
Classe les nombres de chaque ligne dans l'ordre croissant.

\begin{tasks}(1)
    \task $9^3\qquad9^0\qquad9^{13}\qquad9^6\qquad9\qquad9^2$
    \task $8^8\qquad7^8\qquad13^8\qquad2^8\qquad4^8\qquad20^8$
    \task $5^2\qquad2^{9}\qquad4^2\qquad5^1\qquad6^2\qquad10^3$
    \task $3^6\qquad6^8\qquad8^2\qquad6^1\qquad1^{48}\qquad4^8$
\end{tasks}
}{2}
\end{exo}



\begin{exo}{
Classe les nombres de chaque ligne dans l'ordre décroissant.

\begin{tasks}(1)
    \task $3^9\qquad3^6\qquad3^{21}\qquad3^2\qquad3^1\qquad3^{11}$
    \task $6^6\qquad5^6\qquad1^6\qquad4^6\qquad10^6\qquad12^6$
    \task $7^2\qquad2^7\qquad3^7\qquad7^1\qquad2^2\qquad1^7$
    \task $4^3\qquad2^3\qquad3^4\qquad7^3\qquad8^3\qquad15^0$
\end{tasks}
}{2}
\end{exo}


\begin{exo}{
Classe les nombres de chaque ligne dans l'ordre décroissant.

\begin{tasks}(1)
    \task $5^8\qquad13^8\qquad9^8\qquad3^8\qquad1^8\qquad11^8$
    \task $7^4\qquad7^3\qquad7^8\qquad7^9\qquad7^{12}\qquad7^1$
    \task $2^5\qquad4^2\qquad3^2\qquad5^3\qquad1^{21}\qquad2^3$
    \task $7^4\qquad4^2\qquad2^5\qquad8^4\qquad3^4\qquad3^2$
\end{tasks}
}{2}
\end{exo}


\begin{exo}{
Classe les nombres de chaque ligne dans l'ordre croissant.

\begin{tasks}(1)
    \task $2^3\qquad3^2\qquad{(2^2)}^3\qquad2^2\cdot2^3\qquad2^{(2^3)}$
    \task $4^2\qquad4^1\cdot4^2\qquad2^4\qquad4^{(4^2)}\qquad{(4^3)}^2$
\end{tasks}
}{3}
\end{exo}



\begin{exo}{
Classe les nombres de chaque ligne dans l'ordre décroissant.
\begin{tasks}(1)
    \task $2^4\qquad3^4\qquad{(2^1)}^3\qquad2^4:2^3\qquad3^{(1^3)}$
    \task $1^2\qquad2^1\qquad{(2^1)}^2\qquad2^1\cdot2^1\qquad2^{(2^2)}$
\end{tasks}
}{3}
\end{exo}


\begin{exop}{%biceps
Calcule.
\begin{tasks}(2)
    \task $3^2\cdot2^2=$
    \task $6^2+6=$
    \task $12\cdot5^0=$
    \task $64:4^2=$
    \task $8-2^3=$
    \task $2^4+4^3=$
    \task $5\cdot4^2-2=$
    \task $100-7^2=$
    \task $5^3+5^2=$
    \task $10^4-10^3=$
\end{tasks}
}{1}    
\end{exop}


\begin{exop}{%biceps
Calcule.
\begin{tasks}(2)
    \task $100\cdot4^1=$
    \task $2^3\cdot2^3=$
    \task $2^2+2^5=$
    \task $5^2\cdot2^2=$
    \task $8^2-4\cdot3^3=$
    \task $2^7-2^3=$
    \task $72:6^2-1^7=$
    \task $3^3\cdot3=$
    \task $144\cdot0^7=$
    \task $3^3+3^3=$
\end{tasks}
}{1}    
\end{exop}

\begin{exop}{%biceps
Calcule.
\begin{tasks}(2)
    \task $4^4+4^2=$
    \task $5^2\cdot5-5=$
    \task $350-7^3=$
    \task $10^3+10^2+10^1=$
    \task $12^1:1^{12}=$
    \task $10^4:10^0=$
    \task $8^2:4^3=$
    \task $11^1\cdot12=$
    \task $7^2\cdot2-2=$
    \task $9^2-9=$
\end{tasks}
}{1}    
\end{exop}




\begin{exop}{%biceps
Calcule.
\begin{tasks}(2)
    \task $(-3)^2+3^4=$
    \task $2^3\cdot(-3)^2=$
    \task $5+(-5)\cdot(-5)^2=$
    \task $10^5+(-10)^2=$
    \task $5^2\cdot4-(-7)^0=$
    \task $(-5)^3\cdot(-4)^3=$
    \task $250-6^3=$
    \task $3\cdot(-3)^2-(-3)^1=$
    \task $-9^2:3^2=$
    \task $-10^6:2=$
\end{tasks}
}{2}    
\end{exop}

\begin{exop}{%biceps
Calcule.
\begin{tasks}(2)
    \task $(-2)^8:(-4)^4=$
    \task $3^2-(-3)^3=$
    \task $(-8)^2:2+5=$
    \task $7^2-(-7)^2=$
    \task $2\cdot(-12)^2=$
    \task $-100^0\cdot3^4=$
    \task $-200:(-5)^2=$
    \task $(-1)^2\cdot(-2)^1\cdot(-3)^2=$
    \task $50-4^2\cdot(-3)=$
    \task $-8^2+(-8)^2=$
\end{tasks}
}{2}    
\end{exop}


\begin{exo}{
Dans $4$ pays, on s'intéresse à $4$ village chacun. Dans chaque village, $4$ familles ont chacune $4$ chats qui tuent chacun $4$ souris par jour. \\ Combien de souris meurent chaque jour ?
}{2}    
\end{exo}


\begin{exo}{
Dans $2$ écoles différentes, $2$ enseignants par école font passer une évaluation à $2$ de leurs classes. Dans chaque classe, $2$ élèves laisse $2$ questions vides. \\ Combien y a-t-il de questions laissées vides au total ?
}{2}    
\end{exo}

\begin{exo}{
Sur le Papyrus Rhind écrit par le scribe Ahmès, on peut lire l'histoire suivante : Dans chacune des $7$ cabanes, il y a $7$ chats. Chaque chat surveille $7$ souris qui ont chacune $7$ épis de blé. Chaque épi est composé de $7$ grains. Combien de grains de blé y a-t-il en tout ?
}{2}
\end{exo}


\begin{exo}{
Marjorie arrive à son école à 8h du matin et répand une rumeur aux trois premières personnes qu'elle rencontre dans les couloirs. Celles-ci répandent à leur tour cette rumeur à trois nouvelles personnes à 8h10, et ainsi de suite.

Combien de personnes, Marjorie compris, auront connaissance de cette rumeur à 9h ?

}{2}    
\end{exo}


\begin{exo}{
Léonard arrive à son travail à 9h du matin et répand une rumeur aux dix premières personnes qu'il rencontre dans les bureaux. Celles-ci répandent à leur tour cette rumeur à dix nouvelles personnes durant l'heure qui suit, et ainsi de suite.

Après combien de temps la population de Genève, soit environ $\numprint{200000}$ personnes, seront au courant de la rumeur ?
}{2}    
\end{exo}


\begin{exo}{
L'introduction de $12$ lapins en Australie fut un désastre écologique. N'ayant aucun prédateur naturel, ils ont pu proliférer au point de menacer la faune et la flore locale. Avant que leur progression ne soit freinée, la population de lapins avait atteint $\numprint{600000000}$ individus.

En supposant que la population de lapins doublait tous les deux ans, combien de temps a-t-il fallu au moins pour que la population atteigne ce nombre étourdissant ?
}{2}
\end{exo}

\begin{exo}{
Une feuille de papier mesure \tunit{0,1}{\milli m} d’épaisseur. La distance entre la Terre et la Lune est d’environ \tunit{384400}{\kilo m}.
En pliant une feuille de papier en deux, on double son épaisseur. En la repliant en quatre,
l’épaisseur quadruple et ainsi de suite. Combien de fois faut-il plier la feuille de papier pour
obtenir la distance Terre-Lune ?
}{2}
\end{exo}





\end{document}
