\documentclass[a4paper,11pt]{report}
\usepackage[showexo=true,showcorr=true]{packages/coursclasse}
%Commenter ou enlever le commentaire sur la ligne suivante pour montrer le niveau
\toggletrue{montrerNiveaux}
%permet de gérer l'espacement entre les items des env enumerate et enumitem
\usepackage{enumitem}
\setlist[enumerate]{align=left,leftmargin=1cm,itemsep=10pt,parsep=0pt,topsep=0pt,rightmargin=0.5cm}
\setlist[itemize]{align=left,labelsep=1em,leftmargin=*,itemsep=0pt,parsep=0pt,topsep=0pt,rightmargin=0cm}
%permet de gerer l'espacement entre les colonnes de multicols
\setlength\columnsep{20pt}

\begin{document}
%%%%%%%%%%%%%%%%% À MODIFIER POUR CHAQUE SERIE %%%%%%%%%%%%%%%%%%%%%%%%%%%%%
\newcommand{\chapterName}{Espace}
\newcommand{\serieName}{Représentation de solides}


%%%%%%%%%%%%%%%%%% PREMIERE PAGE NE PAS MODIFER %%%%%%%%%%%%%%%%%%%%%%%%
% le chapitre en cours, ne pas changer au cours d'une série
\chapter*{\chapterName}
\thispagestyle{empty}

%%%%% LISTE AIDE MEMOIRE %%%%%%
\begin{amL}{\serieName}{
\item Représentation d’un objet dans l’espace (page 142).
\item Prisme droit (page 145).
\item Parallélépipède rectangle ou pavé droit (page 146).
}
\end{amL}
%%%%%%%%%%%%%%% DEBUT DE LA SERIE NE PAS MODIFIER %%%%%%%%%%%%%%%%%%%%%%%%%%%%%
\section*{\serieName}
\setcounter{page}{1}

\begin{resolu}{Titre de l'exercice résolu}
{Une consigne 
\begin{enumerate}
\item 75 et 234 sont des {\color{blue}nombres}.
\end{enumerate}}{2}
\end{resolu}

\begin{exo}{
Un exercice à faire dans le cahier.
Pose et effectue les calculs suivants :
\begin{enumerate}
\begin{multicols}{2}
\item $64,3+46,9+84,5 =$
\item $1'059,3-774,6=$
\item $4,73\cdot 24,7 =$
\item $70,04 : 1,2 =$
\end{multicols}
\end{enumerate}
\smallskip}{0}
\end{exo}

\begin{exoc}{
	test
}
\end{exoc}

\begin{exop}{
Un exercice à faire sur la série.
Pose et effectue les calculs suivants :
\begin{enumerate}
\begin{multicols}{2}
\item $64,3+46,9+84,5 =$
\item $1'059,3-774,6=$
\item $4,73\cdot 24,7 =$
\item $70,04 : 1,2 =$
\end{multicols}
\end{enumerate}
\smallskip}{0}
\end{exop}

\begin{exof}{NO numéro}{33}{0}
\end{exof}

\begin{exol}{NO numéro}{38}{0}
\end{exol}

\begin{FLP}{67}{0}
\end{FLP}

\begin{QSJ}{78}{0}
\end{QSJ}


\end{document}
